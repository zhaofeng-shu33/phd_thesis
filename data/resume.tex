% !TeX root = ../thuthesis-example.tex

\begin{resume}

  \section*{个人简历}

  1994 年 7 月 11 日出生于山东省枣庄市。

  2013 年 9 月考入清华大学数学系数学与应用数学专业,2017 年 7 月本科毕业并获得理学学士学位。

  2017 年 9 月免试进入清华大学电子工程系攻读信息与通信工程工学博士至今。


  \section*{在学期间完成的相关学术成果}

  \subsection*{学术论文}
  
  \begin{achievements}
    \item Zhao~F, Ye~M, Huang~S~L.
    \newblock Exact recovery of stochastic block model by ising
      model\allowbreak[J].
    \newblock Entropy, 2021{\natexlab{a}}, 23\allowbreak (1): 65.
    (SCI 收录, 检索号:PV6XT, 影响因子:2.738)
    \item Zhao~F, Huang~S~L.
    \newblock On the universally optimal activation function for a class of
      residual neural networks\allowbreak[J].
    \newblock AppliedMath, 2022, 2\allowbreak (4): 574-584.
    \item Zhao~F, Ma~F, Li~Y, et~al.
    \newblock Info-detection: An information-theoretic approach to detect
      outlier\allowbreak[C]//\allowbreak
    International Conference on Neural Information Processing.
    \newblock Springer, 2019: 489-496.
    (EI 收录, 检索号: 20200508099155)
    \item Zhao~F, Niu~X, Huang~S~L, et~al.
    \newblock Reproducing scientific experiment with cloud
      devops\allowbreak[C]//\allowbreak
    2020 IEEE World Congress on Services (SERVICES).
    \newblock IEEE, 2020: 259-264.
    (EI 收录, 检索号: 20210309773638)
    \item Zhao~F, Sima~J, Huang~S~L.
    \newblock On the optimal error rate of stochastic block model with symmetric
      side information\allowbreak[C]//\allowbreak
    2021 IEEE Information Theory Workshop (ITW).
    \newblock IEEE, 2021{\natexlab{b}}: 1-6.
    (EI 收录, 检索号: 20220511542416)
    \item Zhao~F, Chen~X, Wang~J, et~al.
    \newblock Performance-guaranteed ode solvers with complexity-informed neural
      networks\allowbreak[C]//\allowbreak
    The Symbiosis of Deep Learning and Differential Equations.
    \newblock 2021{\natexlab{c}}.
    (已发表)
    \item Sima~J, Zhao~F, Huang~S~L.
    \newblock Exact recovery in the balanced stochastic block model with side
      information\allowbreak[C]//\allowbreak
    2021 IEEE Information Theory Workshop (ITW).
    \newblock IEEE, 2021: 1-6.
    (EI 收录, 检索号: 20220511542376)
  \end{achievements}


%  \subsection*{专利}

%  \begin{achievements}
%    \item 任天令, 杨轶, 朱一平, 等. 硅基铁电微声学传感器畴极化区域控制和电极连接的方法: 中国, CN1602118A[P]. 2005-03-30.
%    \item Ren T L, Yang Y, Zhu Y P, et al. Piezoelectric micro acoustic sensor based on ferroelectric materials: USA, No.11/215, 102[P]. (美国发明专利申请号.)
%  \end{achievements}

\end{resume}
