\chapter{第 \ref{chap:sbmsi} 章补充内容}
\section{引理\ref{lem:I_plus_expression} 的证明}
在证明引理
\ref{lem:I_plus_expression}  之前,
我们先列举泊松分布的一个重要的性质。
\begin{lemma}\label{lem:poisson_property}
假设 $P_0 \sim \Pois(c)$, $P_1 \sim \Pois(d)$.
且 $P_{\lambda}$ 定义成
$$
P_{\lambda}(x) = \frac{P_0^{1-\lambda}(x) P_1^{\lambda} (x)}
{\sum_{x \in \mathcal{X}}P_0^{1-\lambda}(x) P_1^{\lambda} (x)}
$$
则我们有
\begin{enumerate}
    \item $P_{\lambda} \sim \Pois(c^{1-\lambda} d^{\lambda})$
    \item $D_{\mathrm{KL}}(P_{\lambda}||P_0) = 
    \lambda c^{1-\lambda}d^{\lambda}\log(\frac{d}{c}) + c-c^{1-\lambda}d^{\lambda}$
    \item $D_{\mathrm{KL}}(P_{\lambda}||P_1) = (1-\lambda)
    c^{1-\lambda}d^{\lambda}\log(\frac{c}{d})
    + d-c^{1-\lambda}d^{\lambda}$
\end{enumerate}
\end{lemma}
上面的第2、3条性质可由两个泊松分布的KL散度
公式导出。
\begin{proof}[引理\ref{lem:I_plus_expression} 的证明]
由 切尔诺夫信息的定义 \eqref{eq:D_star_lambda_star} 式
可得:
$$
I_+ = D(P_{\lambda^*} || \Pois(\frac{a}{2}))
+ D(P_{1-\lambda^*} || \Pois(\frac{b}{2}))
+ \gamma D(P_{\lambda^*} || P_0)
$$
在引理\ref{lem:poisson_property}
中取$c=\frac{a}{2}, d=\frac{b}{2}$
并应用第2、3条性质可得
\begin{align*}
    I_+ = \left[\lambda^* c^{1-\lambda^*}d^{\lambda^*}
    \log(\frac{d}{c})+ c-c^{1-\lambda^*}d^{\lambda^*}
    \right]
+ \left[\lambda^* c^{\lambda^*}d^{1-\lambda^*}\log(\frac{c}{d})
+ d - c^{\lambda^*}d^{1-\lambda^*}\right]
+ \gamma D(P_{\lambda^*} || P_0)
\end{align*}
即\eqref{equation:I+} 式。
$\lambda^*$ 可由\eqref{eq:C_P_1_P_2_another} 式
进行计算。目标函数为:
\begin{align*}
&\log\left(\sum_{k=0}^{+\infty} \left(\frac{c^k\exp(-c)}{k!}
\right)^{1-\lambda}
\left(\frac{d^k\exp(-d)}{k!} \right)^{\lambda}
\right) \\
& + \log\left(\sum_{k=0}^{+\infty} \left(\frac{c^k\exp(-c)}{k!}
\right)^{\lambda}
\left(\frac{d^k\exp(-d)}{k!} \right)^{1-\lambda}
\right)+
\gamma\log(\sum_{x\in \mathcal{X}}P^{1-\lambda}_0(x) P^{\lambda}_1(x)
)\\
& = c^{1-\lambda} d^{\lambda} 
+ d^{\lambda} c^{1-\lambda} -c -d +
\gamma\log(\sum_{x\in \mathcal{X}}P^{1-\lambda}_0(x) P^{\lambda}_1(x)
)
\end{align*}
极小化上式即等价于\eqref{eq:lambda}式。
\end{proof}

\section{定理 \ref{thm:Pe_new} 的证明}
\begin{proof}[定理\ref{thm:Pe_new} 的证明]

  
对于
某个特定的节点标签
$x^*$,
若
\eqref{eq:mle} 式中的 ML 算法
无法精确恢复 
$x^*$,则
存在 $x\neq x^*$ 使得 $P(y,z|x) > P(y,z|x^*)$。
令 $F_k$ 表示在 $x$ 和 $x^*$ 之间有 $k$ 个不同的对
这个事件,即:
    \begin{equation}\label{eq:Fk}
    F_k:=\{\exists y \in \{\pm 1\}^n |
    \Dist(x, x^*)=2k,
    P(y,z|x) > 
    P(y,z|x^*) \}
    \end{equation}
    由于
    $x$ 要 满足约束
    $\sum_{i=1}^n x_i=0$,
    $\Dist(x, x^*)$
    只能取偶数值。
    在
    $P(y,z|x) > P(y,z|x^*)$
    两边同时取
    对数
    , 并利用
    \eqref{eq:lh} 式,
    可得如下等价不等式:
    \begin{equation}\label{eq:ein}
    \sum_{i=1}^{km}
    \left(\log \frac{P_1(y_{1i})}
    {P_0(y_{1i})}
    + \log \frac{P_0(y_{2i})}
    {P_1(y_{2i})}
    \right)
    \geq \log \frac{p(1-q)}{q(1-p)} \sum_{i=1}^{k(n-2k)}(z_{i} - z'_{i})
    \end{equation}
    
    其中 $y_{1i}(y_{2i})$ 分别采样自
    $P_0(P_1)$,
    且有 $z_{i} \sim \Bern(p), z'_{i} \sim \Bern(q)$。
    
    We denote the event described by \eqref{eq:ein} as $A_k$,
    and each $F_k$ can be regarded as the union of $\binom{n/2}{k}^2$ events
    of $A_k$ for different node indexes.
    
    To obtain an upper bound of $P(A_k)$, we further define several empirical distributions as follows:
    \begin{align*}
    P(\widetilde{X}_j = u) &= \frac{1}{km} \sum_{i=1}^{km} \mathbf{1}[x_{ji} = u] \textrm{ for } u \in \mathcal{X}, j=1,2 \\
    P(\widetilde{Z} = u) &= \frac{1}{k(n-2k)}\sum_{i=1}^{k(n-2k)} \mathbf{1}[z_i = u], u \in \{0, 1\}
    \end{align*}
    and $\widetilde{Z}'$ is defined similarly. Then
    \eqref{eq:ein} is transformed as
    \begin{align}
    &m\left[\sum_{x\in \mathcal{X}}P_{\widetilde{X}_1}(x)\log\frac{p_1(x)}{p_0(x)}
    +\sum_{x\in \mathcal{X}}P_{\widetilde{X}_2}(x)\log\frac{p_0(x)}{p_1(x)}\right] +(n-2k)\notag \\
    &\left[\sum_{z\in\{0,1\}} P_{\widetilde{Z}}(z) \log \frac{p_{B_q}(z)}{p_{B_p}(z)}
    + \sum_{z\in\{0,1\}} P_{\widetilde{Z}'}(z) \log \frac{p_{B_p}(z)}{p_{B_q}(z)}\right] \geq 0 \label{eq:mnk}
    \end{align}
\end{proof}
