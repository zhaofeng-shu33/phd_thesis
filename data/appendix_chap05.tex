\chapter{第 \ref{chap:sbmsi} 章补充内容}
\section{引理\ref{lem:I_plus_expression} 的证明}
在证明引理
\ref{lem:I_plus_expression}  之前,
我们先列举泊松分布的一个重要的性质。
\begin{lemma}\label{lem:poisson_property}
假设 $P_0 \sim \Pois(c)$, $P_1 \sim \Pois(d)$.
且 $P_{\lambda}$ 定义成
$$
P_{\lambda}(x) = \frac{P_0^{1-\lambda}(x) P_1^{\lambda} (x)}
{\sum_{x \in \mathcal{X}}P_0^{1-\lambda}(x) P_1^{\lambda} (x)}
$$
则我们有
\begin{enumerate}
    \item $P_{\lambda} \sim \Pois(c^{1-\lambda} d^{\lambda})$
    \item $D_{\mathrm{KL}}(P_{\lambda}||P_0) = 
    \lambda c^{1-\lambda}d^{\lambda}\log(\frac{d}{c}) + c-c^{1-\lambda}d^{\lambda}$
    \item $D_{\mathrm{KL}}(P_{\lambda}||P_1) = (1-\lambda)
    c^{1-\lambda}d^{\lambda}\log(\frac{c}{d})
    + d-c^{1-\lambda}d^{\lambda}$
\end{enumerate}
\end{lemma}
上面的第2、3条性质可由两个泊松分布的KL散度
公式导出。
\begin{proof}[引理\ref{lem:I_plus_expression} 的证明]
由 切尔诺夫信息的定义 \eqref{eq:D_star_lambda_star} 式
可得:
$$
I_+ = D(P_{\lambda^*} || \Pois(\frac{a}{2}))
+ D(P_{1-\lambda^*} || \Pois(\frac{b}{2}))
+ \gamma D(P_{\lambda^*} || P_0)
$$
在引理\ref{lem:poisson_property}
中取$c=\frac{a}{2}, d=\frac{b}{2}$
并应用第2、3条性质可得
\begin{align*}
    I_+ = \left[\lambda^* c^{1-\lambda^*}d^{\lambda^*}
    \log(\frac{d}{c})+ c-c^{1-\lambda^*}d^{\lambda^*}
    \right]
+ \left[\lambda^* c^{\lambda^*}d^{1-\lambda^*}\log(\frac{c}{d})
+ d - c^{\lambda^*}d^{1-\lambda^*}\right]
+ \gamma D(P_{\lambda^*} || P_0)
\end{align*}
即\eqref{equation:I+} 式。
$\lambda^*$ 可由\eqref{eq:C_P_1_P_2_another} 式
进行计算。目标函数为:
\begin{align*}
&\log\left(\sum_{k=0}^{+\infty} \left(\frac{c^k\exp(-c)}{k!}
\right)^{1-\lambda}
\left(\frac{d^k\exp(-d)}{k!} \right)^{\lambda}
\right) \\
& + \log\left(\sum_{k=0}^{+\infty} \left(\frac{c^k\exp(-c)}{k!}
\right)^{\lambda}
\left(\frac{d^k\exp(-d)}{k!} \right)^{1-\lambda}
\right)+
\gamma\log(\sum_{x\in \mathcal{X}}P^{1-\lambda}_0(x) P^{\lambda}_1(x)
)\\
& = c^{1-\lambda} d^{\lambda} 
+ d^{\lambda} c^{1-\lambda} -c -d +
\gamma\log(\sum_{x\in \mathcal{X}}P^{1-\lambda}_0(x) P^{\lambda}_1(x)
)
\end{align*}
极小化上式即等价于\eqref{eq:lambda}式。
\end{proof}

\section{定理 \ref{thm:Pe_new} 的证明}

在开始定理\ref{thm:Pe_new} 的正式证明之前,
我们先推导
最大似然算法无法精确恢复社群结构这一事件
对应的不等式。

对于
某个特定的节点标签
$x^*$,
若
\eqref{eq:mle} 式中的 ML 算法
无法精确恢复 
$x^*$,则
存在 $x\neq x^*$ 使得 $P(y,z|x) > P(y,z|x^*)$。
令 $F_k$ 表示在 $x$ 和 $x^*$ 之间有 $k$ 个不同的对
这个事件,即:
    \begin{equation}\label{eq:Fk}
    F_k:=\{\exists x \in \{\pm 1\}^n |
    \Dist(x, x^*)=2k,
    P(y,z|x) > 
    P(y,z|x^*) \}
    \end{equation}
    由于
    $x$ 要 满足约束
    $\sum_{i=1}^n x_i=0$,
    $\Dist(x, x^*)$
    只能取偶数值。
    在
    $P(y,z|x) > P(y,z|x^*)$
    两边同时取
    对数
    , 并利用
    \eqref{eq:lh} 式,
    可得如下等价不等式:
    \begin{equation}\label{eq:ein}
    \sum_{i=1}^{km}
    \left(\log \frac{P_1(y_{1i})}
    {P_0(y_{1i})}
    + \log \frac{P_0(y_{2i})}
    {P_1(y_{2i})}
    \right)
    \geq \log \frac{p(1-q)}{q(1-p)} \sum_{i=1}^{k(n-2k)}(z_{i} - z'_{i})
    \end{equation}
    
    其中 $y_{1i}(y_{2i})$ 分别采样自
    $P_0(P_1)$,
    且有 $z_{i} \sim \Bern(p), z'_{i} \sim \Bern(q)$。
    
    我们把 \eqref{eq:ein} 式所描述的事件
    叫做 $A_k$,
    则每个 $F_k$ 
    可看成
     $\binom{n/2}{k}^2$ 个
    对应不同节点标签的
    $A_k$ 事件的并集。
    
    为
    获得 $P(A_k)$
    的上界, 
    我们首先定义如下经验分布:
    \begin{align}
    P(\widetilde{Y}_j = u) &=
    \frac{1}{km} \sum_{i=1}^{km}
    \mathbf{1}[y_{ji} = u] \textrm{ for } u \in \mathcal{Y}, j=1,2 
    \notag \\
    P(\widetilde{Z} = u) &= \frac{1}{k(n-2k)}\sum_{i=1}^{k(n-2k)} \mathbf{1}[z_i = u], u \in \{0, 1\}
    \label{eq:zu}
    \end{align}
     $\widetilde{Z}'$ 
     的定义类似 \eqref{eq:zu}。
 则
    \eqref{eq:ein} 式可化为:
    \begin{align}
    &m\left[\sum_{y\in \mathcal{Y}}
    P_{\widetilde{Y}_1}(y)\log\frac{P_1(y)}
    {P_0(y)}
    +\sum_{y\in \mathcal{Y}}
    P_{\widetilde{Y}_2}(y)\log\frac{P_0(y)}
    {P_1(y)}
    \right] +(n-2k)\notag \\
    &\left[\sum_{z\in\{0,1\}}
    P_{\widetilde{Z}}(z) \log \frac{P_{B_q}(z)}
    {P_{B_p}(z)}
    + \sum_{z\in\{0,1\}} 
    P_{\widetilde{Z}'}(z) \log \frac{P_{B_p}(z)}
    {P_{B_q}(z)}\right] \geq 0 \label{eq:mnk}
    \end{align}
    其中 $P_{B_p}$、$P_{B_q}$ 分别表示参数为$p,q$
    的伯努利随机变量的概率质量函数。
    此外,为证明 \eqref{eq:PeMainL} 给出的
    下界,我们还需要如下两个引理。
    \begin{lemma}\label{lem:single_lower}
        对于 
         在 \eqref{eq:ein}式中给出的事件 $A_1$ 
          (取 $k=1$),
        我们有如下的不等式估计
        \begin{equation}\label{eq:pa1}
        P(A_1) \geq \exp(-(\gamma D_{1/2}(p_0||p_1) + (\sqrt{a}-\sqrt{b})^2 + o(1))\log n )
        \end{equation}
        \end{lemma}
        \begin{proof}[引理\ref{lem:single_lower}的证明] 
当 $k=1$ 时,
不等式 \eqref{eq:ein} 可写成
$\sum_{i=1}^{n-2} (z'_i - z_i)
\geq \epsilon$
其中
\begin{align*}
\epsilon := &\frac{m}{\log a/b}\cdot 
\Bigl[D(P_{\widetilde{Y}_{1}} || P_1) 
- D(P_{\widetilde{Y}_{1}} || P_0) \\
&+D(P_{\widetilde{Y}_{2}} || P_0) -
D(P_{\widetilde{Y}_{2}} || P_1)\Bigr],
\end{align*}
令 $P_{\widetilde{Y}^{*}_1}$
和 $P_{\widetilde{Y}^{*}_2}$
取相同的分布
$P(Y=y)=\frac{\sqrt{P_0(y)P_1(y)}}
{ \sum_{y\in \mathcal{Y}}
\sqrt{P_0(y) P_1(y)}} $,
从而使得 $\epsilon =0$。
对于这组特殊选取的
$P_{\widetilde{Y}^{*}_1}$
和 $P_{\widetilde{Y}^{*}_2}$,
我们使用 Sanov 定理可得
\begin{align*}
&P(A_1)
\geq\frac{1}
{(m+1)^{2|\mathcal{Y}|}}
\exp \left(-m(D(P_{\widetilde{Y}^*_1} || P_0)
+ D(P_{\widetilde{Y}^*_2} || P_1))
\right)
\cdot P\left(\sum_{i=1}^{n-2} (z'_i - z_i) \geq 0\right)\\
& = \exp(-\log n (\gamma D_{1/2}(P_0||P_1)+o(1))) P(\sum_{i=1}^{n-2} (z'_i - z_i) \geq 0).
\end{align*}
由 \citet{abbe2015exact}
中的引理 4,
$P(\sum_{i=1}^{n-2} (z'_i - z_i) \geq 0)$ 下界为
 $n^{-(\sqrt{a} - \sqrt{b})^2 + o(1)}$。
因此得到 \eqref{eq:pa1} 式。
        \end{proof}

\begin{lemma}\label{lem:p0p1}
    令 $P_0,P_1$ 
    是两个定义在字母集
    $\mathcal{X}$ 上的概率分布,
    则下面的不等式成立:
    \begin{equation}\label{eq:32}
        \left(\sum_{x\in \mathcal{X}}
        P^{\frac{1}{3}}_0(x)
        P^{\frac{2}{3}}_1(x)\right)^3
        \leq \left(\sum_{x\in \mathcal{X}}
        \sqrt{P_0(x) P_1(x)}\right)^2
    \end{equation}
\end{lemma}

\begin{proof}
    令 $f(x)=P^{\frac{1}{3}}_0(x)
    P^{\frac{1}{3}}_1(x)$,
    $g(x) = P^{\frac{1}{3}}_1(x)$,
    $p=\frac{3}{2}, q=3$。
    容易验证 $\frac{1}{p} + \frac{1}{q}=1$。
    \newglossaryentry{holder_ieq}{name=赫尔德不等式, description={Hölder's inequality}}

    由 \gls{holder_ieq}
    :
    \begin{equation}\label{eq:holder}
        (\sum_{x\in\mathcal{X}}f(x)g(x))\leq (\sum_{x\in\mathcal{X}} f^p(x))^{\frac{1}{p}}
        (\sum_{x\in\mathcal{X}} g^q(x))^{\frac{1}{q}}
    \end{equation}
    因为 $\sum_{x\in\mathcal{X}} P_1(x)=1$, \eqref{eq:holder} 蕴含
    \eqref{eq:32}。
\end{proof}
    
\begin{proof}[定理\ref{thm:Pe_new} 的证明]

以下我们使用切尔诺夫不等式来给出 $P(A_k)$
的上界,即我们证明:
$$
P(A_k) \leq n^{-k\theta^*_k} 
\,\textrm{ 其中 }\, \theta^*_k=\gamma D_{1/2}(P_0||P_1)
+\left(1-\frac{2k}{n} \right)\left(\sqrt{a}-\sqrt{b}
\right)^2
$$
由 $A_k$ 的定义式 \eqref{eq:ein} 我们有
\begin{align*}
P(A_k) &\leq \mathbb{E}
\left[\exp \left( s\sum_{i=1}^{km}
\left( \log \frac{P_1(y_{1i})}
    {P_0(y_{1i})}
+ \log \frac{P_0(y_{2i})}
    {P_1(y_{2i})} \right)
\right)\right]\cdot \mathbb{E}
\left[\exp\left(s\log \frac{a}{b}\sum_{i=1}^{k(n-2k)} (z'_i - z_i )\right)
\right] \\
& \stackrel{(a)}{=}
\left(\sum_{x\in \mathcal{X}}
P_0^{1-s}(x)P_1^{s}(x)\right)^{km}
\left(\sum_{x\in \mathcal{X}} P_1^{1-s}(x) P_0^{s}(x)
\right)^{km}\\
&   \cdot \exp\left(
    k\log n \left(1-\frac{2k}{n} \right)
    (-a-b+a^sb^{1-s}+b^sa^{1-s} +o(1)) 
    \right)
\end{align*}
其中 $(a)$ 由各样本相互独立可得。 取 $s=\frac{1}{2}$,
则我们有
$P(A_k) \leq  n^{-k(\theta^*_k+o(1))}$.

当 $k \geq \frac{n}{\sqrt{\log n}}$ 时,
由引理\ref{lem:sigmaX},
$P(F_k)$ 指数衰减。
对于 $2\leq k < \frac{n}{\sqrt{\log n}}$
的情形,
错误概率通过如下的方式进行分析:
由布尔不等式,
我们可以给出 $P(F_k)$ 的上界为
\begin{equation}\label{eq:FAk}
P(F_k) \leq \binom{n/2}{k}^2 P(A_k)
\end{equation}
and by $\binom{n}{k} \leq (\frac{ne}{k})^k$,
我们可以进一步 向上放缩 $P(F_k)$ 为
\begin{align*}
P(F_k) & \leq \left(\frac{ne}{2k}\right)^{2k} P(A_k) 
\leq \exp(k(2\log(\frac{ne}{2k})-\theta_k^* \log n) + o(1))
\end{align*}
\begin{align*}
P_e &\leq P(F_1)+(1+o(1))\sum_{k=2}^{\frac{n}{\sqrt{\log n}}} P(F_k) \\
& \leq (1+o(1))\cdot \sum_{k=2}^{\frac{n}{\sqrt{\log n}}}
\exp\left(k(-\mu \log n + \frac{2k}{n} \log n(\sqrt{a} - \sqrt{b})^2 - 2\log 2k + 2)
\right)
\end{align*}
其中 $\mu$ 定义为
\begin{equation}\label{eq:mu_def}
	\mu = (\sqrt{a} - \sqrt{b})^2-2 + \gamma D_{1/2}(p_0||p_1) > 0	
\end{equation}
对于 $P(F_1)$, 我们有 $P(F_1)\leq (n/2)^2
P(A_1)\leq \frac{1}{4}n^{-\mu+o(1)}$。
对于 $2\leq k \leq \frac{n}{\sqrt{\log n}}$,
使用不等式
$$
\frac{2k}{n}(\sqrt{a} - \sqrt{b})^2\log n -2\log2k+2\leq  C\sqrt{\log n}
$$
我们可以得到
\begin{align*}
P_e &\leq \frac{1}{4}n^{-\mu + o(1)} +(1+o(1)) \sum_{k=2}^{\frac{n}{\sqrt{\log n}}} \exp(k((-\mu + o(1)) \log n )) \\
& =\frac{1}{4}n^{-\mu + o(1)}+(1+o(1)) \frac{n^{-\mu + o(1)}}{1-n^{-\mu + o(1)}} \\
&= (\frac{1}{4}+o(1))n^{-\mu + o(1)}
\end{align*}
因此, \eqref{eq:PeMain} 成立。

为证明误差速率下界表达式
\eqref{eq:PeMainL},
我们首先定义事件 $A_{ij}$ 为
$P(y,z|x) > P(y,z|x^*)$,
其中 $x$ 定义为
$x_s=-x^*_s,
s=i,j$,除此之外
$x_s=x^*_s$
。

另外定义 $S:=\{(i,j)| 1\leq i < j\leq n,
x_i^*=-x_j^*
\}$。
则容易得到 $\cup_{(i,j) \in S} A_{ij} \subset F$,
其中 $F$ 表示 最大似然算法 无法精确恢复社群标签
这一事件。
由 Bonferroni 不等式,我们有
\begin{equation}\label{eq:bonf}
	P(\bigcup_{(i,j)\in S} A_{ij}) \geq
	\sum_{(i,j)\in S} P(A_{ij})
	- \sum_{(i,j) \neq (r,s)} P(A_{ij} \cap A_{rs})		
\end{equation}
为根据 \eqref{eq:bonf}
得到
$P(\cup_{(i,j)\in S} A_{ij})$
的下界,
我们需要得到 $\sum_{(i,j)\in S} P(A_{ij})$
的下界 和
$\sum_{(i,j) \neq (r,s)} P(A_{ij} \cap A_{rs})$
的上界。
我们首先 处理 $P(A_{ij})$ 的情况。
注意到 单个事件 $A_{ij}$
等价于 $A_1$。
由 引理 \ref{lem:single_lower},
$P(A_{ij})$ 的下界是
$n^{-\gamma D_{1/2}(p_0 || p_1)-(\sqrt{a} - \sqrt{b})^2 +o(1)}$。
因为
$|S|=(n/2)^2$,
项 $\sum_{(i,j) \in S} P(A_{ij})$ 的 阶数是 
$\frac{1}{4}n^{-\mu+o(1)}$。
下面 我们根据两种情况 给出 $P(A_{ij} \cap A_{rs})$ 的上界。

首先是  $|\{i,j,r,s\}|=4$
的情形。 则 $A_{ij} \cap A_{rs}$ 蕴含事件
$A_{ijrs}: P(y,z|x^{(1)})P(y,z|x^{(2)}) > P^2(y,z|x^*)$,
其中 $x^{(1)}$ 在 $(i,j)$ 两个位置与 $x^*$ 不同
而 $x^{(2)}$ 在 $(r,s)$ 与 $x^*$
不同。
取对数并化简后,
$A_{ijrs}$ 的不等式表达
与 $A_2$ 一致。
因此, $P(A_{ij} \cap A_{rs}) \leq n^{-2(\theta^*_2 + o(1))} $。
集合 $S_1:=\{(i,j,r,s)| i<j, r<s, |\{i,j,r,s\}|=4\}$
的元素个数是 $\binom{n}{4} \leq n^4$.
因此,概率和
$\sum_{(i,j,r,s) \in S_1} P(A_{ij} \cap A_{rs})
\leq n^{-2\mu +o(1)}$,因为$\mu > 0$,
其阶数比 $n^{-\mu+o(1)}$ 小。

Another case happens when $|\{i,j,r,s\}|=3$. Under such case, without loss of generality we can assume $i=r, y^{(1)}_i = y^{(2)}_r = 1$.
For the case $y^{(1)}_i = y^{(2)}_r = -1$, we only need to exchange
$p_0$ and $p_1$, and the following analysis is still valid.
Then
\begin{align}
A_{ijrs}: &\, 2\sum_{i=1}^m  \log \frac{p_1(x_{1i})}{p_0(x_{2i})}
+ \sum_{i=1}^{2m} \log \frac{p_0(x_{2i})}{p_1(x_{2i})} \\
& +\log\frac{a}{b}\left(
\sum_{i=1}^{n} (z'_i - z_i) + 2\sum_{i=n+1}^{3n/2} (z'_i - z_i)\right)  \geq 0\notag
\end{align}
Using Chernoff's inequality, we can write an upper bound of $P(A_{ijrs})$ as
\begin{align*}
&P(A_{ijrs}) \leq
\left(\sum_{x\in \mathcal{X}} p_0^{1-2s}p_1^{2s}\right)^m
\left(\sum_{x\in \mathcal{X}} p_1^{1-s}p_0^{s}\right)^{2m} \\
& \cdot\exp\Bigl(\log n (-\frac{3}{2}(a+b)+a\exp(-s\log \frac{a}{b})+b\exp(s\log \frac{a}{b}) \\
&+ \frac{a}{2}\exp(-2s\log \frac{a}{b})+\frac{b}{2}\exp(2s\log \frac{a}{b})+o(1))\Bigr)
\end{align*}
Let $s=\frac{1}{3}$. We then have
\begin{align*}
P(A_{ijrs})&\leq  (\sum_{x\in \mathcal{X}} p_0^{1/3}(x)p_1^{2/3}(x))^{3m}\\
& \cdot \exp(\frac{3}{2}\log n (-a-b+a^{1/3}b^{2/3}+a^{2/3}b^{1/3}+o(1))) \\
&\leq   \exp(-\log n(\gamma D_{1/2}(p_1 || p_0) \\
&+ \frac{3}{2} (a+b-a^{1/3}b^{2/3}-a^{2/3}b^{1/3})+o(1)))
\end{align*}
where the last inequality follows from Lemma \ref{lem:p0p1}.
It then follows that
$$
P(A_{ijrs}) \leq n^{-\mu'/2-1-(\gamma  D_{1/2}(p_0||p_1) + (\sqrt{a} - \sqrt{b})^2) + o(1)}
$$
where $\mu'=(\sqrt{a}-\sqrt{b})^2-2 
- 3a^{1/3}b^{1/3}(a^{1/6}-b^{1/6})^2>0$ from \eqref{eq:oneC}. 
The set $S_2:=\{(i,j,r,s)| i<j, r<s, |\{i,j,r,s\}|=3\}$
has at most $n^3$ such terms,
then we have $\sum_{(i,j,r,s)\in S_2}
P(A_{ij}\cap A_{rs}) \leq n^{-\mu'/2-\mu+o(1)}$,
which has smaller order than $n^{-\mu+o(1)}$.

Based on the above discussion, 
$\sum_{(i,j,r,s)\in S} P(A_{ij} \cap A_{rs})\leq n^{-2\mu + o(1)}
+ n^{-\mu - \mu'/2} = o(1) n^{-\mu + o(1)}$.
Then we conclude that 
\begin{align*}
P_e=P(F) & \geq P(\cup_{(i,j)\in S} A_{ij}) \\
& \geq \frac{1}{4} n^{-\mu+o(1)}- o(1)n^{-\mu + o(1)},
\textrm{ from } \eqref{eq:bonf}  \\
&=(\frac{1}{4}+o(1))n^{-\mu + o(1)}
\end{align*}

\end{proof}
