\chapter{第 \ref{chap:sbmsi} 章补充内容}
\section{引理\ref{lem:I_plus_expression} 的证明}
在证明引理
\ref{lem:I_plus_expression}  之前,
我们先列举泊松分布的一个重要的性质。
\begin{lemma}\label{lem:poisson_property}
假设 $P_0 \sim \Pois(c)$, $P_1 \sim \Pois(d)$.
且 $P_{\lambda}$ 定义成
$$
P_{\lambda}(x) = \frac{P_0^{1-\lambda}(x) P_1^{\lambda} (x)}
{\sum_{x \in \mathcal{X}}P_0^{1-\lambda}(x) P_1^{\lambda} (x)}
$$
则我们有
\begin{enumerate}
    \item $P_{\lambda} \sim \Pois(c^{1-\lambda} d^{\lambda})$
    \item $D_{\mathrm{KL}}(P_{\lambda}||P_0) = 
    \lambda c^{1-\lambda}d^{\lambda}\log(\frac{d}{c}) + c-c^{1-\lambda}d^{\lambda}$
    \item $D_{\mathrm{KL}}(P_{\lambda}||P_1) = (1-\lambda)
    c^{1-\lambda}d^{\lambda}\log(\frac{c}{d})
    + d-c^{1-\lambda}d^{\lambda}$
\end{enumerate}
\end{lemma}
上面的第2、3条性质可由两个泊松分布的KL散度
公式导出。
\begin{proof}[引理\ref{lem:I_plus_expression} 的证明]
由 切尔诺夫信息的定义 \eqref{eq:D_star_lambda_star} 式
可得:
$$
I_+ = D(P_{\lambda^*} || \Pois(\frac{a}{2}))
+ D(P_{1-\lambda^*} || \Pois(\frac{b}{2}))
+ \gamma D(P_{\lambda^*} || P_0)
$$
在引理\ref{lem:poisson_property}
中取$c=\frac{a}{2}, d=\frac{b}{2}$
并应用第2、3条性质可得
\begin{align*}
    I_+ = \left[\lambda^* c^{1-\lambda^*}d^{\lambda^*}
    \log(\frac{d}{c})+ c-c^{1-\lambda^*}d^{\lambda^*}
    \right]
+ \left[\lambda^* c^{\lambda^*}d^{1-\lambda^*}\log(\frac{c}{d})
+ d - c^{\lambda^*}d^{1-\lambda^*}\right]
+ \gamma D(P_{\lambda^*} || P_0)
\end{align*}
即\eqref{equation:I+} 式。
$\lambda*$ 可由\eqref{eq:C_P_1_P_2_another} 式
进行计算。目标函数为:
\end{proof}