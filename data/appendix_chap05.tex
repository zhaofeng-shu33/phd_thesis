\section{定理证明}
\subsection{定理 \ref{thm:Pe} 的证明}
\begin{lemma}\label{lem:p0p12}
    设 $P_{\widetilde{Y}}$ 是定义在字母集
    $\mathcal{Y}$ 上的分布函数。
    对于 $\SBMSI(n,m,p,q,P_0,P_1)$,定义
    $I_1, I_2$ 如下
    \begin{align}
        I_1 &=\min_{P_{\widetilde{Y}}}\  \gamma D(P_{\widetilde{Y}}|| P_0)+ \frac{1}{2} g(a,b, 2\epsilon),
        \text{ 其中}\nonumber\\
        \epsilon &= \gamma \frac{D(P_{\widetilde{Y}} || P_1) - D(P_{\widetilde{Y}} || P_0) }{\log \frac{a}{b}}\label{eq:I1}
    \end{align}
    函数 $g(a,b,\epsilon)$ 在式 \eqref{equation:g}中定义,而 $I_2$ 的定义通过交换 $P_0, P_1$ 的位置得到:
    \begin{align}
        I_2 & = \min_{P_{\widetilde{Y}}} \gamma D(P_{\widetilde{Y}}|| P_1)+ \frac{1}{2} g(a,b, 2\epsilon),
        \text{ 其中}\nonumber\\
        \epsilon &= \gamma \frac{D(P_{\widetilde{Y}} || P_0) - D(P_{\widetilde{Y}} || P_1) }{\log \frac{a}{b}}
        \label{eq:I2}
    \end{align}
     则我们有 $I_1=I_2=I_+$。
\end{lemma}
\begin{proof}
    我们只需证明 $I_1=I_+$ 成立即可。
另一半 $I_2=I_+$ 可由 $I_+$ 定义的对称性导出。

我们用 拉格朗日乘子法 来求解 \eqref{eq:I1}。
令
\begin{align*}
L(P_{\widetilde{Y}},\epsilon, \lambda)
=\gamma D(P_{\widetilde{Y}}|| P_0)+
\frac{1}{2} g(a,b, 2\epsilon)
- \lambda(\epsilon \log\frac{a}{b}-\gamma
D(P_{\widetilde{Y}} || P_1) + \gamma D(P_{\widetilde{Y}} || P_0))
\end{align*}
对于给定的 $\epsilon$,极小化 $L(P_{\widetilde{Y}},\epsilon, \lambda)$
相当于极小化
$(1-\lambda)D(P_{\widetilde{Y}} || P_0) +
\lambda D(P_{\widetilde{Y}} || P_1) $。通过类似
文献 \inlinecite{cover1999elements} 里 式(11.199) 中求导的方式,
我们得到 $P_{\widetilde{Y}}(y) = P_{\lambda}(y)$,
其中 $P_{\lambda}(y)$ 的定义见式\eqref{eq:P_lambda_0_1}。
另一方面,由 $\frac{\partial L(P_{\widetilde{Y}},\epsilon, \lambda)}{\partial \epsilon}=0$
并将 式\eqref{eq:P_lambda_0_1}
代入到 式\eqref{eq:I1}, 我们得到
\begin{align*}
\lambda \log \frac{a}{b}
& = \log \frac{\epsilon + \sqrt{\epsilon^2+ab}}{b} \\
\epsilon \log \frac{a}{b}
& = \gamma\frac{\sum_{Y \in \mathcal{Y}}P_0^{1-\lambda}(y) P_1^{\lambda} (y)
\log \frac{P_0(y)}{P_1(y)}}{\sum_{y \in \mathcal{Y}}
P_0^{1-\lambda}(y) P_1^{\lambda} (y)}
\end{align*}
上述两式消掉 $\epsilon$ 后,我们可以得到关于
$\lambda$ 的方程:
\begin{align*}
\frac{\log\frac{a}{b}}{2}
\left(a^{\lambda} b^{1-\lambda}
-a^{1-\lambda} b^{\lambda}\right)
+ \gamma \frac{\sum_{y \in \mathcal{Y}}
P_0^{1-\lambda}(y) P_1^{\lambda}(y)
\log \frac{P_1(y)}{P_0(y)}}
{\sum_{y \in \mathcal{Y}}P_0^{1-\lambda}(y)
P_1^{\lambda} (y)}
=0
\end{align*}
上式恰等价于 式\eqref{eq:lambda} 中目标函数对 $\lambda$求导后整个式子等于零。
因此$P_{\widetilde{Y}}=P_{\lambda^*}(\lambda)$。
代入$I_1$ 的表达式中化简后我们可得
$I_1 = I_+$。
\end{proof}

当 $I_+ > 1$ 时精确恢复的可实现性可由 定理 \ref{thm:sdp}  给出的关于 SDP算法可实现精确恢复的论断证明。
此处我们证明当 $I_+ < 1$ 时 
精确恢复的不可实现性。
%同 \ref{sec:thm_sdp_proof} 节,
令$S_1$ 和 $S_2$ 分别表示标签为 1 和 -1 的节点集合。
对于 节点 $i$ 和节点集 $S\subset [n]$,
令 $E(i,S)$ 表示图$G$中连接$i$和$S$中的节点的边的数量。
采用和文献 \inlinecite{abbe2015exact} 中第五节关于下界的讨论 类似的方法
我们可以得到,当 $I_1<1$时,存在大小为 $\frac{n}{\log^3 n}$  的 子集合
    $H_1\subset S_1$ 和 节点 $i_1\in H_1$,使得下式以接近于1的概率成立:
\begin{align}
    \sum_{j=1}^{m} \log \frac{P_1(y^{i_1}_{j})}{P_0(y^{i_1}_{j})}
    \ge \log \frac{p(1-q)}{q(1-p)}\Big(\frac{\log n}{\log\log n}+1
    +E(i_1, S_1 \backslash H_1) - E(i_1, S_2) \Big) \label{eq:G1}
\end{align}
这里条件中出现的 $I_1$ 在 式 \eqref{eq:I1} 中定义。
对称地, 若在 式\eqref{eq:I2} 中定义的$I_2$满足  $I_2<1$的条件,
则存在大小为 $\frac{n}{\log^3 n}$  的集合
    $H_2\subset S_2$ 和 节点 $i_2\in H_2$ 使得下式以接近于1的概率成立:
    \begin{align}
        \sum_{j=1}^{m} \log \frac{P_0(y^{i_2}_{j})}{P_1(y^{i_2}_{j})}
        \ge \log \frac{p(1-q)}{q(1-p)}\Big(\frac{\log n}{\log\log n}+1
        +E(i_2, S_2 \backslash H_2) - E(i_2, S_1) \Big) \label{eq:G2}
    \end{align}	
将 式\eqref{eq:G1} 与 式\eqref{eq:G2} 相加可得:
\begin{align}
    \sum_{j=1}^{m} \log \frac{P_1(y^{i_1}_{j})}{P_0(y^{i_1}_{j})}
    +\sum_{j=1}^{m} \log \frac{P_0(y^{i_2}_{j})}{P_1(y^{i_2}_{j})}
    &\ge \log \frac{p(1-q)}{q(1-p)}\Big(2\frac{\log n}{\log\log n}+2+E(i_1, S_1 \backslash H_1) \nonumber\\
    &+ E(i_2, S_2 \backslash H_2)- E(i_1, S_2 ) - E(i_2, S_1 )\Big) \label{eq:G1ij}
\end{align}	
    注意到 若 \eqref{eq:G1ij} 成立,考虑到 $\frac{\log n}{\log \log n} > E(i_1, H_1)$ 
    以接近1的概率成立,则有
\begin{align}
    \sum_{j=1}^{m} \log \frac{P_1(y^{i_1}_{j})}{P_0(y^{i_1}_{j})}
    +\sum_{j=1}^{m} \log \frac{P_0(y^{i_2}_{j})}{P_1(y^{i_2}_{j})}
    \ge &\log \frac{p(1-q)}{q(1-p)}(E(i_1, S_1 \backslash \{i_1\}) + E(i_2, S_2 \backslash \{i_2\})\nonumber\\
    &- E(i_1, S_2 \backslash \{i_2\}) - E(i_2, S_1 \backslash \{i_1\})) \label{eq:F1ij}
\end{align}
因此, 在 $I_1<1$ 和 $I_2<1$的条件下,以接近1的概率,
式\eqref{eq:F1ij} 成立。
根据 引理 \ref{lem:p0p12},
该条件即为 $I_+<1$。
又根据 式 \eqref{eq:lh} 和 $X$ 是均匀分布的假设,
式 \eqref{eq:F1ij} 等价于在式 \eqref{eq:lh} 中定义的 似然函数
$P(Y,Z|(X_1,\ldots,X_n),X_{i_1}=-1,X_{i_2}=1)$
比似然函数
$P(Y,Z|(X_1,\ldots,X_n),X_{i_1}=1,X_{i_2}=-1)$
大。
这导致了最大似然算法无法实现精确恢复。
因此,我们得出结论,当 $I_+<1$时,随着 $n$的增大,精确恢复失败的概率接近1。

\subsection{定理 \ref{thm:Pe_new} 的证明}

在开始定理 \ref{thm:Pe_new}  的正式证明之前,
我们先推导
最大似然算法无法精确恢复社团结构这一事件
对应的不等式。

对于
某个特定的节点标签
$x^*$,
若式
\eqref{eq:mle} 中的 ML 算法
无法精确恢复 
$x^*$,则
存在 $x\neq x^*$ 使得 $P(y,z|x) > P(y,z|x^*)$。
令 $F_k$ 表示在 $x$ 和 $x^*$ 之间有 $k$ 个不同的对
这个事件,即:
    \begin{equation}\label{eq:Fk}
    F_k:=\{\exists x \in \{\pm 1\}^n |
    \Dist(x, x^*)=2k,
    P(y,z|x) > 
    P(y,z|x^*) \}
    \end{equation}
    由于
    $x$ 要 满足约束
    $\sum_{i=1}^n x_i=0$,
    $\Dist(x, x^*)$
    只能取偶数值。
    在
    $P(y,z|x) > P(y,z|x^*)$
    两边同时取
    对数
    , 并利用式
    \eqref{eq:lh} ,
    可得如下等价不等式:
    \begin{equation}\label{eq:ein}
    \sum_{i=1}^{km}
    \left(\log \frac{P_1(y_{1i})}
    {P_0(y_{1i})}
    + \log \frac{P_0(y_{2i})}
    {P_1(y_{2i})}
    \right)
    \geq \log \frac{p(1-q)}{q(1-p)} \sum_{i=1}^{k(n-2k)}(z_{i} - z'_{i})
    \end{equation}
    
    其中 $y_{1i}(y_{2i})$ 分别采样自
    $P_0(P_1)$,
    且有 $z_{i} \sim \Bern(p), z'_{i} \sim \Bern(q)$。
    
    我们把式 \eqref{eq:ein} 所描述的事件
    叫做 $A_k$,
    则每个 $F_k$ 
    可看成
     $\binom{n/2}{k}^2$ 个
    对应不同节点标签的
    $A_k$ 事件的并集。
    
    为
    获得 $P(A_k)$
    的上界, 
    我们首先定义如下经验分布:
    \begin{align}
    P(\widetilde{Y}_j = u) &=
    \frac{1}{km} \sum_{i=1}^{km}
    \mathbf{1}[y_{ji} = u] \textrm{ 对于 } u \in \mathcal{Y}, j=1,2 
    \notag \\
    P(\widetilde{Z} = u) &= \frac{1}{k(n-2k)}\sum_{i=1}^{k(n-2k)} \mathbf{1}[z_i = u], u \in \{0, 1\}
    \label{eq:zu}
    \end{align}
     $\widetilde{Z}'$ 
     的定义类似 \eqref{eq:zu}。
 则
 式 \eqref{eq:ein} 可化为:
    \begin{align}
    &m\left[\sum_{y\in \mathcal{Y}}
    P_{\widetilde{Y}_1}(y)\log\frac{P_1(y)}
    {P_0(y)}
    +\sum_{y\in \mathcal{Y}}
    P_{\widetilde{Y}_2}(y)\log\frac{P_0(y)}
    {P_1(y)}
    \right] +(n-2k)\notag \\
    &\left[\sum_{z\in\{0,1\}}
    P_{\widetilde{Z}}(z) \log \frac{P_{B_q}(z)}
    {P_{B_p}(z)}
    + \sum_{z\in\{0,1\}} 
    P_{\widetilde{Z}'}(z) \log \frac{P_{B_p}(z)}
    {P_{B_q}(z)}\right] \geq 0 \label{eq:mnk}
    \end{align}
    其中 $P_{B_p}$、$P_{B_q}$ 分别表示参数为$p,q$
    的伯努利随机变量的概率质量函数。
    此外,为证明 \eqref{eq:PeMainL} 给出的
    下界,我们还需要如下两个引理。
    \begin{lemma}\label{lem:single_lower}
        对于 
         在式 \eqref{eq:ein}中给出的事件 $A_1$ 
          (取 $k=1$),
        我们有如下的不等式估计
        \begin{equation}\label{eq:pa1}
        P(A_1) \geq \exp\left(-
        \left(\gamma D_{1/2}(P_0||P_1) + \left(\sqrt{a} - \sqrt{b}\right)^2 + o(1)
        \right)\log n
        \right)
        \end{equation}
        \end{lemma}
        \begin{proof}[引理 \ref{lem:single_lower} 的证明] 
当 $k=1$ 时,
不等式 \eqref{eq:ein} 可写成
$\sum_{i=1}^{n-2} (z'_i - z_i)
\geq \epsilon$
其中
\begin{align*}
\epsilon := &\frac{m}{\log a/b}\cdot 
\Bigl[D(P_{\widetilde{Y}_{1}} || P_1) 
- D(P_{\widetilde{Y}_{1}} || P_0) + D(P_{\widetilde{Y}_{2}} || P_0) -
D(P_{\widetilde{Y}_{2}} || P_1)\Bigr]
\end{align*}
令 $P_{\widetilde{Y}^{*}_1}$
和 $P_{\widetilde{Y}^{*}_2}$
取相同的分布
$P(Y=y)=\frac{\sqrt{P_0(y)P_1(y)}}
{ \sum_{y\in \mathcal{Y}}
\sqrt{P_0(y) P_1(y)}} $,
从而使得 $\epsilon =0$。
对于这组特殊选取的
$P_{\widetilde{Y}^{*}_1}$
和 $P_{\widetilde{Y}^{*}_2}$,
我们使用 Sanov 定理可得
\begin{align*}
P(A_1)
&\geq\frac{1}
{(m+1)^{2|\mathcal{Y}|}}
\exp \left(-m(D(P_{\widetilde{Y}^*_1} || P_0)
+ D(P_{\widetilde{Y}^*_2} || P_1))
\right)
\cdot P\left(\sum_{i=1}^{n-2} (z'_i - z_i) \geq 0\right)\\
& = \exp(-\log n \cdot (\gamma D_{1/2}(P_0||P_1)+o(1))) 
P\left(\sum_{i=1}^{n-2} (z'_i - z_i) \geq 0 \right)
\end{align*}
由 文献\inlinecite{abbe2015exact}
中的引理 4,
$P(\sum_{i=1}^{n-2} (z'_i - z_i) \geq 0)$ 下界为
 $n^{-\left(\sqrt{a} - \sqrt{b}\right)^2 + o(1)}$。
因此得到 式\eqref{eq:pa1} 。
        \end{proof}

\begin{lemma}\label{lem:p0p1}
    令 $P_0,P_1$ 
    是两个定义在字母集
    $\mathcal{X}$ 上的概率分布,
    则下面的不等式成立:
    \begin{equation}\label{eq:32}
        \left(\sum_{x\in \mathcal{X}}
        P^{\frac{1}{3}}_0(x)
        P^{\frac{2}{3}}_1(x)\right)^3
        \leq \left(\sum_{x\in \mathcal{X}}
        \sqrt{P_0(x) P_1(x)}\right)^2
    \end{equation}
\end{lemma}

\begin{proof}
    令 $f(x)=P^{\frac{1}{3}}_0(x)
    P^{\frac{1}{3}}_1(x)$,
    $g(x) = P^{\frac{1}{3}}_1(x)$,
    $p=\frac{3}{2}, q=3$。
    容易验证 $\frac{1}{p} + \frac{1}{q}=1$。
    \newglossaryentry{holder_ieq}{name=赫尔德不等式, description={Hölder's inequality}}

    由\gls{holder_ieq}:
    \begin{equation}\label{eq:holder}
        (\sum_{x\in\mathcal{X}}f(x)g(x))\leq (\sum_{x\in\mathcal{X}} f^p(x))^{\frac{1}{p}}
        (\sum_{x\in\mathcal{X}} g^q(x))^{\frac{1}{q}}
    \end{equation}
    因为 $\sum_{x\in\mathcal{X}} P_1(x)=1$, \eqref{eq:holder} 蕴含
    \eqref{eq:32}。
\end{proof}
    
\begin{proof}[定理 \ref{thm:Pe_new} 的证明]

以下我们使用切尔诺夫不等式来给出 $P(A_k)$
的上界,即我们证明:
$$
P(A_k) \leq n^{-k\theta^*_k} 
\,\textrm{ 其中 }\, \theta^*_k=\gamma D_{1/2}(P_0||P_1)
+\left(1-\frac{2k}{n} \right)\left(\sqrt{a}-\sqrt{b}
\right)^2
$$
由 $A_k$ 的定义式 \eqref{eq:ein} 我们有
\begin{align*}
P(A_k) &\leq \mathbb{E}
\left[\exp \left( s\sum_{i=1}^{km}
\left( \log \frac{P_1(y_{1i})}
    {P_0(y_{1i})}
+ \log \frac{P_0(y_{2i})}
    {P_1(y_{2i})} \right)
\right)\right]\\
&\quad \cdot \mathbb{E}
\left[\exp\left(s\log \frac{a}{b}\sum_{i=1}^{k(n-2k)} (z'_i - z_i )\right)
\right] \\
& \stackrel{(a)}{=}
\left(\sum_{y\in \mathcal{Y}}
P_0^{1-s}(y)P_1^{s}(y)\right)^{km}
\left(\sum_{y\in \mathcal{Y}} P_1^{1-s}(y) P_0^{s}(y)
\right)^{km}\\
&   \quad \cdot \exp\left(
    k\log n \left(1-\frac{2k}{n} \right)
    (-a-b+a^sb^{1-s}+b^sa^{1-s} +o(1)) 
    \right)
\end{align*}
其中 $(a)$ 由各样本相互独立可得。 取 $s=\frac{1}{2}$,
则我们有
$P(A_k) \leq  n^{-k(\theta^*_k+o(1))}$。

当 $k \geq \frac{n}{\sqrt{\log n}}$ 时,
由引理 \ref{lem:sigmaX},
$P(F_k)$ 指数衰减。
对于 $2\leq k < \frac{n}{\sqrt{\log n}}$
的情形,
错误概率通过如下的方式进行分析:
由布尔不等式,
我们可以给出 $P(F_k)$ 的上界为
\begin{equation}\label{eq:FAk}
P(F_k) \leq \binom{n/2}{k}^2 P(A_k)
\end{equation}
并且结合不等式
\begin{equation}\label{eq:nk_binom_small}
    \binom{n}{k} \leq \left(\frac{ne}{k} \right)^k
\end{equation}
我们可以进一步 向上放缩 $P(F_k)$ 为
\begin{align}
P(F_k) & \leq \left(\frac{ne}{2k}\right)^{2k} P(A_k) 
\label{eq:P_F_k_less_A_k}\\
&\leq \exp\left(k\left(2\log
\left(\frac{ne}{2k}\right)
-\theta_k^* \log n\right) + o(1) \right)
\notag
\end{align}
\begin{align*}
P_e &\leq P(F_1)+(1+o(1))\sum_{k=2}^{\frac{n}{\sqrt{\log n}}} P(F_k) \\
& \leq (1+o(1))\cdot \sum_{k=2}^{\frac{n}{\sqrt{\log n}}}
\exp\left(k\left(-\mu \log n + \frac{2k}{n} \log n\left(\sqrt{a} - \sqrt{b}\right)^2 - 2\log 2k + 2\right)
\right)
\end{align*}
其中 $\mu$ 定义为
\begin{equation}\label{eq:mu_def}
	\mu = \left(\sqrt{a} - \sqrt{b}\right)^2-2 + \gamma D_{1/2}(P_0||P_1) > 0	
\end{equation}
对于 $P(F_1)$, 我们有 $P(F_1)\leq (n/2)^2
P(A_1)\leq \frac{1}{4}n^{-\mu+o(1)}$。
对于 $2\leq k \leq \frac{n}{\sqrt{\log n}}$,
使用不等式
$$
\frac{2k}{n}\left(\sqrt{a} - \sqrt{b}\right)^2\log n -2\log2k+2\leq  C\sqrt{\log n}
$$
我们可以得到
\begin{align*}
P_e &\leq \frac{1}{4}n^{-\mu + o(1)} +(1+o(1)) \sum_{k=2}^{\frac{n}{\sqrt{\log n}}} \exp(k((-\mu + o(1)) \log n )) \\
& =\frac{1}{4}n^{-\mu + o(1)}+(1+o(1)) \frac{n^{-\mu + o(1)}}{1-n^{-\mu + o(1)}} \\
&= \left(\frac{1}{4}+o(1)\right)n^{-\mu + o(1)}
\end{align*}
因此, \eqref{eq:PeMain} 成立。

为证明误差率下界的表达式
\eqref{eq:PeMainL},
我们首先定义事件 $A_{ij}$ 为
$P(y,z|x) > P(y,z|x^*)$,
其中 $x$ 定义为
$x_s=-x^*_s,
s=i,j$,除此之外
$x_s=x^*_s$
。

另外定义 $S:=\{(i,j)| 1\leq i < j\leq n,
x_i^*=-x_j^*
\}$。
则容易得到 $\cup_{(i,j) \in S} A_{ij} \subset F$,
其中 $F$ 表示 最大似然算法 无法精确恢复社团标签
这一事件。
由 Bonferroni 不等式,我们有
\begin{equation}\label{eq:bonf}
	P\left(\bigcup_{(i,j)\in S} A_{ij} \right) \geq
	\sum_{(i,j)\in S} P(A_{ij})
	- \sum_{(i,j) \neq (r,s)} P(A_{ij} \cap A_{rs})		
\end{equation}
为根据 \eqref{eq:bonf}
得到
$P(\cup_{(i,j)\in S} A_{ij})$
的下界,
我们需要得到 $\sum_{(i,j)\in S} P(A_{ij})$
的下界 和
$\sum_{(i,j) \neq (r,s)} P(A_{ij} \cap A_{rs})$
的上界。
我们首先 处理 $P(A_{ij})$ 的情况。
注意到 单个事件 $A_{ij}$
等价于 $A_1$。
由 引理 \ref{lem:single_lower},
$P(A_{ij})$ 的下界是
$n^{-\gamma D_{1/2}(P_0 || P_1)-\left(\sqrt{a} - \sqrt{b}\right)^2 +o(1)}$。
因为
$|S|=(n/2)^2$,
项 $\sum_{(i,j) \in S} P(A_{ij})$ 的 阶数是 
$\frac{1}{4}n^{-\mu+o(1)}$。
下面 我们根据两种情况 给出 $P(A_{ij} \cap A_{rs})$ 的上界。

首先是  $|\{i,j,r,s\}|=4$
的情形。 则 $A_{ij} \cap A_{rs}$ 蕴含事件
$A_{ijrs}: P(y,z|x^{(1)})P(y,z|x^{(2)}) > P^2(y,z|x^*)$,
其中 $x^{(1)}$ 在 $(i,j)$ 两个位置与 $x^*$ 不同
而 $x^{(2)}$ 在 $(r,s)$ 与 $x^*$
不同。
取对数并化简后,
$A_{ijrs}$ 的不等式表达
与 $A_2$ 一致。
因此, $P(A_{ij} \cap A_{rs}) \leq n^{-2(\theta^*_2 + o(1))} $。
集合 $S_1:=\{(i,j,r,s)| i<j, r<s, |\{i,j,r,s\}|=4\}$
的元素个数是 $\binom{n}{4} \leq n^4$。
因此,概率和
$\sum_{(i,j,r,s) \in S_1} P(A_{ij} \cap A_{rs})
\leq n^{-2\mu +o(1)}$,因为$\mu > 0$,
其阶数比 $n^{-\mu+o(1)}$ 小。

另一种情形是 $|\{i,j,r,s\}|=3$。
在这种情形下,
不失一般性我们假设 $i=r, x^{(1)}_i = x^{(2)}_r = 1$。
而对于 $x^{(1)}_i = x^{(2)}_r = -1$ 的情形,
我们只需交换
$P_0$ 和 $P_1$ 的位置,
如下的分析仍旧成立。
可以推出
\begin{align}
A_{ijrs}: &\, 2\sum_{i=1}^m  \log \frac{P_1(y_{1i})}{P_0(y_{1i})}
+ \sum_{i=1}^{2m} \log \frac{P_0(y_{2i})}{P_1(y_{2i})} 
 +\log\frac{a}{b}\left(
\sum_{i=1}^{n} (z'_i - z_i) + 2\sum_{i=n+1}^{3n/2} (z'_i - z_i)\right)  \geq 0\notag
\end{align}
由 切尔诺夫不等式,
$P(A_{ijrs})$ 的上界可以写成
\begin{align*}
P(A_{ijrs}) &\leq
\left(\sum_{y\in \mathcal{Y}}
P_0^{1-2s}(y) P_1^{2s}(y)\right)^m
\left(\sum_{y\in \mathcal{Y}} P_1^{1-s}(y)P_0^{s}(y)\right)^{2m} \\
& \cdot\exp\Bigl(\log n (-\frac{3}{2}(a+b)+a\exp(-s\log \frac{a}{b})+b\exp(s\log \frac{a}{b}) \\
&+ \frac{a}{2}\exp(-2s\log \frac{a}{b})+\frac{b}{2}\exp(2s\log \frac{a}{b})+o(1))\Bigr)
\end{align*}
令 $s=\frac{1}{3}$,则我们有
\begin{align*}
P(A_{ijrs})&\leq  \left(\sum_{y\in \mathcal{Y}}
P_0^{1/3}(y) P_1^{2/3}(y) \right)^{3m} \\
&\quad\cdot 
\exp\left(\frac{3}{2}\log n \cdot \left(
    -a-b+a^{1/3}b^{2/3}+a^{2/3}b^{1/3}+o(1) 
\right) \right) \\
&\leq   \exp \left(-\log n
\cdot \left(\gamma D_{1/2}(P_1 || P_0) + \frac{3}{2} (a+b-a^{1/3}b^{2/3}-a^{2/3}b^{1/3})+o(1)
\right) \right)
\end{align*}
其中 最后一个不等式由 引理 \ref{lem:p0p1}
得出。
因此我们有
$$
P(A_{ijrs}) \leq n^{-\mu'/2-1-
\left(\gamma  D_{1/2}(P_0||P_1) + \left(\sqrt{a} - \sqrt{b}\right)^2 \right) + o(1)}
$$
其中 $\mu'=\left(\sqrt{a} - \sqrt{b}\right)^2-2 
- 3a^{1/3}b^{1/3}\left(a^{1/6} - b^{1/6}\right)^2$。
由 式\eqref{eq:oneC}, $\mu'>0$。
集合 $S_2:=\{(i,j,r,s)| i<j, r<s, |\{i,j,r,s\}|=3\}$
的元素个数小于 $n^3$,
则我们有 $\sum_{(i,j,r,s)\in S_2}
P(A_{ij}\cap A_{rs}) \leq n^{-\mu'/2-\mu+o(1)}$,
其阶数比 $n^{-\mu+o(1)}$ 小。

基于以上讨论,
$\sum_{(i,j,r,s)\in S} P(A_{ij} \cap A_{rs})\leq n^{-2\mu + o(1)}
+ n^{-\mu - \mu'/2} = o(1) n^{-\mu + o(1)}$。
因此我们推出
\begin{align*}
P_e=P(F) & \geq P\left(\bigcup_{(i,j)\in S} A_{ij} \right) \\
& \geq \frac{1}{4} n^{-\mu+o(1)}- o(1)n^{-\mu + o(1)},
\textrm{ 由 } \eqref{eq:bonf}  \\
&=\left(\frac{1}{4}+o(1)\right)n^{-\mu + o(1)}
\end{align*}
\end{proof}

\subsection{定理 \ref{thm:I_plus_relation} 的证明}

\begin{proof}[定理 \ref{thm:I_plus_relation} 的证明]
    由引理 \ref{lem:p0p12},
    我们只需证明
    $$
    I_1 \geq 
    \frac{1}{2}\left[
        \gamma D_{1/2}(P_0||P_1)
    + \left(\sqrt{a} - \sqrt{b}\right)^2
    \right]
    $$
    由于定义在 式\eqref{equation:g} 中的 $g(a,b,\epsilon)$ 是关于 $\epsilon$
    的凸函数,则我们有
    \begin{equation}\label{eq:g_linear}
            g(a,b,\epsilon) \geq g(a,b,0) + g'(a,b,0)\epsilon = \left(\sqrt{a} - \sqrt{b}\right)^2 + \frac{\epsilon}{2}\log \frac{a}{b}
        \end{equation}
        则由引理 \ref{lem:p0p12},
        \begin{align}
            I_1 &\geq \frac{1}{2}\left(\sqrt{a} - \sqrt{b}\right)^2+\min_{P_{\widetilde{Y}}}\ 
            \gamma
            \frac{D(P_{\widetilde{Y}} || P_1) + D(P_{\widetilde{Y}} || P_0)}{2} \notag\\
            &= \frac{1}{2}\left(\left(\sqrt{a} - \sqrt{b}\right)^2+\gamma D_{1/2}(P_0||P_1)
            \right)
            \label{eq:I_1_ab_equal_renyi}
        \end{align}
        不难验证,$D(P_{\widetilde{Y}} || P_1) + D(P_{\widetilde{Y}} || P_0)$
        的最小值是雷尼散度$D_{1/2}(P_0||P_1)$,从而
        式\eqref{eq:I_1_ab_equal_renyi} 成立。
\end{proof}

\subsection{定理 \ref{thm:constant} 的证明}

\begin{proof}[定理 \ref{thm:constant} 的证明]
$A_k$采用式\eqref{eq:ein}中给出的定义。
当 $p,q$ 是常数时,
由式 \eqref{eq:mnk},
$P(A_k)$ 可用 Sanov 定理来估计。
即当 $n\to \infty$ 时,
$-\frac{1}{kn}\log P(A_k) \to \theta^*_k$。
这里$\theta^*_k$ 定义为
\begin{align*}
\theta^*_k &= \min_{\widetilde{Y}_1, \widetilde{Y}_2, \widetilde{Z}, \widetilde{Z}'}
\gamma (D(P_{\widetilde{Y}_1}||P_0) + D(P_{\widetilde{Y}_2}||P_1)) \\
&\quad + (1-\frac{2k}{n})
(D(P_{\widetilde{Z}}||\Bern(p)) + D(P_{\widetilde{Z}'}||\Bern(q)))  \\
\textrm{s.t. } \, & (\widetilde{Y}_1, \widetilde{Y}_2, \widetilde{Z}, \widetilde{Z}')
\textrm{ 满足 } \eqref{eq:mnk}
\end{align*}
通过拉格朗日乘子法求解上述优化问题,我们可得
\begin{align*}
P_{\widetilde{Y}_1}(x) = c_1 P_0^{1-\lambda}(y)P_1^{\lambda}(y),\quad &
P_{\widetilde{Y}_2}(x) = c_2 P_1^{1-\lambda}(y)P_0^{\lambda}(y) \\
P_{\widetilde{Z}}(z) = c_3 P_{B_p}^{1-\lambda}(x)P_{B_q}^{\lambda}(z),\quad &
P_{\widetilde{Z}'}(z) = c_4 P_{B_q}^{1-\lambda}(x)P_{B_p}^{\lambda}(z)
\end{align*}
其中$B_p, B_q$的定义同式\eqref{eq:mnk},
而  $c_1, \dots, c_4$ 是这些分布的 归一化系数 。
系数 $\lambda$ 的选取 要使得式\eqref{eq:mnk}
变成等式约束,从而有 $\lambda=\frac{1}{2}$。
因此, $\theta^*_k = \gamma D_{1/2}(P_0 || P_1) + (1-\frac{2k}{n}) D_{1/2}(\Bern(p)||\Bern(q))$。
记 $C_1=\gamma D_{1/2}(P_0 || P_1)$和
$C_2=D_{1/2}(\Bern(p)||\Bern(q))$,
则对于较大的 $n$, $P(A_k) \leq \exp(-knC_1-k(n-2k) C_2)
$ 。
由 式 \eqref{eq:P_F_k_less_A_k} 和 式\eqref{eq:nk_binom_small}
\begin{align*}
P_e \leq \sum_{k=1}^{n/4} \binom{n/2}{k}^2 P(A_k)\leq \sum_{k=1}^{n/4} \exp(-nf(k))
\end{align*}
其中
$f(k) = \frac{2k}{n}\log \frac{2k}{ne} + k(C_1+C_2) - \frac{2k^2}{n}C_2$。
进一步地, $f'(x)= \frac{2}{n} \log \frac{2x}{n} + C_1+C_2 - \frac{4C_2x}{n}$,
$1\leq x \leq \frac{n}{4}$。
由 $f'(1) > 0, f'(\frac{n}{4}) > 0 $ 可推出 $f'(x) > 0$
对于 $1\leq x \leq \frac{n}{4}$ 均成立。
因此, $f(x)$ 在区间 $[1, \frac{n}{4}]$ 递增,
且 $f(k) \geq f(1)$ 对于 $1\leq k \leq \frac{n}{4}$ 成立。
\begin{equation}
P_e \leq \frac{n}{4}\exp(-nf(1)) = \exp(-n (C_1+C_2+o(1)))
\end{equation}
另一方面, $P_e \geq P(A_1) = \exp(-n(C_1+C_2+o(1)))$。
最后,我们有
$-\frac{1}{n} \lim_{n \to \infty} \log P_e = C_1+C_2$,
定理 \ref{thm:constant}  证毕。
\end{proof}
\subsection{定理 \ref{thm:sdp} 的证明}
\label{sec:thm_sdp_proof}
在证明定理 \ref{thm:sdp}  之前,我们先给出优化问题式\eqref{eq:matrix_mle}中目标函数表达式的推导过程。

由 式\eqref{eq:lh} 可得似然函数为
$$
L=\prod_{i=1}^n \prod_{j=1}^m P_0^{x_i}(y_i^{(j)})P_1^{1-x_i}(y_i^{(j)})
\prod_{(i,j) \in E} p^{\delta(x_i, x_j)}q^{1-\delta(x_i, x_j)}
\prod_{(i,j)\not\in E} (1-p)^{\delta(x_i, x_j)}(1-q)^{1-\delta(x_i, x_j)}
$$
其中 $x_i \in \{0,1\}$。
对给定的图$G$,忽略掉常数项, 极大化 $ \log L$ 等价于
$$
\max \sum_{i=1}^n \sum_{j=1}^m x_i \log \frac{P_0(y_i^{(j)})}{P_1(y_i^{(j)})}
+\log\frac{p}{q}\sum_{(i,j) \in E} \delta(x_i, x_j)
+\log \frac{1-p}{1-q}\sum_{(i,j)\not\in E} \delta(x_i, x_j)
$$
令 $\kappa = -\log\frac{1-p}{1-q} / \log\frac{p}{q} \sim \frac{a-b}{\log a/b}\frac{\log n}{n}$,
则我们只需考虑
$$
\max\  \frac{1}{\log a/b}\sum_{i=1}^n \sum_{j=1}^m x_i \log \frac{P_0(y_i^{(j)})}{P_1(y_i^{(j)})}
+\sum_{(i,j) \in E} \delta(x_i, x_j)
-\kappa\sum_{(i,j)\not\in E} \delta(x_i, x_j)
$$
令 $x_i = (v_i+1)/2$,其中 $v_i \in \{\pm 1 \}$。
则我们有 $\delta(x_i, x_j) = \frac{v_i v_j + 1}{2}$。
则上式等价于:
$$
\max\  \frac{1}{\log a/b}\sum_{i=1}^n \sum_{j=1}^m v_i \log \frac{P_0(y_i^{(j)})}{P_1(y_i^{(j)})}
+\sum_{(i,j) \in E} v_i v_j
-\kappa\sum_{(i,j)\not\in E} v_i v_j
$$
取 $h_i = \frac{1}{\log a/b}\sum_{j=1}^m \log \frac{P_0(y_i^{(j)})}{P_1(y_i^{(j)})}$
并
将上式乘以2再减去常数 $\frac{1}{2}v^T(J_n-I_n)v$ 可得
$$
2\sum_{i=1}^n h_iv_i + \sum_{(i,j)\in E} v_i v_j - (2\kappa+1) \sum_{(i,j)\not\in E} v_i v_j
$$

定义 矩阵 $B$ 为 \eqref{eq:def_B_sdp} 的形式,则
优化问题 化为$2\max h^T v + \frac{1}{4}v^T B v$ (忽略 $\kappa=o(1)$),
即为式\eqref{eq:matrix_mle}。

下述引理是
文献\inlinecite{abbe2015exact}
中引理 8 的推广:
\begin{lemma}\label{lem:zxt}
    假设 $m > n, Z \sim \Binom(m, \frac{b\log n}{n}), X\sim \Binom(m, \frac{a\log n}{n})$。
    对于 $ t \geq \frac{m}{n}(b - a)$, 我们有
    \begin{align}\label{eq:estimation}
        P(Z - X \geq t \log n) 
        \leq \exp\left(-\frac{m}{n}\log n \cdot 
        \left( g(a, b, \frac{n}{m}t) + O\left(\frac{\log n}{n} \right)
        \right)
        \right)
    \end{align}
    其中 $g(a,b,\epsilon)$ 在 式\eqref{equation:g} 中定义。
\end{lemma}
\begin{proof}
    沿用引理 \ref{lem:enhanced_fb} 的证明,
在 式\eqref{eq:exp_s_B_A_sigma}
中取  $N_{A_{\bar{\sigma}}}
=N_{B_{\bar{\sigma}}}=m$,利用
切尔诺夫不等式我们有
\begin{equation*}
    P(Z - X \geq t \log n) \leq \exp\left(-\frac{m}{n}\log n \cdot
    \left(a+b-ae^{-s}-be^s+\frac{n}{m}ts + O\left(\frac{\log n}{n} \right)
    \right)
    \right)
\end{equation*}
其中 $s \geq 0$。
由 $g(a,b,\epsilon) = 
\max_{s \geq 0} (a+b-a e^{-s} - b e^s + \epsilon s)$
可得 式 \eqref{eq:estimation}。
\end{proof}
考虑 式\eqref{eq:sdp} 的对偶问题
\begin{align}
    \min_{a_1, \dots, a_{n+1}}\, &\sum_{i=1}^{n+1} a_i \nonumber\\
    \label{eq:dual}
    \textrm{s.t. } &\, \diag\{a_1, \dots, a_{n+1}\} + \Xi - \widetilde{B} \succeq 0
\end{align}
其中 $(n+1)\times (n+1)$ 维的对称矩阵
$\Xi$ 定义成 
\begin{equation}
    \Xi_{ij} = \begin{cases}
        \lambda_1, & i=1\text{ 或 }j=1 \text{, 且 }i\ne j\\
        \lambda_i + \lambda_j, & i, j\in\{2,\ldots,n+1\}
    \end{cases}
\end{equation}
$\lambda_i$ 是只出现在约束条件中的优化变量。
假设 $g=(1,\ldots,1,-1,\ldots,-1)$ 是节点真实的标签,
也即
前一半和后一半的标签分别是  $1$ 和 $-1$。
与文献 \inlinecite{abbe2015exact} 中采用的方法类似,
$(1,g)(1,g)^T$ 是 \eqref{eq:sdp} 的唯一最优解
 被如下条件所保证:
\begin{enumerate}
    \item[(a)] $(1,g)(1,g)^T$ 是 原问题 \eqref{eq:sdp} 的可行解。
    \item[(b)] 存在 \eqref{eq:dual} 的可行解 $(a_1,\ldots,a_{n+1},$ $\lambda_1,\ldots,\lambda_{n+1})$。
    \item[(c)] $(\diag\{a_1, \dots, a_{n+1}\} + \Xi - \widetilde{B})(1,g)=0$,即$(1,g)$是矩阵 $(\diag\{a_1, \dots, a_{n+1}\} + \Xi - \widetilde{B})$ 特征值为0对应的特征向量。
    \item[(d)]  矩阵 $(\diag\{a_1, \dots, a_{n+1}\} + \Xi - \widetilde{B})$ 第二小的特征值 大于零。 
\end{enumerate}
由对偶互补条件,可知(a-d) 保证了 
$\sum_{i=1}^{n+1} a_i=\langle(1,g)(1,g)^T,\widetilde{B} \rangle$,
即原问题和对偶问题最优值相等。

首先直接将$(1,g)(1,g)^T$代入验证可知条件 (a) 满足。
下面我们构造  $(a_1,\ldots,a_{n+1},\lambda_1,\ldots,\lambda_{n+1})$  满足条件 (c)。
令 $\mu=\frac{1}{n}\mathbf{1}_n^T h$ 且
$\lambda = -\frac{\mu}{n}$。 先以如下方式取定 $(\lambda_1,\ldots,\lambda_{n+1})$
$$
\lambda_1=\mu+\lambda, \text{  }\lambda_{i+1}=g_i\lambda + \lambda,~i\in\{1,\ldots,n\}
$$
则根据 条件 (c) 解出 $(a_1, \dots, a_{n+1})$ 为
\begin{align}
    a_1 &= h^T g \nonumber\\
    a_{i+1} & = (h_i -\lambda)g_i  + \frac{1}{2}\diag\{Bgg^T\}_i, i = 1, \dots, n
\end{align}
接下来只需证明
\begin{align}
    &\diag(\{a_1, \dots, a_{n+1}\}) + \Xi - \widetilde{B} \nonumber\\
    =& \begin{pmatrix} h^T g & -h^T +(\mu + \lambda)\mathbf{1}_n^T \\
        -h+(\mu + \lambda )\mathbf{1}_n & \Xi_n \end{pmatrix}
    \succeq 0
\end{align}
其中
\begin{align}\label{eq:Xi_n}
    \Xi_n =\diag(hg^T + Agg^T -\lambda \mathbf{1}_ng^T)
    + \left(\frac{1}{2} +2\lambda \right)
    J_n  - A + 2\lambda \Xi'
\end{align}

矩阵 $A=(B+J_n-I_n)/2$ 是图的邻接矩阵,
而矩阵 $\Xi'$ 由 $\Xi'_{ij}=g_i + g_j$ 给出。

\newglossaryentry{cauchy_interlace}
{name=柯西交错定理,
description={Cauchy's interlacing theorem}}

接下来我们证明 $\Xi_n$ 是正定矩阵。
若此结论成立,根据\gls{cauchy_interlace} \cite{hwang},
条件 (d) 成立,
从而完成证明。

我们只需说明 $x^T \Xi_n x>0$ 对 任意 满足 $\norm{x}=1$ 的 $x \in \mathbb{R}^n$ 成立。
我们可以将 给定的 $x$ 分解成 $x=\frac{\beta}{\sqrt{n}} g
+ \sqrt{1-\beta^2} g^{\perp}$, 其中 $g^Tg^{\perp}=0, \beta \in [0,1]$,
且 $\norm{g^{\perp}}=1$。则我们有 
\begin{align}\label{eq:xax}
    x^T \Xi_n x = \frac{\beta^2}{n} g^T \Xi_n g  
    +		\frac{\beta}{\sqrt{n}}\sqrt{1-\beta^2} g^T \Xi_n g^{\perp}
    +
    (1-\beta^2)(g^{\perp})^T \Xi_n g^{\perp}
\end{align}
接下来我们考察 式\eqref{eq:xax} 右端3项
的下界。
对于第一项,我们有
\begin{align*}
    g^T \Xi_n g = g^T(h -\lambda \mathbf{1}_n)   = g^T h
\end{align*}
因为 $\sum_{y \in \mathcal{Y}} \sqrt{P_0(y)P_1(y)} < 1$,
由 切尔诺夫不等式, 
$P(g^T h < 0) \leq (\sum_{y \in \mathcal{Y}} \sqrt{P_0(y)P_1(y)})^{mn}$
随 $n$ 的增大指数衰减。
因此,以接近1的概率,第一项 $g^T \Xi_n g$ 为正。

对于第二项,令  $\tilde{h}=\Xi_n g
=(n-1)\lambda\mathbf{1}_n+h$,则
$g^T \Xi_n g^{\perp} = \tilde{h}^T g^{\perp} \geq -\norm{\tilde{h}-\frac{1}{n}(\tilde{h}^Tg)g}$,
其中 模长 $\norm{\tilde{h}-\frac{1}{n}(\tilde{h}^Tg)g}$ 满足
\begin{align*}
    \norm{\tilde{h}-\frac{1}{n}(\tilde{h}^Tg)g}^2
    =\norm{\tilde{h}}^2 - \frac{1}{n}(\tilde{h}^Tg)^2
\end{align*}
令 $\hat{g}_1 = \frac{1}{2}(g + \mathbf{1}_n)$ 且
$\hat{g}_2 = \frac{1}{2}(-g +\mathbf{1}_n)$。
由 $\tilde{h}^T\mathbf{1}_n=\mu$,我们有
\begin{align*}
    \frac{\norm{\tilde{h}}^2}{n} - \left(\frac{1}{n}\tilde{h}^Tg \right)^2
    &=\frac{\norm{\tilde{h}}^2}{n} - 2\frac{(\tilde{h}^T \hat{g}_1)^2}{n^2} - 2\frac{(\tilde{h}^T \hat{g}_2)^2}{n^2} + \frac{(\tilde{h}^T\mathbf{1}_n)^2}{n^2} \\
    &=2\frac{ \sum_{i<j,i,j\in S_1} (\tilde{h}_i - \tilde{h}_j)^2 + \sum_{i<j,i,j\in S_2} (\tilde{h}_i - \tilde{h}_j)^2 }{n^2}+ \frac{\mu^2}{n^2}\\
    & = I_1 + I_2 + \frac{\mu^2}{n^2}
\end{align*}
其中 对于 $j\in\{1,2\}$,
$I_j=\frac{\sum_{i\in S_j} h_i^2}{n} - 2\frac{(h^T \hat{g}_j)^2}{n^2}$, 
$S_1$ 和 $S_2$ 分别表示标签为 1 和 -1 的节点集合。
对于 $i\in S_j$ 和 $j\in\{1,2\}$
记 $\mathbb{E}[h_i]=m D_j, \Var[h_i]=m \bar{D}_j, 
\mathbb{E}[h_i^2]=\Var[h_i]+\mathbb{E}^2[h_i]=m \bar{D}_j+m^2 D^2_j, \Var[h_i^2] \leq m^4 \bar{C}_j$,
易得
$D_j,\bar{D}_j, \bar{C}_j,j\in\{1,2\}$ 是常数。
由切比雪夫不等式,对于 $j\in\{1,2\}$
我们有
\begin{align*}
    P\left(\Big| \frac{\sum_{i\in S_j} h_i^2}{n} - \frac{1}{2}(m \bar{D}_j + m^2D_j^2) \Big| \geq \log n \right)
    & \leq \frac{\bar{C}_j m^4}{2n\log^2 n} \\
    P\left(\Big| \frac{h^T \hat{g}_j}{n} - \frac{m}{2}D_j\Big| \geq \log^{-1} n \right)
    & \leq \frac{\bar{D}_j m \log^{2} n}{2n}
\end{align*}
因此,对于 $j\in\{1,2\}$ 下式以概率 $1-n^{-1-o(1)}$ 成立。
\begin{align*}
    I_j \leq \frac{1}{2}m\bar{D}_j + \frac{1}{2}m^2 D_j^2 + \log n - 2(\frac{1}{2}m D_j - \log^{-1} n)^2 = O(\log n)
\end{align*}
从而我们得到式\eqref{eq:xax}第二项的一个下界
$$
\frac{1}{\sqrt{n}} g^T \Xi_n g^{\perp} \geq -\sqrt{\frac{\norm{\tilde{h}}^2}{n} - (\frac{1}{n}\tilde{h}^Tg)^2} = O(\sqrt{\log n})
$$
对于最后一项 $(g^{\perp})^T \Xi_n g^{\perp}$,
由 式\eqref{eq:Xi_n} 和 $(g^{\perp})^Tg=0$,我们有
\begin{align*}
    (g^{\perp})^T \Xi_n g^{\perp} 
    =& (g^{\perp})^T \diag( -\lambda \mathbf{1}_ng^T+hg^T + Agg^T) g^{\perp} + p \\
    +& \frac{1}{2}(1-p-q+2\lambda)(g^{\perp})^TJ_n g^{\perp}
    -(g^{\perp})^T(A-\E[A])g^{\perp},
\end{align*}
在推导上式中我们用到了 $\E[A] = \frac{p-q}{2}gg^T + \frac{p+q}{2}J_n - pI_n$。
注意到 $p,q,$ 以及 $\lambda$ 都是 $o(1)$ 并且
$(g^{\perp})^TJ_n g^{\perp}\ge 0$。
由 文献\inlinecite{lei2015consistency} 中的定理 5.2,
以概率 $1-n^{-r}$, 存在
常数 $r$ 和 $c$,使得不等式 $\lambda_{\max}(A-\mathbb{E}[A]) \leq c\sqrt{\log n}$
成立。这里$\lambda_{\max}$ 表示矩阵的最大特征值。
因此,我们有
\begin{align}\label{eq:lastterm}
    (g^{\perp})^T \Xi_n g^{\perp} \geq \min_i\{(-\lambda + h_i) g_i+\lambda + g_i (Ag)_i \} - c \sqrt{\log n}
\end{align}
% The error probability is then bounded by
% \begin{align*}
% P(\Xi_{ii} \leq c\sqrt{\log n}, \forall 1\leq i \leq n)
% \end{align*}
下面我们估计下式对于 $i\in [n]$ 均成立的概率
\begin{align}\label{eq:nodei}
    (-\lambda + h_i) g_i+\lambda + g_i (Ag)_i  - c \sqrt{\log n}\ge 0
\end{align}
对于 $g_i=1$,
项 $g_i(Ag)_i$ 可以写成两个独立的二项分布的差的形式:
$A_r^0-A_r^1$,其中 $A_r^0, A_r^1$ 定义见 式\eqref{eq:energy_diff}, 参数 $k=2$。
$h_i$ 可以写成 $h_i=m \frac{D(P_{\widetilde{Y}_i} || P_1) - D(P_{\widetilde{Y}_i} || P_0) }{\log (a /b)}$
的形式,
其中 $P_{\widetilde{Y}_i}$ 是
样本 $\{Y^i_j\}^m_{j=1}$ 在节点 $i$ 处的经验分布。
类似定理 \ref{thm:constant} 的证明开始时
使用的方法,
根据 引理 \ref{lem:zxt}  和  Sanov 定理,
对于使得 $g_i=1$ 的 $i$, 
式\eqref{eq:nodei} 不成立的概率
的上界为
$n^{-I_1 + o(1)}$,
其中 $I_1$ 在 式 \eqref{eq:I1} 中定义。
类似地,
对于 满足 $g_i=-1$ 的  $i$,
式 \eqref{eq:nodei} 不成立的概率的上界为 
$n^{-I_2 + o(1)}$。
由引理 \ref{lem:p0p12},$I_1=I_2=I_+>1$。
由布尔不等式, 存在 $i$ 使得 式 \eqref{eq:nodei} 不成立的概率 不超过 $n^{-I_+ + 1 + o(1)}$。
由条件 $I_+>1$,$n^{-I_+ + 1 + o(1)}$ 随$n$的增大 衰减到 $0$。
因此,以接近1的概率 $(g^{\perp})^T \Xi_n g^{\perp}\ge 0$,从而证得定理 \ref{thm:sdp}。 


