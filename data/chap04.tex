% !TeX root = ../thuthesis-example.tex

\chapter{随机块模型的精确恢复问题}


\section{基于玻茨模型的算法}
类似于\ref{sec:sibm_model}节
介绍的SIBM模型,我们将
随机块模型
与 \ref{sec:ising} 节介绍的 玻茨模型
复合起来可得到随机块-玻茨模型。该模型可以看成是SIBM模型的推广,
每个节点由两状态变成了多状态。由于我们的研究重点并非
从随机块-玻茨模型的多个样本恢复节点的原始标签,而是利用
玻茨模型对随机块模型进行社群发现,因此我们首先给出
由随机块模型生成的图上定义的玻茨模型:
\begin{definition}\label{def:ising}
	给定从$\SSBM(n,k,\A,\B)$ 中生成的随机图 $G$,
    定义在$G$上的玻茨模型($k$个状态的伊辛模型)带有两个参数 $\gamma,\beta>0$,
	并且是关于状态向量$\sigma\in W^n$ 的概率分布
。该分布的概率质量函数为
\begin{align} \label{eq:isingma}
	P_{\sigma|G}(\sigma=\bar{\sigma})=\frac{\exp(-\beta H(\bar{\sigma}))}{Z_G(\gamma,\beta)}
	\end{align}
其中能量函数$H(\bar{\sigma})$的定义是
\begin{equation}\label{eq:energy}
	H(\bar{\sigma}) := \gamma \frac{\log n}{n} \sum_{\{i,j\}\not\in E(G)} \delta(\bar{\sigma}_i, \bar{\sigma}_j)
	- \sum_{\{i,j\}\in E(G)} \delta(\bar{\sigma}_i, \bar{\sigma}_j)
	\end{equation}
	
	$P_{\sigma|G}$ 中的下标表示该分布依赖于$G$,
    $Z_G(\gamma,\beta)$ 是该分布的归一化常数。
\end{definition}

式\eqref{eq:isingma}中
各符号的含义同式\eqref{eq:canonical_ensemble}。
而表示汉密尔顿能量的式\eqref{eq:energy}
则是式\eqref{eq:ising_modified}的推广。
当  $k=2$ 时 式 \eqref{eq:energy}
在评注\ref{rem:equivalence_H_energy}的意义下
退化成 式 \eqref{eq:ising_modified}。 

这里的能量函数$H(\bar{\sigma})$ 同样有两部分组成:
没有边相连的节点之间排斥力的势能和
有边相连的节点之间吸引力的势能。
$\gamma$ 参数衡量了两种势能之间的比率而
$\frac{\log n}{n}$ 则是对由于边的数量不均匀造成两种势能量阶不同的修正项,
因为两节点间有边相连的概率仅为 $O(\frac{\log n}{n})$。

定义 \ref{def:ising} 实际上给出了一个估计$X$的随机算法 $\hat{X}^*$。
这里,$\hat{X}^*$ 表示产生于玻茨模型的一个样本,
记作$\hat{X}^* \sim \textrm{Potts}_G(\gamma, \beta)$。
沿用评注\ref{rem:metric_exact_recovery} 中的记号,
使用$\hat{X}^*$估计$X$在精确恢复度量下的错误概率
记为 $P_e(\hat{X}^*) := \sum_{G \in \cG_n} P_G(G) P_{\sigma | G}(S^c_k(X))$。
类似定理\ref{thm:sibm_phase_trans}中关于 SIBM 模型的讨论,
玻茨模型的两个参数$(\gamma, \beta)$ 的取值
对$P_e(\hat{X}^*)$
也有着决定性的作用。
当它们取合适的值时, 
$ P_e(\hat{X}^*)\to 0$,
随机块模型的精确恢复可以实现。
反之,如果 $(\gamma, \beta)$ 取其他值时,
$P_e(\hat{X}^*) \to 1$。
这两种情况可总结为如下的定理:

\begin{theorem}\label{thm:phase_transition}
	假设 $\sqrt{a} - \sqrt{b} > \sqrt{k}$,
	定义函数 $g(\beta)$ 和 $ \tilde{g}(\beta)$ 如下:
	\begin{equation}
		\label{eq:g_beta_main_article}
		g(\beta) = \frac{be^{\beta} + a e^{-\beta}}{k} - \frac{a+b}{k} +1
	\end{equation}
	且
	\begin{equation}
		\label{eq:g_tilde_beta_main_article}
	\tilde{g}(\beta) = \begin{cases}
	g(\beta) & \beta \leq \bar{\beta} = \frac{1}{2}\log \frac{a}{b} \\
	g(\bar{\beta}) = 1 - \frac{(\sqrt{a} - \sqrt{b})^2}{k} & \beta > \bar{\beta}
	\end{cases}
	\end{equation}
	其中
	$\bar{\beta} =  \displaystyle\arg\min_{\beta > 0} g(\beta)$。
	令 $\beta^*$ 定义成
	\begin{equation}\label{eq:beta_star}
	\beta^* = \log\left(\frac{a + b - k - \sqrt{(a + b - k)^2 - 4 a b)}}{2  b}\right)
	\end{equation}
	可以验证 $\beta^*$ 是方程 $g(\beta) = 0$ 的解 并且满足  $\beta^* < \bar{\beta}$。
	则取决于 $(\gamma, \beta)$ 如何取值,对于给定的 $\epsilon > 0$, 
	$G\sim \SSBM(n, k, \A, \B)$, $\hat{X}^* \sim \textrm{Potts}_G(\gamma, \beta)$,
	当 $n$ 充分大时, 我们有:
	\begin{enumerate}
	\item 当 $\gamma > b$ 且 $\beta > \beta^*$ 时,$P_e(\hat{X}^*) \leq n^{\tilde{g}(\beta)/2 + \epsilon}$;
	\item 当 $\gamma > b$ 且 $\beta < \beta^*$ 时,$P_a(\hat{X}^*) \leq (1+o(1))\max\{n^{g(\bar{\beta})}, n^{-g(\beta) + \epsilon}\}$;
	\item 当 $\gamma < b$ 时,对于任意给定的 $C>0$	均有 $P_a(\hat{X}^*) \leq \exp(-C n)$。
	\end{enumerate}
\end{theorem}

\begin{figure}[H]
	\begin{subfigure}{0.43\textwidth}
		\includegraphics[width=\textwidth]{g-16-4-2.pdf}
		\caption{当 $a=16,b=4,k=2$时,$g(\beta)$和
		$\tilde{g}(\beta)$的函数图像}\label{fig:g}
	\end{subfigure}~
	\begin{subfigure}{0.55\textwidth}
		\includegraphics[width=\textwidth]{phase_trans.pdf}
		\caption{
			在 $(\beta, \gamma)$ 平面内
			相变区域图示。
			随机块模型的精确
			恢复仅在区域 I 可实现。}\label{fig:pt}
	\end{subfigure}
	\caption{定理 \ref{thm:phase_transition}
	的图示}
	\label{fig:phase_transition_theorem_illustration}
\end{figure}

与式\eqref{eq:beta_star_sibm}的情形类似,注意到条件$\sqrt{a} - \sqrt{b} > \sqrt{k}$保证了
式\eqref{eq:beta_star}中的根号下的项非负。
这个条件来自于定理\ref{thm:sbmk_phase_transition},
保证了随机块模型本身精确恢复可实现。


通过简单的计算可知,对于 $\beta> \beta^*$ 我们有
$\tilde{g}(\beta) < 0$ 而 对于 $\beta < \beta^*$有
$g(\beta)>0$。
另外,由 $\sqrt{a} - \sqrt{b} > \sqrt{k}$ 可知
$g(\bar{\beta}) < 0$。
$g(\beta), \tilde{g}(\beta)$ 的图像
如图 \ref{fig:phase_transition_theorem_illustration}a 所示。
因此, 对于充分小的
$\epsilon$ 并且当 $n \to \infty$,
定理\ref{thm:phase_transition} 中给出的上界
均至少以多项式的速度趋向于 $0$。
从而定理\ref{thm:phase_transition} 
刻划了
玻茨模型的相变性质。
如图\ref{fig:phase_transition_theorem_illustration}b所示
, 对于玻茨模型, 只有参数$(\beta, b)$落在区域I时,
随机块模型的精确
恢复才能实现。

定理\ref{thm:phase_transition}
也可以从$\sigma$的边缘分布的角度理解。
玻茨模型取到某个特定状态$\bar{\sigma}$的概率为
 $\sigma: P_{\sigma}(\sigma =\bar{\sigma})
=\sum_{G \in \cG_n}P_G(G)P_{\sigma |G}(\sigma=\bar{\sigma})$.
\newglossaryentry{not:sigma_nearest_to_the_latter}
{
  type=notation,
  name={\ensuremath{D(\sigma, \sigma')}},
  description={$\sigma$ 相比于它的所有置换其本身离 $\sigma'$ 最近
  这个事件}
}
我们用符号 \gls{not:sigma_nearest_to_the_latter} 表示 \glsdesc{not:sigma_nearest_to_the_latter}
。
也即
\begin{equation}
	\label{eq:D_sigma_sigma_prime}
D(\sigma, \sigma') := \{ \sigma = \arg\min_{f \in S_k} \Dist(f(\sigma), \sigma')  \}
\end{equation}

则 定理 \ref{thm:phase_transition} 
对边缘分布 $P_{\sigma}$ 蕴含着如下的论断
:
\begin{corollary}\label{cor:phase4}
假设 $\gamma > b$, 取决于 $\beta$ 的取值我们有
\begin{enumerate}
	\item 当 $\beta > \beta^*$时,$P_{\sigma}(\sigma = X | D(\sigma, X))  = 1-o(1)$;
	\item 当 $\beta < \beta^*$时,$P_{\sigma}(\sigma = X | D(\sigma, X))  = o(1)$。
\end{enumerate}
\end{corollary}

下面我们简单阐述下 定理 \ref{thm:phase_transition} 证明的思路。
该思路主要通过对翻转一个状态的前后能量差的分析来估计概率的主项。
下面的引理总结了翻转一个状态汉密尔顿能量的变化:
\begin{lemma}\label{lem:lemmaDiff}
	假设 $\bar{\sigma}'$ 仅在第$r$个位置上与 $\bar{\sigma}$ 不同,
	差别为 $\bar{\sigma}'_r = \omega^s \cdot \bar{\sigma}_r$。
	则 $\bar{\sigma}'$ 和 $\bar{\sigma}$ 的能量差为
\begin{align}
	H(\bar{\sigma}') - H(\bar{\sigma}) &= (1+\gamma \frac{\log n}{n})\sum_{i \in N_r(G)} J_s(\bar{\sigma}_r, \bar{\sigma}_i)
	\notag \\
	&+ \gamma \frac{\log n}{n} (m(\omega^s \cdot \bar{\sigma}_r)-m(\bar{\sigma}_r)+1) \label{eq:DeltaH}
	\end{align}
	上式中 $m(\omega^j) := |\{i \in [n] | \bar{\sigma}_i = \omega^j | \}$, $N_r(G):=\{j | (r, j) \in E(G) \}$ 且 $J_s(x, y) = \delta(x, y) - \delta(\omega_s \cdot x, y)$.
\end{lemma}
引理 \ref{lem:lemmaDiff} 提供了用以比较两个相邻状态的    概率
的方法,即:
\begin{equation}\label{eq:Pratio}
\frac{P_{\sigma |G } (\sigma = \bar{\sigma}')}{P_{\sigma |G } (\sigma = \bar{\sigma})}
= \exp(-\beta(H(\bar{\sigma}') - H(\bar{\sigma})))
\end{equation}

另外, 我们注意到 因为图的边的数量是稀疏的,每个节点
平均有 $O(\log n)$ 个相连的邻居节点。
按照式 \eqref{eq:DeltaH} 
计算能量差的时间复杂度也是 $O(\log n)$。

当 $H(\bar{\sigma}') > H(\bar{\sigma})$ 时, 
由式\eqref{eq:Pratio} 可得
$P_{\sigma | G}(\sigma = \bar{\sigma}')$
小于
$P_{\sigma | G}(\sigma = \bar{\sigma})$。
粗略而言, 如果
$ \sum_{\Dist(\sigma', X)=1}\exp(-\beta(H(\bar{\sigma}') - H(X))) $
收敛到零,
我们可以预期,与$S_k(X)$不同的所有其他状态的概率收敛到零。
与之相反, 如果
$ \sum_{\Dist(\sigma', X)=1}\exp(-\beta(H(\bar{\sigma}') - H(X))) $
趋于无穷大,
则 $P_{\sigma}(S_k(X))$ 趋于零。
此种分析展示了证明定理 \ref{thm:phase_transition}的
主要思路。
严格的证明参见附录 \ref{sec:appendix_theorem_proof_phase_trans}。

\section{通过能量最小化进行社群发现}\label{sec:em}
因为 $\beta^*$ 和 $n$ 无关,
当 $\gamma>b$ 时,
我们可以选取一个足够大的 $\beta$ 使得
$\beta > \beta^*$,
则 由 定理\ref{thm:phase_transition}, 当$n$充分大时 $\sigma \in S_k(X)$ 几乎必然发生。
这蕴含了 $P_{\sigma | G}(\sigma = X)$
对于几乎所有的 从 SBM 采样的图  $G$来说 取得最大值。
因此
相比于从伊辛模型中采样的方法,我们可以直接最大化条件概率,以
找到概率最大值的状态。
等价地, 我们可以通过最小化 \eqref{eq:energy}式的方式来获得状态的估计量:
\begin{equation}\label{eq:hatX}
\hat{X}' := \arg\min_{\bar{\sigma} \in W^n} H(\bar{\sigma})
\end{equation}

在 \eqref{eq:hatX} 中, 我们让 $\bar{\sigma}$ 在 $W^n$ 中取值。
因为我们已知 $X$取各个标签的位置数相等,即对于每个标签值 $u$
有 $|\{v \in [n] : X_v = u\}| = \frac{n}{k}$,
我们可以把搜索空间限制到
$W^*:= \{\sigma\in W^n \big\vert |\{v \in [n] : \sigma_v = \omega^s\}| = \frac{n}{k}, s=0,\dots, k-1 \}$
上。
当 $\sigma \in W^*$ 时, 最小化 $H(\sigma)$ 等价于:
\begin{equation}\label{eq:hatX_double_prime}
\hat{X}'' := \arg\min_{\sigma \in W^*} \sum_{\{i,j\} \not\in E(G) } \delta(\sigma_i, \sigma_j)
\end{equation}
其中最小值是不同社群之间的最小分割。

当 $\hat{X}'' \neq X$ 时,
我们必须有  $\Dist(\hat{X}'' ,X)\geq 2$
以满足 硬约束 $\hat{X}'' \in W^*$。
此外,估计量 $\hat{X}''$ 是无参的而 $\hat{X}'$的值受
参数 $\gamma$ 的影响。
在
$\hat{X}'$ 的表达式中出现的额外的参数 $\gamma$ 可视为
某种整数规划的拉格朗日乘子。
因此,通过引入惩罚因子项和将搜索空间从$W^*$ 扩大 到$W^n$
,求解 $\hat{X}'$
的优化问题 转变为 $\hat{X}''$的优化问题。

当 $\beta > \bar{\beta}$ 时,
$\tilde{g}(\beta)$ 是一个常数。
因此,从 Theorem \ref{thm:phase_transition} 中我们可以获得的关于
伊辛模型的估计量 $\hat{X}^*$ 最紧的上界
是  $n^{g(\bar{\beta})/2}$。

对于 $\hat{X}'$ 和 $\hat{X}''$ 这两个估计量,
我们可以获得 更紧的误差上界。我们把这一结果总结为如下定理:
\begin{theorem}\label{thm:error_rate}
当 $\sqrt{a} - \sqrt{b} > \sqrt{k}$ 时,
对于充分大的 $n$,我们有 
\begin{enumerate}
	\item 若 $\gamma > b$, $P_G(\hat{X}' \not\in S_k(X)) \leq (k-1+o(1))n^{g(\bar{\beta})}$;
	\item $P_G(\hat{X}'' \not\in S_k(X)) \leq ((k-1)^2+o(1))n^{2g(\bar{\beta})}$。
\end{enumerate}
\end{theorem}
因为 $g(\bar{\beta})<0$, 我们有大小关系 $n^{2g(\bar{\beta})} < n^{g(\bar{\beta})} < n^{g(\bar{\beta})/2}$。
从而 定理 \ref{thm:error_rate} 说明了在三个估计量中,
$P_e(\hat{X}'')$ 
的上界最紧。
这可以直观地理解为搜索空间缩小的结果。
定理\ref{thm:error_rate} 的证明技术是对于 $\Dist(\bar{\sigma}, X) \geq 1$,考虑事件
$H(X) > H(\bar{\sigma})$发生的概率。
然后通过布尔不等式, 这些概率可以相加。
关于估计量 $\hat{X}''$ 误差上界的研究早在\citet{abbe2015exact} 的工作中即有涉及,
但他只得到了一个比较松的上界 $n^{g(\bar{\beta})/4}$,且仅针对 $k=2$ 的情形。
对于一般的情形,
当条件
$\sqrt{a} - \sqrt{b} > \sqrt{k}$ 满足时,考虑到
$\tilde{g}(\beta) = 1- \frac{(\sqrt{a} - \sqrt{b})^2}{k}$,
定理 \ref{thm:error_rate} 说明了  $\hat{X}'$ 和 $\hat{X}''$
均可实现
精确恢复。


估计量 $\hat{X}'$ 有一个参数 $\gamma$。
当 $\gamma$ 取特定的值时, $\hat{X}'$在渐近情形下
可以等价于 最大似然估计或最大模块度估计。
以下分析从直观地角度说明了
这种联系。

通过最大化对数似然函数可得到最大似然估计量。
由 \eqref{eq:GmL} 式,该似然函数可以进一步写成:
\begin{equation}\label{eq:PG_energy}
	\log P_G(Z|X=\sigma) = -\log\frac{a}{b} \cdot H(\sigma) + C
\end{equation}

这里 在$H(\sigma)$ 的表达式中 参数 $\gamma$ 的取值满足
\begin{equation}\label{eq:special_gamma_ML}
	\gamma \frac{\log n}{n} = \frac{1}{\log(a/b)}(\log (1-\B) - \log (1-\A))	 
\end{equation}

而 $C$ 是一个和 $\sigma$ 无关的常数。
当 $n$ 充分大时,我们有 $\gamma \to \gamma_{ML} := \frac{a-b}{\log(a/b)}$。   
That is, the  maximum likelihood estimator is equivalent to $\hat{X}'$ when $\gamma = \gamma_{ML}$ asymptotically.

这里我们顺便得到了最大似然算法和最小割问题在有$W^*$约束
下的等价性,这个等价性不需要渐近的条件也成立。
因为最大似然算法首先等价于极小化能量$H(\sigma)$,
$\gamma$从\eqref{eq:special_gamma_ML}式中取值。其次,
由\eqref{eq:hatX_double_prime}式,极小化能量$H(\sigma)$
等价于$\min \sum_{\{i,j\} \not\in E(G) } \delta(\sigma_i, \sigma_j)$
因为在$W^*$的约束下,
$\sum_{1<i<j<n} \delta(\sigma_i, \sigma_j)$
是一个常数,\eqref{eq:hatX_double_prime}式又等价于
$-\sum_{ \{i,j\} \in E} \delta(\sigma_i, \sigma_j)$,
其与 \eqref{eq:minimum_k_cut}式只相差一个常数。

The maximum modularity estimator is obtained by maximizing the modularity of a graph~\cite{clauset2004finding}, which is defined by:
See \ref{eq:Q}

For the $i$-th node, $d_i$ is its degree and $C_i$ is its community belonging. $A$ is the adjacency matrix.
Up to a scaling factor, the~modularity $Q$ can be re-written using the label vector $\sigma$~as:
\begin{align}
Q(\sigma) = &-\sum_{\{i,j\} \not\in E(G) } \frac{d_i d_j}{2 |E|}\delta(\sigma_i,\sigma_j) \notag \\
&+ \sum_{\{i,j\} \in E(G) } (1 - \frac{d_i d_j}{2 |E|}) \delta(\sigma_i,\sigma_j)  \label{eq:Qtransform}
\end{align}

From \eqref{eq:Qtransform}, we can see that $Q(\sigma) \to -H(\sigma)$ with $\gamma = \gamma_{MQ} = \frac{a+b}{2}$ as $n\to \infty$.
Indeed, we have $d_i \sim \frac{(a+b)\log n}{2}, |E| \sim \frac{1}{2}n d_i$. Therefore, we have $\frac{d_id_j}{2|E|} \to \gamma_{MQ} \frac{\log n}{n} $. That is, the maximum modularity estimator is equivalent with $\hat{X}'$ when $\gamma = \gamma_{MQ}$ asymptotically.


Using $a>b$ and the inequality $x-1>\log x > 2 \frac{x-1}{x+1}$ for $x>1$ we can verify that $\gamma_{MQ} > \gamma_{ML} > b$. That is, both the maximum likelihood and the maximum modularity estimator satisfy the exact recovery conditions $\gamma > b $ in Theorem \ref{thm:error_rate}.

\section{参数估计}

