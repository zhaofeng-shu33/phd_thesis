% !TeX root = ../thuthesis-example.tex

\chapter{随机块模型的精确恢复问题}


\section{随机块-玻茨模型}
类似于\ref{sec:sibm_model}节
介绍的SIBM模型,我们将
随机块模型
与 \ref{sec:ising} 节介绍的 玻茨模型
复合起来可得到随机块-玻茨模型。该模型可以看成是SIBM模型的推广,
每个节点由两状态变成了多状态。由于我们的研究重点并非
从随机块-玻茨模型的多个样本恢复节点的原始标签,而是利用
玻茨模型对随机块模型进行社群发现,因此我们首先给出
由随机块模型生成的图上定义的玻茨模型:
\begin{definition}\label{def:ising}
	给定从$\SSBM(n,k,\A,\B)$ 中生成的随机图 $G$,
    定义在$G$上的玻茨模型($k$个状态的伊辛模型)带有两个参数 $\gamma,\beta>0$,
	并且是关于状态向量$\sigma\in W^n$ 的概率分布
。该分布的概率质量函数为
\begin{align} \label{eq:isingma}
	P_{\sigma|G}(\sigma=\bar{\sigma})=\frac{\exp(-\beta H(\bar{\sigma}))}{Z_G(\gamma,\beta)}
	\end{align}
其中能量函数$H(\bar{\sigma})$的定义是
\begin{equation}\label{eq:energy}
	H(\bar{\sigma}) := \gamma \frac{\log n}{n} \sum_{\{i,j\}\not\in E(G)} \delta(\bar{\sigma}_i, \bar{\sigma}_j)
	- \sum_{\{i,j\}\in E(G)} \delta(\bar{\sigma}_i, \bar{\sigma}_j)
	\end{equation}
	
	$P_{\sigma|G}$ 中的下标表示该分布依赖于$G$,
    $Z_G(\gamma,\beta)$ 是该分布的归一化常数。
\end{definition}

式\eqref{eq:isingma}中
各符号的含义同式\eqref{eq:canonical_ensemble}。
而表示汉密尔顿能量的式\eqref{eq:energy}
则是式\eqref{eq:ising_modified}的推广。
当  $k=2$ 时 式 \eqref{eq:energy}
在评注\ref{rem:equivalence_H_energy}的意义下
退化成 式 \eqref{eq:ising_modified}。 

这里的能量函数$H(\bar{\sigma})$ 同样有两部分组成:
没有边相连的节点之间排斥力的势能和
有边相连的节点之间吸引力的势能。
$\gamma$ 参数衡量了两种势能之间的比率而
$\frac{\log n}{n}$ 则是对由于边的数量不均匀造成两种势能量阶不同的修正项,
因为两节点间有边相连的概率仅为 $O(\frac{\log n}{n})$。

定义 \ref{def:ising} 实际上给出了一个估计$X$的随机算法 $\hat{X}^*$。
这里,$\hat{X}^*$ 表示产生于玻茨模型的一个样本,
记作$\hat{X}^* \sim \textrm{Potts}_G(\gamma, \beta)$。
沿用评注\ref{rem:metric_exact_recovery} 中的记号,
使用$\hat{X}^*$估计$X$在精确恢复度量下的错误概率
记为 $P_e(\hat{X}^*) := \sum_{G \in \cG_n} P_G(G) P_{\sigma | G}(S^c_k(X))$。
类似定理\ref{thm:sibm_phase_trans}中关于 SIBM 模型的讨论,
玻茨模型的两个参数$(\gamma, \beta)$ 的取值
对$P_e(\hat{X}^*)$
也有着决定性的作用。
当它们取合适的值时, 
$ P_e(\hat{X}^*)\to 0$,
随机块模型的精确恢复可以实现。
反之,如果 $(\gamma, \beta)$ 取其他值时,
$P_e(\hat{X}^*) \to 1$。
这两种情况可总结为如下的定理:

\begin{theorem}\label{thm:phase_transition}
	定义函数 $g(\beta)$ 和 $ \tilde{g}(\beta)$ 如下:
	\begin{equation}
	g(\beta) = \frac{be^{\beta} + a e^{-\beta}}{k} - \frac{a+b}{k} +1
	\end{equation}
	且
	\begin{equation}
	\tilde{g}(\beta) = \begin{cases}
	g(\beta) & \beta \leq \bar{\beta} = \frac{1}{2}\log \frac{a}{b} \\
	g(\bar{\beta}) = 1 - \frac{(\sqrt{a} - \sqrt{b})^2}{k} & \beta > \bar{\beta}
	\end{cases}
	\end{equation}
	其中
	$\bar{\beta} =  \displaystyle\arg\min_{\beta > 0} g(\beta)$。
	令 $\beta^*$ 定义成
	\begin{equation}\label{eq:beta_star}
	\beta^* = \log\left(\frac{a + b - k - \sqrt{(a + b - k)^2 - 4 a b)}}{2  b}\right)
	\end{equation}
	可以验证 $\beta^*$ 是方程 $g(\beta) = 0$ 的解 并且满足  $\beta^* < \bar{\beta}$。
	则取决于 $(\gamma, \beta)$ 如何取值,对于给定的 $\epsilon > 0$, 
	$G\sim \SSBM(n, k, \A, \B)$, $\hat{X}^* \sim \textrm{Potts}_G(\gamma, \beta)$,
	当 $n$ 充分大时, 我们有:
	\begin{enumerate}
	\item 当 $\gamma > b$ 且 $\beta > \beta^*$ 时,$P_e(\hat{X}^*) \leq n^{\tilde{g}(\beta)/2 + \epsilon}$;
	\item 当 $\gamma > b$ 且 $\beta < \beta^*$ 时,$P_a(\hat{X}^*) \leq (1+o(1))\max\{n^{g(\bar{\beta})}, n^{-g(\beta) + \epsilon}\}$;
	\item 当 $\gamma < b$ 时,对于任意给定的 $C>0$	均有 $P_a(\hat{X}^*) \leq \exp(-C n)$。
	\end{enumerate}
\end{theorem}

\section{参数估计}

\section{与其他算法的联系}

在附录中,我们给出一个这个论断的数学推导。