% !TeX root = ../thuthesis-example.tex

\chapter{背景知识}

\section{社群发现}

社群发现(community detection)虽然没有严格的数学定义,但公认的做法是
将现实世界的社群建模成图(graph)结构,在图结构上进行社群发现。社群根据其属性
是否随时间变化可分为静态社群和动态社群,本文研究的是静态社群,它可以用
静态图来建模,以下简称图。
一个图结构$G$ 由节点集 $V$ 和边的集合 $E$ 组成,可记为 $G(V,E)$。
根据边是否有方向可将图分为有向图和无向图。无向图可以看成是每条边
有两个方向,从而针对有向图设计的社群发现算法也可以用到无向图上面来。
此外,根据边是否有权值可将图分为有权图和无权图。有权图中节点 $i$
和节点 $j$ 之间的权值可以用 $w_{ij}$ 表示。若有权图有方向,
则 $w_{ij}$ 可以不等于 $w_{ji}$。
无向图可以看成是每条边
的权值是1,从而针对有权图设计的社群发现算法也可以用到无权图上面来。

社群发现算法根据发现的社群是否重叠可分为两类,本文研究的是非重叠的
社群发现算法,即$V$可以分为互不重叠的若干子集。

不同的社群发现算法根据自己的标准寻找社群结构,这给一般意义上的比较造成了困难。
在本节中我们介绍的社群发现算法采用的标准叫做最小化平均误差。
这个概念最早是日本学者永野清仁于2010年提出\cite{mac},但其数学形式早已有之。
在这里我们采用其最为广泛的形式,即针对带权的有向图的社群发现算法。
\begin{equation}\label{eq:IP}
  \lambda(1) := \min_{\P}\frac{ f[\P] }{  \abs{\P} - 1 } 
\end{equation}
在式 \eqref{eq:IP} 中,
$\P$ 表示 $V$的一个分割, 也即  $P=\{C_1, \dots, C_k\}, \cup_{i=1}^k C_i=V$。
$\abs{\mathcal{P}}$ 表示集合 $\P$ 中元素的个数。作用在$V$的子集$C$上的函数$f$
表示$C$和 $C$的补集 $V\backslash C$ 之间边的权值之和,即
$f(C)=\sum_{i \not\in C, j\in C, (i,j) \in E} w_{ij}$.
而作用在$V$的分割$\P$上的函数 $f[\P]$ 则是$f$分别作用$\P$中每一个元素之和。
即 $f[\P]=\sum_{C \in \P} f(C)$。有了如上的符号解释,
式 \eqref{eq:IP} 即不难理解,它表示各分割之间所有边的和对分割总数的平均值。

式 \eqref{eq:IP} 实际给出了一个社群发现的标准,对于扁平化的社群而言,
最小值$\lambda(1)$ 对应的社群为
分割$\P^*$的子集。

下面介绍如何求解 $\lambda(1)$ 及其对应的分割$\P^*$。xx的研究指出,
求解 \eqref{eq:IP} 等价于求解下面的组合优化问题

图片通常在 \env{figure} 环境中使用 \cs{includegraphics} 插入,如图~\ref{fig:example} 的源代码。
建议矢量图片使用 PDF 格式,比如数据可视化的绘图;
照片应使用 JPG 格式;
其他的栅格图应使用无损的 PNG 格式。
注意,LaTeX 不支持 TIFF 格式;EPS 格式已经过时。

\begin{figure}
  \centering
  \includegraphics[width=0.5\linewidth]{example-image-a.pdf}
  \caption*{国外的期刊习惯将图表的标题和说明文字写成一段,需要改写为标题只含图表的名称,其他说明文字以注释方式写在图表下方,或者写在正文中。}
  \caption{示例图片标题}
  \label{fig:example}
\end{figure}

若图或表中有附注,采用英文小写字母顺序编号,附注写在图或表的下方。
国外的期刊习惯将图表的标题和说明文字写成一段,需要改写为标题只含图表的名称,其他说明文字以注释方式写在图表下方,或者写在正文中。

如果一个图由两个或两个以上分图组成时,各分图分别以 (a)、(b)、(c)...... 作为图序,并须有分图题。
推荐使用 \pkg{subcaption} 宏包来处理, 比如图~\ref{fig:subfig-a} 和图~\ref{fig:subfig-b}。

\begin{figure}
  \centering
  \subcaptionbox{分图 A\label{fig:subfig-a}}
    {\includegraphics[width=0.35\linewidth]{example-image-a.pdf}}
  \subcaptionbox{分图 B\label{fig:subfig-b}}
    {\includegraphics[width=0.35\linewidth]{example-image-b.pdf}}
  \caption{多个分图的示例}
  \label{fig:multi-image}
\end{figure}



\section{表格}

表应具有自明性。为使表格简洁易读,尽可能采用三线表,如表~\ref{tab:three-line}。
三条线可以使用 \pkg{booktabs} 宏包提供的命令生成。

\begin{table}
  \centering
  \caption{三线表示例}
  \begin{tabular}{ll}
    \toprule
    文件名          & 描述                         \\
    \midrule
    thuthesis.dtx   & 模板的源文件,包括文档和注释 \\
    thuthesis.cls   & 模板文件                     \\
    thuthesis-*.bst & BibTeX 参考文献表样式文件    \\
    \bottomrule
  \end{tabular}
  \label{tab:three-line}
\end{table}

表格如果有附注,尤其是需要在表格中进行标注时,可以使用 \pkg{threeparttable} 宏包。
研究生要求使用英文小写字母 a、b、c……顺序编号,本科生使用圈码 ①、②、③……编号。

\begin{table}
  \centering
  \begin{threeparttable}[c]
    \caption{带附注的表格示例}
    \label{tab:three-part-table}
    \begin{tabular}{ll}
      \toprule
      文件名                 & 描述                         \\
      \midrule
      thuthesis.dtx\tnote{a} & 模板的源文件,包括文档和注释 \\
      thuthesis.cls\tnote{b} & 模板文件                     \\
      thuthesis-*.bst        & BibTeX 参考文献表样式文件    \\
      \bottomrule
    \end{tabular}
    \begin{tablenotes}
      \item [a] 可以通过 xelatex 编译生成模板的使用说明文档;
        使用 xetex 编译 \file{thuthesis.ins} 时则会从 \file{.dtx} 中去除掉文档和注释,得到精简的 \file{.cls} 文件。
      \item [b] 更新模板时,一定要记得编译生成 \file{.cls} 文件,否则编译论文时载入的依然是旧版的模板。
    \end{tablenotes}
  \end{threeparttable}
\end{table}

如某个表需要转页接排,可以使用 \pkg{longtable} 宏包,需要在随后的各页上重复表的编号。
编号后跟表题(可省略)和“(续)”,置于表上方。续表均应重复表头。

\begin{longtable}{cccc}
    \caption{跨页长表格的表题} \\
    \toprule
    表头 1 & 表头 2 & 表头 3 & 表头 4 \\
    \midrule
  \endfirsthead
    \caption[]{跨页长表格的表题(续)} \\
    \toprule
    表头 1 & 表头 2 & 表头 3 & 表头 4 \\
    \midrule
  \endhead
    \bottomrule
  \endfoot
  Row 1  & & & \\
  Row 2  & & & \\
  Row 3  & & & \\
  Row 4  & & & \\
  Row 5  & & & \\
  Row 6  & & & \\
  Row 7  & & & \\
  Row 8  & & & \\
  Row 9  & & & \\
  Row 10 & & & \\
\end{longtable}



\section{算法}

算法环境可以使用 \pkg{algorithms} 或者 \pkg{algorithm2e} 宏包。

\renewcommand{\algorithmicrequire}{\textbf{输入:}\unskip}
\renewcommand{\algorithmicensure}{\textbf{输出:}\unskip}

\begin{algorithm}
  \caption{Calculate $y = x^n$}
  \label{alg1}
  \small
  \begin{algorithmic}
    \REQUIRE $n \geq 0$
    \ENSURE $y = x^n$

    \STATE $y \leftarrow 1$
    \STATE $X \leftarrow x$
    \STATE $N \leftarrow n$

    \WHILE{$N \neq 0$}
      \IF{$N$ is even}
        \STATE $X \leftarrow X \times X$
        \STATE $N \leftarrow N / 2$
      \ELSE[$N$ is odd]
        \STATE $y \leftarrow y \times X$
        \STATE $N \leftarrow N - 1$
      \ENDIF
    \ENDWHILE
  \end{algorithmic}
\end{algorithm}
