\chapter{总结与展望}
\section{工作归纳}

\begin{figure}[!ht]
    \includegraphics[width=0.9\linewidth]{screenshot-20230130-193759.png}
\end{figure}

\section{分析技术评注}
\section{研究展望}
本文的研究可在多方面进行拓展。

在算法设计层面,由于实际处理的图的规模可能比较大,
我们可以考虑HPSP,梅特罗波利斯算法的并行化。
而在研究玻茨模型的相变性质层面,我们可以从分布
\eqref{eq:isingma}独立采集多个
样本,通过系综学习(如多数同意规则)的方式进行社群发现,
然后研究相变点$\beta$和样本数的关系。

此外,本文研究的是带有额外信息的两社群随机块模型,
该研究思路有望拓展到其他随机图模型或多社群随机块模型上,并
通过似然函数构建类似\eqref{eq:matrix_mle}式
的优化目标,从而指导
一般性的有额外信息的图模型的
算法设计。然后我们可以将该方法与通过其他准则\cite{chunaev2020community}构建的算法进行
比较探索其适用场景。

另一方面,随着深度学习技术的兴起,社群发现任务亦可由
神经网络模型藉以完成,尤其是在有额外信息的场景下 \cite{cao2018incorporating}。
仿照本文研究梅特罗波利斯算法、半正定方法的思路,
未来可研究神经网络模型在随机块模型上的恢复误差,即可解释性问题。