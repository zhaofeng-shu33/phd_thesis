% !TeX root = ../thuthesis-example.tex

\chapter{补充内容}

%附录是与论文内容密切相关、但编入正文又影响整篇论文编排的条理和逻辑性的资料,例如某些重要的数据表格、计算程序、统计表等,是论文主体的补充内容,可根据需要设置。


\section{定理\ref{thm:DPX}的证明}

\begin{lemma}\label{lem:xyz}
	若 $(X,Y)$ 与 $Z$ $\epsilon$-弱独立,
  则 $X$ 与 $Z$ $\epsilon$-弱独立。
\end{lemma}
\begin{proof}
	由两随机变量$\epsilon$-弱独立的定义式\ref{def:weak_indepedent},
  我们有
	\begin{equation}\label{eq:pxy_eps}
	P_{X,Y|Z=z}(x,y) = P_{X,Y}(x,y)\left(1+\epsilon \frac{\phi_z(x,y)}{\sqrt{P_{X,Y}(x,y)}}
  \right), z \in \mathcal{Z}
	\end{equation}
	对式\ref{eq:pxy_eps} 关于 $y\in \mathcal{Y}$ 求和我们有
	\begin{equation}
	P_{X|Z=z}(x) = P_X(x)\left(1+\epsilon\frac{\tilde{\phi}_z(x)}{\sqrt{P_X(x)}} \right),
	\textrm{ 其中 } \tilde{\phi}_z(x) = \frac{\sum_{y\in \mathcal{Y}} \sqrt{P_{X,Y}(x,y) }\phi_z(x,y)}{\sqrt{P_X(x)}}
	\end{equation}
	由 柯西不等式, $||\tilde{\phi}_z(x)||^2 \leq \frac{1}{P_X(x)}
	\sum_{y\in \mathcal{Y}}(P_{X,Y}(x,y))
	\sum_{y\in \mathcal{Y}} \phi_z^2(x,y) \leq 1
	$
	从而推出 $X$  与 $Z$ 弱独立。
\end{proof}
\begin{proof}[定理\ref{thm:DPX}的证明]
  由多个随机变量弱独立的定义式 \ref{def:general},
  我们可以找到 一个在字母集 $\{1, 2,\dots, K\}$上的
  离散分布 $U$, 使得 $Z_1, \dots, Z_n$
  关于 $U$ 条件独立。不失一般性,我们假设
  $(Z_1, \dots, Z_n)$ $\epsilon^n$-弱独立。
  则 $(Z_1, \dots, Z_n)$ 和 $U$ $\epsilon$-弱独立,
  由引理 \ref{lem:xyz} 可得 $Z_i$
  和 $U$ $\epsilon$-弱独立,也即有 
\begin{equation}\label{XUk}
P_{Z_i | U=k}(z) = P_{Z_i} (z) \left( 1 + \epsilon {\phi^{(k,i)}(z) \over \sqrt{P_{Z_i}(z)}} \right)
\end{equation}
利用条件独立有
\begin{align}
P_{Z_i, Z_j | U = k}(z_i, z_j)
=& P_{Z_i | U=k}(z_i)
P_{Z_j | U=k}(z_j) \notag \\
=& P_{Z_i}(z_i)P_{Z_j}(z_j)
\left(1 + \epsilon
\left(\frac{\phi^{(k,i)}(z_i)}{\sqrt{P_{Z_i}(z_i)}}
+ \frac{\phi^{(k,j)}(z_j)}{\sqrt{P_{Z_j}(z_j)}}
\right) +
\epsilon^2\frac{\phi^{(k,i)}(z_i)
	\phi^{(k,j)}(z_j)}{\sqrt{P_{Z_i}(z_i)P_{Z_j}(z_j)}}
  \right)
  \label{eq:XiXj}
\end{align}
又因为
\begin{align*}
P_{Z_i}(z) &= \sum_{k=1}^{K} P_{Z_i | U=k}(z) P_U(u_k) \\
& =  \sum_{k=1}^{K}P_U(u_k)P_{Z_i} (z)
\left( 1 + \epsilon {\phi^{(k,i)}(z) \over \sqrt{P_{Z_i}(z)}} 
\right) \textrm{ from } \eqref{XUk}\\
\Rightarrow & \sum_{k=1}^{K} P_U(u_k){\phi^{(k, i)}(z) \over \sqrt{P_{Z_i}(z)}} =0,\forall i, z\in \mathcal{Z}
\end{align*}
因此 \eqref{eq:XiXj} 化简为
\begin{equation}\label{eq:PXiXj}
P_{Z_i, Z_j}(z_i, z_j) = P_{Z_i}(z_i)
P_{Z_j}(z_j) \left(
  1+\epsilon^2 \sum_{k=1}^K P_U(u_k)
\frac{\phi^{(k,i)}(z_i)
	\phi^{(k,j)}(z_j)}{\sqrt{P_{Z_i}(z_i)P_{Z_j}(z_j)}}
  \right)
\end{equation}
对于2个以上的随机变量:
\begin{align*}
P_{Z_1,\dots,Z_n}(z_1,\dots,z_n)  &= \sum_{k=1}^{K} P_{Z_1,\dots,Z_n | U=k}(z_1,\dots,z_n) P_U(u_k) \\
&=  \sum_{k=1}^{K}P_U(u_k) \prod_{i=1}^n P_{Z_i|U=k}(z_i)\\
&= \sum_{k=1}^{K} P_U(u_k)\prod_{i=1}^n \left(P_{Z_i} (z_i)( 1 + \epsilon {\phi^{(k,i)}(z_i ) \over \sqrt{P_{Z_i}(z_i)}} )\right)\\
&=  \sum_{k=1}^{K}P_U(u_k) (\prod_{i=1}^n  P_{Z_i} (z_i))
\left( 1 + \epsilon\sum_{i=1}^n {\phi^{(k,i)}(z_i) \over \sqrt{P_{Z_i}(z_i)}} + \epsilon^2\sum_{i\neq j}{\phi^{(k,i)}(z_i)\phi^{(k,j)}(z_j)\over \sqrt{P_{Z_i}(z_i)P_{Z_j}(z_j)} }\right)+o(\epsilon^2) \\
&= (\prod_{i=1}^n  P_{Z_i} (z_i))
\Big(1+\epsilon\sum_{i=1}^n \sum_{k=1}^{K} P_U(u_k){\phi^{(k,i)}(z_i) \over \sqrt{P_{Z_i}(z_i)}} \\
&+\epsilon^2 \sum_{k=1}^{K} P_U(u_k)\sum_{i\neq j}{\phi^{(k,i)}(z_i)\phi^{(k,j)}(z_j)\over \sqrt{P_{Z_i}(z_i)P_{Z_j}(z_j)} } 
\Big) + o(\epsilon^2)\\
&= (\prod_{i=1}^n  P_{Z_i} (z_i))
\left(1 +\epsilon^2\sum_{i\neq j} \sum_{k=1}^{K}P_U(u_k){\phi^{(k,i)}(z_i)\phi^{(k,j)}(z_j)\over \sqrt{P_{Z_i}(z_i)P_{Z_j}(z_j)} }\right) + o(\epsilon^2)
\end{align*}
由 \eqref{eq:PXiXj},
令 $B_{ij}(z_i, z_j)={P_{Z_i, Z_j}(z_i,z_j) - P_{Z_i}(z_i)P_{Z_j}(z_j) \over \sqrt{P_{Z_i}(z_i)P_{Z_j}(z_j)}} $ 可得:
\begin{align}
\epsilon^2\sum_{k=1}^{K}P_U(u_k)
{\phi^{k,i}(z_i)\phi^{k,j}(z_j)\over \sqrt{P_{Z_i}(z_i)P_{Z_j}(z_j)} } & = {P_{Z_i, Z_j}(z_i, z_j) - P_{Z_i}(z_i)P_{Z_j}(z_j) \over P_{Z_i}(z_i)P_{Z_j}(z_j)} \notag\\
& = {B_{ij}(z_i, z_j) \over \sqrt{P_{Z_i}(z_i)P_{Z_j}(z_j)} } \label{eq:Bsecond}
\end{align}
因此,我们有
\begin{equation}\label{eq:sep}
P_{Z_1,\dots,Z_n}(z_1,\dots,z_n) =  (\prod_{i=1}^n  P_{Z_i} (z_i))\left ( 1 + \sum_{i\neq j}{B_{ij}(z_i, z_j) \over \sqrt{P_{Z_i}(z_i)P_{Z_j}(z_j)} }\right) +o(\epsilon^2)
\end{equation}
因此 $P_{Z_1,\dots, Z_n}$ 在 $P_{Z_1}\dots P_{Z_n}$ 的
$\epsilon$ 邻域内,且特征函数是
$$\phi(z_1,\dots, z_n)=
\sqrt{P_{Z_1}(z_1)\dots P_{Z_n}(z_n)}
\left(\sum_{i\neq j}{B_{ij}(z_i, z_j) 
\over \sqrt{P_{Z_i}(z_i)P_{Z_j}(z_j)} }\right)
+o(\epsilon^2)$$

由 信息几何的式\eqref{eq:approx:ig} 可得
\begin{align*}
D(P_{Z_1,\dots, Z_n}|| P_{Z_1}\dots P_{Z_n}) & ={1 \over 2} \sum_{z_1,\dots,z_n}\phi^2(z_1,\dots, z_n) \\
& = {1\over 2}\sum_{z_1,\dots,z_n} (\prod_{i=1}^n  P_{Z_i} (z_i)) \left(\sum_{i\neq j}{B_{ij}(z_i, z_j) \over \sqrt{P_{Z_i}(z_i)P_{Z_j}(z_j)} }\right)^2 +o(\epsilon^2) 
\end{align*}
由 $B_{ij}$
的定义式 \eqref{eq:Ixy},
上式可化为对 $\norm{B_{ij}}^2_F$
的求和(平方和中的交叉项外面再求和得零)。
因此,对于分割$\P=\{\{i\},i\in V\}$ 我们得到
\begin{equation}
D(P_{Z_1,\dots, Z_n}|| P_{Z_1}\dots P_{Z_n}) =   {1 \over 2} \sum_{i\neq j} \norm{B_{ij}}^2_F + o(\epsilon^2)
\end{equation}
对于任意的分割 $\P$,
由式\eqref{eq:sep}可得,对于 $C\in \P$,
我们有
\begin{equation}
P_{Z_C}(z_C) = \prod_{i\in C} P_{Z_i}(z_i)
\left(1 + \epsilon^2 \sum_{i\neq j,i,j\in C} \frac{B_{ij}(z_i, z_j)}{\sqrt{P_{Z_i}(z_i)P_{Z_j}(z_j)}}
\right) + o(\epsilon^2)
\end{equation}
将上式相乘可得:
\begin{equation}
\prod_{C\in \P}P_{Z_C}(z_C) = \prod_{i=1}^n P_{Z_i}(z_i)
\left(1+\epsilon^2 \sum_{C\in\P}\sum_{i\neq j,i,j\in C}\frac{B_{ij}(z_i, z_j)}{\sqrt{P_{Z_i}(z_i)P_{Z_j}(z_j)}}
\right) + o(\epsilon^2)
\end{equation}
所以 $\prod_{C\in \P}P_{Z_C}$ 在 $P_{Z_1}\dots P_{Z_n}$ 的$\epsilon$ 邻域内,
且 $$\phi_{\P}(z_1,\dots, z_n)=
\sqrt{P_{Z_1}(z_1)\dots P_{Z_n}(z_n)}\left(\sum_{C\in\P}\sum_{i\neq j,i,j\in C}\frac{B_{ij}(z_i, z_j)}{\sqrt{P_{Z_i}(z_i)P_{Z_j}(z_j)}}\right)+o(\epsilon^2)$$
由  \eqref{eq:approx:ig} 得:
\begin{align*}
D(P_{Z_1,\dots, Z_n}|| \prod_{C\in \P}P_{Z_C}) & ={1 \over 2} \sum_{z_1,\dots,z_n}(\phi(z_1,\dots, z_n)-\phi_{\P}(z_1, \dots, z_n))^2 \\
& = {1\over 2}\sum_{z_1,\dots,z_n} \prod_{i=1}^n  P_{Z_i} (z_i) \left(\sum_{\substack{(i,j) \not\in C\\ C\in \P}} {B_{ij}(z_i, z_j) \over \sqrt{P_{Z_i}(z_i)P_{Z_j}(z_j)} }\right)^2 +o(\epsilon^2) \\
& = \frac{1}{2} \sum_{\substack{(i,j) \not\in C\\ C\in \P}} \norm{B_{ij}}_F^2 + o(\epsilon^2)
\end{align*}
\end{proof}



\section{定理\ref{thm:triangle} 的证明}
首先我们证明如下引理:
\begin{lemma}\label{thm:trival}
  分割 $\P_k = \{\{1\},\{2\},\dots,\{\abs{V}\}\}$ 
  使得 $\frac{f[\P]}{\abs{\P}-1}$
  最小 当且仅当 条件\eqref{eq:GF} 成立。
  \begin{equation}\label{eq:GF}
  \frac{f[\P]}{\abs{\P}-1} \geq \frac{f[\P_k]}{\abs{V}-1} \textrm{ for any } \P \in \Pi'
  \end{equation}
  \end{lemma}
  \begin{proof}
  由\citet{narayanan} 的定理3可知,
  $h(\lambda)$ 是分段线性函数,
  $h(\lambda)$ 的第一段
  是 $ - \lambda $ 且 
  最后一段是 $ f[\P_k] - \abs{V} \lambda$。
  它们的交点是
  $(\lambda_{+}, -\lambda_{+})$,
  其中
  $\lambda_{+} = \frac{f[\P_k]}{\abs{V}-1}$。
  因为 $\frac{f[\P]}{\abs{\P}-1} \geq \lambda_{+} \iff f[\P] - \abs{\P}\lambda_{+} \geq - \lambda_{+}$,
  方程 \eqref{eq:GF}
  相当于说 $h(\lambda)$ 在 $\lambda = \lambda_{+}$
  的最小值点是
  $\{V\}$ 或
  $\{\{1\},\{2\},\dots,\{\abs{V}\}\}$
  并且 $h(\lambda)$ 只有两段。
  从几何的观点来看,$\min\frac{f[\P]}{\abs{\P}-1}$
  是 $h(\lambda)$ 第一个分界点,
  该分界点与$h(\lambda)$最后一个分界点
  相等 当且仅当 \eqref{eq:GF} 成立。
  \end{proof}
  \begin{proof}[定理\ref{thm:triangle} 的证明]
    我们将使用归纳法证明对于任意的分割$\P \in \Pi$,
    式\eqref{eq:GF} 成立。
    \begin{equation}\label{eq:GF}
    f[\P] \geq \frac{\abs{\P}-1}{\abs{V}-1} \sum_{(i,j) \in E} w_{ij}
    \end{equation}
    
    然后由引理 \ref{thm:trival}
    即可得证。
    
    设 $n=\abs{V}$。
    若 $\abs{\P}=n$,
    易得式 \eqref{eq:GF} 取等号。
    
    假设 式\eqref{eq:GF} 对 任意
    $\abs{\P} \geq k+1(k\geq 2)$成立,
    下面我们讨论 $\P=\{C_1, \dots, C_k\}$
    的情形。
    因为$\P$是非平凡的分割,
    因此存在某个$\abs{C_i}\geq 2$。
    不妨设$\abs{C_1}=n_1\geq 2$,
    构造$\P$的一个细分$\P'$如下:
    $\P_{C_1} = \{\{i\}| i \in C_1\}, \P'=\P_{C_1} \cup \P \backslash \{C_1\}$。
    则我们有 $\abs{\P'} = k+n_1-1$。
    对$\P'$运用 式\eqref{eq:GF} 有
    \begin{align}\label{eq:PPrelation}
    f[\P'] &
    \geq \frac{k+n_1 -2}{n-1}\sum_{(i,j) \in E} w_{ij} \\
    f[\P] & =f[P'] - \sum_{(i,j) \in E(C_1)} w_{ij}
    \end{align}
    其中
    $E(C_1) =\{ (i,j) |i<j, i, j\in C_1 \}$。
    对任意$k \not\in C_1$运用三角不等式
    $w_{ij} \leq w_{ik} + w_{jk}$,
    并关于所有 $i, j \in C_1, i\neq j$求和,
    我们有
    $$
    \sum_{(i,j) \in E(C_1)} w_{ij} \leq \sum_{(i,j) \in E(C_1)} (w_{ik} + w_{jk}) = (n_1-1)\sum_{i\in C_1} w_{ik}
    $$
    再将上式对所有 $k \not\in C_1$ 求和
    $$
    (n - n_1) \sum_{(i,j) \in E(C_1)} w_{ij} \leq (n_1 - 1) \sum_{i \in C_1, k \not\in C_1} w_{ik}
    $$
    另外,注意到
    \begin{align*}
    \sum_{(i,j) \in E} w_{ij}  & \geq \sum_{(i,j) \in E(C_1)} w_{ij} + \sum_{i\in C_1, k\not\in C_1} w_{ik} \\
    (n_1 - 1)\sum_{(i,j) \in E} w_{ij}  & \geq (n_1 -1 )\sum_{(i,j) \in E(C_1)} w_{ij} + (n_1-1)\sum_{i\in C_1, k\not\in C_1} w_{ik} \\
    & \geq (n_1 -1 )\sum_{(i,j) \in E(C_1)} w_{ij} + (n - n_1) \sum_{(i,j) \in E(C_1)} w_{ij}\\
    & = (n-1) \sum_{(i,j) \in E(C_1)} w_{ij}
    \end{align*}
   因此我们得到 $\sum_{(i,j) \in E(C_1)} w_{ij} \leq \frac{n_1-1}{n-1}\sum_{(i,j) \in E} w_{ij}$。
   从式 \eqref{eq:PPrelation} 可得
    $f[\P] \geq \frac{k-1}{n-1}\sum_{(i,j) \in E} w_{ij}$。
    因此,结论对 $\abs{\P}=k$ 成立。
    \end{proof}
\section{定理 \ref{thm:alg_complexity} 的证明}
\begin{proof}[定理 \ref{thm:alg_complexity} 的证明]
  令
  $\mu_i = \frac{n_i}{n} \leq \frac{1}{2}$。
  下面我们使用归纳法证明
  $T(n) = O(n^4)$。
  更具体地,我们将证明 
  $T(n) \leq q C n^4$, 其中 $ q = \frac{16}{5}$。
  $T(3)$ 是一个常数,我们取$C$使得 $T(3)= q C\times 3^4$。
  下面假设
	$T(m) \leq qC m^4$
  对所有 $m \leq n-1$。
  然后 对于 $T(n)$,
	我们首先证明下式成立:
	\begin{equation}\label{eq:outerI}
	\sum_{i=1}^k T(n_i) \leq 10 T(\frac{n}{2})
	\end{equation}
	因为 $\sum_{i=1}^k T(n_i) \leq qC n^4\sum_{i=1}^k u_i^4$ 并且
  从式 \eqref{eq:Tn} 中可得
  $10 T(\frac{n}{2}) \geq 10Cn^4 (\frac{1}{2})^4$ 
	\begin{equation}\label{eq:innerI}
       q\sum_{i=1}^k u_i^4 \leq 10 (\frac{1}{2})^4 
	\end{equation}
	所以我们有 \eqref{eq:innerI} $\Rightarrow$ \eqref{eq:outerI}。
  由定理 \ref{thm:alg_complexity} 条件,我们不妨假设
  $u_1\leq u_2 \leq \dots \leq u_k \leq \frac{1}{2}$
  且 $\sum_{i=1}^k u_i = 1$。
  因此,我们有
  $u_1 \leq \frac{1}{k}, u_2 \leq \frac{1}{k-1}, \dots, u_{k-1} \leq \frac{1}{2}$。
	根据
  \begin{equation}\label{eq:outerOne}
	 q[2(\frac{1}{2})^4 + \sum_{i=3}^k (\frac{1}{i})^4] \leq 10 (\frac{1}{2})^4
	\end{equation}
	我们有 \eqref{eq:outerOne} $\Rightarrow$ \eqref{eq:innerI}。
  并且代入$q=\frac{16}{5}$,
  \eqref{eq:outerOne} 等价于
	$\sum_{i=3}^k (\frac{1}{i})^4 \leq \frac{9}{8}(\frac{1}{2})^4$
	\begin{align*}
		\sum_{i=3}^k (\frac{1}{i})^4 & < \frac{1}{9}\sum_{i=3}^k \left(\frac{1}{i}
    \right)^2 \\
		& < \frac{1}{9}\sum_{i=3}^k \frac{1}{i(i-1)} \\
		& < \frac{1}{18}
    < \frac{9}{8}\left(\frac{1}{2}
    \right)^4
	\end{align*}
	因此,
  式 \eqref{eq:outerI} 成立。然后
  由 式\eqref{eq:Tn} 和 $T(k) \leq T(\frac{n}{2})$ 可得
	\begin{align}
		T(n)  & \leq Cn^4 + 11T\left(\frac{n}{2} \right) \\
		& \leq C n^4 + 11 q C \left(\frac{n}{2}\right)^4
    = \frac{16}{5} C n^4
	\end{align}
\end{proof}
\section{定理\ref{thm:main_N}的证明}
    \begin{proof}[定理\ref{thm:main_N}的证明]
      我们首先做出如下的符号约定:
      $w(C) = \displaystyle\sum_{(i,j) \in E(C)} w_{ij},$
      $w(A, C) = \displaystyle\sum_{\substack{i \in A, j \in C \\ (i,j) \in E(V')}} (w_{ij}+w_{ji})$ and
      $\P_C = \{\{i \}| i \in C \}$。
      假设 $\gamma_N = I(Z_B)$,
      由式 \eqref{eq:largest_threshold} 可得
      $\gamma_N =\frac{f[\P_B]} {\abs{B} -1}$。
      因为 $V' = B \cup \{i'\}$,我们有
      $f[\P_{V'}] = f[\P_B] + \sum_{i \in B}w_{ii'}$。
      
      若 $ \sum_{i \in B} w_{ij} > \gamma_N$, 我们可得
      $$
      \gamma'_N \geq I_{\P_{V'}}(Z_{V'}) = \frac{(\abs{B}-1)\gamma_N
      + \sum_{i \in B}w_{ii'}}{\abs{B}} > \gamma_N
      $$
      
      另一方面, 假设 
      $\gamma'_N = I(Z_K) > \gamma_N$。 则 $K$ 必包含 $i'$。
      若 $K=V'$, 则结论成立。
      否则, 假设 $K = \{i'\} \cup B', B=B'\cup J, J\neq \emptyset$,
      则我们有
      \begin{equation}\label{eq:gammaNF}
      \gamma_N = \frac{w(J,B') + w(J) + w(B')}{ \abs{B'} + \abs{J} - 1 }
      \end{equation}
      因为 $I(Z_K) = I_{\P_K}(Z_K)$ 是最大的,
      我们有 $I_{\P_K}(Z_K) > I_{\P_{V'}}(Z_{V'})$ 且
      $I_{\P_K}(Z_K) \geq I_{\P_{B'}}(Z_{B'})$。
      \begin{align*}
      \frac{w(\{i'\}, B') + w(B')}{\abs{B'}} >& \frac{w(B') + w(J, B') + w(J) + w(\{i'\}, B') + w(\{i'\}, J)}{\abs{B'} + \abs{J}}  \\
      \frac{w(\{i'\}, B') + w(B')}{\abs{B'}} \geq & \frac{w(B')}{\abs{B'} - 1}
      \end{align*}
      于是可得
      \begin{align}
      \abs{J} \left(w(\{i'\}, B') + w(B') \right) > &
      \abs{B'} \left(w(J, B') + w(J) + w(\{i'\}, J) \right)
      \label{eq:target}
      \\
      (\abs{B'} - 1)  w(\{i'\}, B') \geq & w(B') \label{eq:convert}
      \end{align}
      将 式\eqref{eq:convert} 代入 式\eqref{eq:target} 可得
      \begin{equation}\label{eq:summation}
      \abs{J} w(\{i'\}, B') > w(J, B') + w(J) + w(\{i'\}, J)
      \end{equation}
      结合 式\eqref{eq:gammaNF},
      将 式\eqref{eq:convert} 与 式\eqref{eq:summation} 相加我们可得
      $w(\{i'\}, B') > \gamma_N$。
      所以 $\sum_{j \in B}w_{ii'} > \gamma_N $ 成立。
    \end{proof}
  
\section{数据集}

\subsection{数据聚类数据集}
Gaussian 数据集是用\texttt{scikit-learn}
\cite{scikit-learn}
的 make\_blobs 函数生成,每类25个数据点,共4类。
其中心分别为$[3,3],[-3,-3],[3,-3],[-3,3]$,
标准差均为1。

Circle 数据集共3类,每一类的数据点个数分别为 $[60,100,140]$,
并使用如下的极坐标公式生成:
\begin{align*}
r &= 0.1 \times i + 0.01(2u-1), i = 1, 2, 3\\
\theta & = 2\pi v
\end{align*}
其中 $u,v$ 是 $[0,1]$ 间的随机数,彼此独立。

\subsection{PSP算法性能比较所用数据集}
Gaussian-blob 数据集与 Gaussian 数据集
的区别在于每类$n$个数据点,其中$n=[1,2,3,4,5]*100$。

Two-level graph 数据集中取参数
$z_{\mathrm{in}_1}=s^2-2,
z_{\mathrm{in}_2}=s-1,
z_{\mathrm{out}}=1$,其中$s=[3,4,5,6,7]$。
\subsection{异常值检测所用数据集}
GaussianBlob 数据集 共有 300 个点,
其中 有 15\% 的点采样自 $[-6,6]\times [-6,6]$
内的均匀分布,为异常值点。其余的点采样自标准差为0.5,
中心在原点的 二维高斯分布, 为正常值点。

Moon 数据集类似 GaussianBlob 数据集, 共有 345 个点,
其中 有45 个点采样自 $[-6,6]\times [-6,6]$,经过个别点的坐标修正
后视为异常值点,
其余的点用 scikit-learn 的 make\_moons 函数生成,并做适当的
伸缩变换后视为正常值点。

Lymphography 数据集共148个数据点,
分成4类,因为第1和第4类总共有6个数据点,被视为异常值点,
其余的点为正常值点。

Glass 数据集在数据聚类中用到过,这里因为其第6个类别的点数量最少,
视其为整体的异常值点,而其他数据点为正常值点。

Ionosphere 数据集经常预处理后共 351个数据点,特征的维度为32,
其中 126个为异常值点。

\section{算法超参数}
\subsection{聚类算法}
在表\ref{tab:clustering_dataset}对应的实验中,
除信息聚类算法外,我们使用 scikit-learn 算法包提供
的接口,各算法的超参数如表\ref{tab:clustering_alg_hyperparameter}
所示。
系统聚类法(AgglomerativeClustering)统一使用欧式距离,
谱聚类(SpectralClustering)
统一使用 k近邻(8)计算相似度矩阵,括号中的数字表示
邻居节点的数量。
近邻传播和信息聚类的相似度矩阵因数据集而异。
这里,我们允许信息聚类取rbf和k近邻
两种相似度度量的复合,即它们数值的相乘。
表中 rbf 度量和 laplacian 度量
后面括号中的参数 $\gamma$的含义与
式 \ref{eq:rbf_kernel} 相同。
针对系统聚类和信息聚类两个层次聚类算法,
为获得非层次聚类结果,需要指定相应的超参
从聚类树中抽取局部的聚类结果,这里为方便
起见,我们给出抽取后的类别数作为派生的超参。

\begin{table}[!ht]
  \begin{adjustbox}{width=\columnwidth,center}
  \begin{tabular}{|l|r|r|r|r|r|r|}
  \hline
   聚类算法  &   参数名称 & Gaussian &   Circle &   Iris &   Glass &   Libras \\
  \hline
   \multirow{2}{*}{系统聚类法}  &  linkage     &   complete&     single &   average
   & ward &   complete \\
   \cline{2-7}
   & 类别数 & 4 & 3 & 3 & 7 & 15 \\
   \hline
   \multirow{3}{*}{近邻传播}  &    affinity & euclidean &  k近邻(16)
   &  k近邻(10)
   & euclidean &   euclidean \\
   \cline{2-7}
   & preference & -50 & 50 & 50 &
   -50 & -50\\
   \cline{2-7}
   & 衰减系数 & 0.5 & 0.8 & 0.5 & 0.8 & 0.5 \\
   \hline
   \multirow{2}{*}{信息聚类}  & 
   \multirow{2}{*}{affinity}      &       rbf(0.4) &    rbf(0.01)&  laplacian(8) & k近邻(14) &   rbf(0.2) \\
   & & &  k近邻(16)  & && k近邻(4)\\
   \cline{2-7}
   & 类别数 & 4 & 3 & 3 & 42 & 38\\
   \hline
   k-平均算法     &     类别数    &
   4 &    3 &  3 &   6 &    15 \\
   \hline
   谱聚类   &    类别数 & 4 &   3 &    3 &   6 &    15 \\
  \hline
  \end{tabular}
  \end{adjustbox}
\caption{聚类算法超参数}\label{tab:clustering_alg_hyperparameter}
\end{table}

另外,在图\ref{fig:shc}对应的实验中。系统聚类法我们使用 scipy 算法包中的 linkage 来完成。
对于贝叶斯玫瑰树算法,
%参考了xxx的开源实现,
混合比例取 $\alpha=0.5$且先验分布是
正态逆威沙特 (Normal-inverse-Wishart)分布。
%(a mixing proportion)

\subsection{社群的层次发现算法}
针对社群的层次发现场景,信息聚类和 GN 算法均是不含
超参的。
BHCD 算法的超参数较多,
我们取 $\delta=0.2, \lambda=0.85$,其他参数取自文章\citet{RN23}
中的默认值。
\subsection{异常值检测算法}
除信息检测算法外,我们使用 scikit-learn 算法包提供
的接口,除接口默认的参数外,各算法的超参数如表\ref{tab:outlier_detection_alg_hyperparameter}
所示。信息检测算法根据式\ref{eq:outlier_detection_proposal}介绍的规则自动判断异常值点的数量,
其他算法则根据已知的异常值点的数量对异常值进行检测。

\begin{table}[!ht]
  \begin{adjustbox}{width=\columnwidth,center}
  \begin{tabular}{|l|c|c|c|c|c|c|}
  \hline
   异常值检测算法  &   参数名称 &
   GaussianBlob &  Moon &  Lymphography &   Glass &   Ionosphere \\
  \hline
   \multirow{2}{*}{信息聚类}  & affinity
   & rbf & rbf & laplacian
   & rbf & rbf \\
   \cline{2-7}
   & $\gamma$ & 0.5 & 0.8 & 0.1 & 0.05 & 0.29 \\
   \hline
   局部异常因子  &  
   邻居数量 & 20 &  10
   &  60
   & 2 &   9 \\
   \hline
   \multirow{2}{*}{一类支持向量机}  & 
   affinity & rbf &  rbf
   & rbf
   & rbf & rbf \\
   \cline{2-7}
   & $\gamma$ & 0.1 & 0.1 & 0.1 & 0.1 & 0.4\\
  \hline
  \end{tabular}
\end{adjustbox}
\caption{异常值检测算法超参数}\label{tab:outlier_detection_alg_hyperparameter}
\end{table}


\section{定理\ref{thm:phase_transition}的证明}
\label{sec:appendix_theorem_proof_phase_trans}

\begin{proof}[引理 \ref{lem:lemmaDiff}的证明]
	首先我们将
  \eqref{eq:energy} 式改写成
	\begin{equation*}
	H(\bar{\sigma}) = \gamma \frac{\log n}{n} \sum_{i < j} \delta(\bar{\sigma}_i, \bar{\sigma}_j)
	- (1 + \gamma\frac{\log n}{n}) \sum_{ \{i, j\} \in E(G)} \delta(\bar{\sigma}_i, \bar{\sigma}_j)
	\end{equation*}
	
	接着我们逐步计算能量差:
  \begin{align*}
	H(\bar{\sigma}') - H(\bar{\sigma}) =& 
  \left(1 + \gamma\frac{\log n}{n}
  \right)
  \cdot \sum_{i \in N_r(G)} (\delta(\bar{\sigma}_r, \bar{\sigma}_i) -
	\delta(\omega^s \cdot \bar{\sigma}_r, \bar{\sigma}_i)) \\
	&+ \gamma \frac{\log n}{n}\sum_{i\neq r}
	( \delta(\omega^s \cdot \bar{\sigma}_r, \bar{\sigma}_i) -
	\delta( \bar{\sigma}_r, \bar{\sigma}_i) ) \\
	 =& \left(
     1 + \gamma\frac{\log n}{n}
     \right)
   \sum_{i \in N_r(G)} J_s(\bar{\sigma}_r, \bar{\sigma}_i) \\
	&+ \gamma \frac{\log n}{n}\sum_{i=1}^n
	( \delta(\omega^s \cdot \bar{\sigma}_r, \bar{\sigma}_i) -
	\delta( \bar{\sigma}_r, \bar{\sigma}_i) +1) \\
	=& \left(1+\gamma \frac{\log n}{n}
  \right)
  \sum_{i \in N_r(G)} J_s(\bar{\sigma}_r, \bar{\sigma}_i)\\
	&+ \gamma \frac{\log n}{n} (m(\omega^s \cdot \bar{\sigma}_r)-m(\bar{\sigma}_r)+1)
	\end{align*}
\end{proof}

在开始定理\ref{thm:phase_transition}的技术证明之前,我们需要介绍一些额外的
符号。
当 $\bar{\sigma}'$ 与 $X$ 仅在位置 $r$
上不同时, 在引理 \ref{lem:lemmaDiff}
中取$\bar{\sigma}=X$,
结合条件  $|\{v \in [n] : X_v = u\}| = \frac{n}{k}$,
有 $m(\omega^s \cdot \bar{\sigma}_r)
=m(\bar{\sigma}_r)$,
进而:
\begin{equation}\label{eq:energy_diff}
H(\bar{\sigma}') - H(X) = (1+\gamma \frac{\log n}{n})(A^0_r - A^s_r) + \gamma\frac{\log n}{n}
\end{equation}
其中 $A^s_r$ 定义成
$A^s_r := |\{j \in [n]\backslash \{r\}: \{j, r\} \in E(G), X_j = \omega^s \cdot X_r \}|$。
因为$G$中各边的存在与否是独立的,
对于 $s\neq 0$
我们有 $A^s_r \sim \textrm{Binomial}\left(
  \frac{n}{k}, \B \right) $,
而 $A^0_r \sim \textrm{Binomial}
\left(
  \frac{n}{k}-1, \A \right)$。

% Some additional notation is needed for the proof.
对于一般的情形,我们可以将能量差写成:
\begin{equation}\label{eq:Hgeneral}
H(\bar{\sigma}) - H(X)=
(1 + \gamma \frac{ \log n}{n})[A_{\bar{\sigma}} - B_{\bar{\sigma}}] + \gamma\frac{ \log n}{n} N_{\bar{\sigma}}
\end{equation}
在上式中 我们使用
$A_{\bar{\sigma}}$ 或
$B_{\bar{\sigma}}$ 来表示
服从参数分别为 $\A$ 或 $\B$ 的二项分布的随机变量,并且
$N_{\bar{\sigma}}$ 是一个依赖于 $\bar{\sigma}$ 但与图结构无关的非随机的正整数。
引理\ref{lem:minus} 给出了 $A_{\bar{\sigma}}, B_{\bar{\sigma}}$ 和 $N_{\bar{\sigma}}$
的表达式:

\begin{lemma}\label{lem:minus}
	对于 $\SSBM(n,k,p,q)$,
  我们假设 $\bar{\sigma}$ 
  在总共$|\cI|:=\Dist(\bar{\sigma}, X)$ 
  个坐标位置上不同于真值标签向量
  $X$。
	对于 $i\neq j$,令 $I_{ij} = \Big|\{r\in [n] \big| X_r = w^i, \sigma_r = w^j \}\Big|$
  且 $I_{ii} = 0$。
  我们进一步将行和$I_i$
  和列和$I'_i$分别表示为:
  \begin{align}
  I_i &= \frac{n}{k} -
  \Big|\{r\in [n] \big| X_r = w^i, \sigma_r=w^i \}\Big|
  =
  \sum_{j=0}^{k-1} I_{ij} \label{eq:I_i_horizontal} \\
  I'_i &=
  \Big|\{r\in [n] \big| X_r \neq w^i,
  \sigma_r=w^i \}\Big|
  =\sum_{j=0}^{k-1} I_{ji}\label{eq:I_prime_i_vertical}
  \end{align}
	则
\begin{align}
	N_{\bar{\sigma}} &= \frac{1}{2}\sum_{i=0}^{k-1} (I_i - I_i')^2 \label{eq:N_w} \\
	B_{\bar{\sigma}} & \sim \textrm{Binomial}(N_{B_{\bar{\sigma}}} , q),
  N_{B_{\bar{\sigma}}}=\frac{n}{k}|\cI| + \frac{1}{2}\sum_{i=0}^{k-1}  (-2 I'_i I_i  + I'^2_i - \sum_{j=0}^{k-1} I^2_{ji})
  \label{eq:B_w}\\
	A_{\bar{\sigma}} &
  \sim \textrm{Binomial}(N_{A_{\bar{\sigma}}}, p),
  N_{A_{\bar{\sigma}}}=
  \frac{n}{k}|\cI| - \frac{1}{2}\sum_{i=0}^{k-1}
   (I^2_i + \sum_{j=0}^{k-1} I^2_{ij})
  \label{eq:A_w}
	\end{align}
\end{lemma}
\begin{proof}
  由式\eqref{eq:isingma}、\eqref{eq:energy}可得
	\begin{align}
  \log P_{\sigma|G}(\sigma=\bar{\sigma})
  & = 
  \beta(1 + \frac{\gamma \log n}{n}) \sum_{\{i,j\} \in E(G)}  \delta(\bar{\sigma}_i, \bar{\sigma}_j)  
	- \beta\frac{\gamma \log n}{n} \sum_{1\leq i<j \leq n} \delta(\bar{\sigma}_i, \bar{\sigma}_j)  - \log Z_G(\gamma, \beta)
  \notag\\
  & = 
  \beta(1 + \frac{\gamma \log n}{n}) \sum_{\bar{\sigma}_i = \bar{\sigma}_j} Z_{ij}
	- \beta\frac{\gamma \log n}{n}
  \sum_{\bar{\sigma}_i = \bar{\sigma}_j} 1 
  - \log Z_G(\gamma, \beta)
  \label{eq:log_P_sigma_G_bar_sigma}
  \end{align}
	其中 $Z_{ij}$ 服从伯努利分布,
  $P(Z_{ij}=1)$ 等于节点$i$与节点$j$之间有边相边的概率。
	对于 $\sigma = X$, 我们有
	\begin{align}
	\sum_{X_i = X_j} Z_{ij} &
  = \sum_{X_i = X_j, \bar{\sigma}_i = \bar{\sigma}_j}
  Z_{ij} + \sum_{X_i = X_j, \bar{\sigma}_i \neq \bar{\sigma}_j}
  Z_{ij}
  \label{eq:split_zij_tmp_X_ij}\\
	\sum_{X_i = X_j} 1 &= \frac{1}{2} \sum_{i=0}^{k-1} \frac{n}{k} 
  \left( \frac{n}{k} - 1 
  \right)\label{eq:split_one_tmp_X_ij}
	\end{align}
	对于 $\sigma = \bar{\sigma}$,我们有
	\begin{align}
	\sum_{\bar{\sigma}_i = \bar{\sigma}_j} Z_{ij} &= \sum_{X_i = X_j, \bar{\sigma}_i = \bar{\sigma}_j} Z_{ij} + \sum_{X_i \neq X_j, \bar{\sigma}_i = \bar{\sigma}_j} Z_{ij} 
  \label{eq:split_zij_tmp_sigma_ij}\\
	\sum_{\bar{\sigma}_i = \bar{\sigma}_j} 1 &= \frac{1}{2} \sum_{i=0}^{k-1} (I'_i + \frac{n}{k} - I_i) ( I'_i + \frac{n}{k} - I_i - 1)
  \label{eq:split_one_tmp_sigma_ij}
	\end{align}
  其中,式 \eqref{eq:split_one_tmp_sigma_ij} 是根据式
  \eqref{eq:I_i_horizontal}和 \eqref{eq:I_prime_i_vertical}
  而来。
	注意到
  $
  \Big|\{r\in [n] \big| X_r = w^i, \sigma_r=w^i \}\Big|
  =\frac{n}{k} - I_i$和
	$|\{r\in [n] | X_r \neq \omega^i, \sigma_r = \omega^i \}| = I'_i$,
	我们因此有  
  \begin{equation}\label{eq:number_of_sigma_r_equal_omega_i}
    |\{r\in [n] | \sigma_r = \omega^i \}|
    = I'_i + \frac{n}{k} - I_i
  \end{equation}
  类比式\eqref{eq:split_one_tmp_X_ij}从而得到
  式\eqref{eq:split_one_tmp_sigma_ij}。

  由式\eqref{eq:Pratio} 我们得到
  \begin{equation}
    \frac{P_{\sigma |G } (\sigma = \bar{\sigma})}
    {P_{\sigma |G } (\sigma = X)}
    = \exp(-\beta(H(\bar{\sigma})
    - H(X)))
  \end{equation}
  结合式\eqref{eq:log_P_sigma_G_bar_sigma}
  可得
  \begin{equation}\label{eq:Hgeneral_tmp_derived}
    H(\bar{\sigma}) - H(X)=
    -\left(1 + \gamma \frac{ \log n}{n}
    \right)
    \left[
      \sum_{\bar{\sigma}_i = \bar{\sigma}_j}Z_{ij} - \sum_{X_i = X_j}Z_{ij}
      \right]
    + \gamma\frac{ \log n}{n}
    \left[\sum_{\bar{\sigma}_i = \bar{\sigma}_j}1 - \sum_{X_i = X_j}1
    \right]
    \end{equation}
  将式\eqref{eq:Hgeneral_tmp_derived} 与式\eqref{eq:Hgeneral}
  对比可得:
  \begin{align}
    B_{\bar{\sigma}} - A_{\bar{\sigma}} &=
    \sum_{\bar{\sigma}_i = \bar{\sigma}_j}Z_{ij}
    - \sum_{X_i = X_j}Z_{ij}
    \label{eq:B_bar_A_bar_lem_minus}\\
    N_{\bar{\sigma}} &=
    \sum_{\bar{\sigma}_i = \bar{\sigma}_j} 1  -\sum_{X_i = X_j} 1
    \label{eq:N_bar_sigma_lem_minus}
  \end{align}
  由式\eqref{eq:split_one_tmp_X_ij},
  \eqref{eq:split_one_tmp_sigma_ij}, 
  \eqref{eq:N_bar_sigma_lem_minus} 可得
  $N_{\bar{\sigma}} = \sum_{\bar{\sigma}_i = \bar{\sigma}_j} 1  -\sum_{X_i = X_j} 1 = \frac{1}{2}\sum_{i=0}^{k-1} (I_i - I_i')^2 $,
  即式 \eqref{eq:N_w}。
	另一方面,由式\eqref{eq:B_bar_A_bar_lem_minus},
  \eqref{eq:split_zij_tmp_X_ij}, 
  \eqref{eq:split_zij_tmp_sigma_ij}
  可得 $B_{\bar{\sigma}} - A_{\bar{\sigma}} = \sum_{X_i \neq X_j, \bar{\sigma}_i = \bar{\sigma}_j} Z_{ij} - \sum_{X_i = X_j, \bar{\sigma}_i \neq \bar{\sigma}_j} Z_{ij}$。
	对于在求和式 $\sum_{X_i \neq X_j, \bar{\sigma}_i = \bar{\sigma}_j} Z_{ij}$
  中出现的 $Z_{ij}$, $Z_{ij} \sim \textrm{Bernoulli}(q)$
  并且我们有
  \begin{align}
  \sum_{X_i \neq X_j, \bar{\sigma}_i = \bar{\sigma}_j} 1 
  &= \sum_{i=0}^{k-1}[(\frac{n}{k} - I_i) I_i' + \frac{1}{2} \sum_{j=0}^{k-1} I_{ji}(I'_i - I_{ji}) ] 
  \label{eq:x_i_neq_x_j_bar_sigma_i_eq_bar_sigma_j_0}\\
	&=\frac{n}{k}|\cI| + \frac{1}{2}\sum_{i=0}^{k-1}  (-2 I'_i I_i  + I'^2_i - \sum_{j=0}^{k-1} I^2_{ji}) 
  \label{eq:x_i_neq_x_j_bar_sigma_i_eq_bar_sigma_j_1}
  \end{align}
  在式\eqref{eq:x_i_neq_x_j_bar_sigma_i_eq_bar_sigma_j_0}
  最外层的求和中,首先遍历的是$X_i=1, \omega, \dots, \omega^{k-1}$。  
  而式\eqref{eq:split_one_tmp_sigma_ij}
  利用了
  $|\cI| = \sum_{i=0}^{k-1} I'_i$。
  类似的,
  对于在求和式 $\sum_{X_i = X_j, \bar{\sigma}_i \neq \bar{\sigma}_j} Z_{ij}$
  中出现的 $Z_{ij}$, $Z_{ij} \sim \textrm{Bernoulli}(p)$。
	$\sum_{X_i = X_j, \bar{\sigma}_i \neq \bar{\sigma}_j} 1
	= \sum_{i=0}^{k-1}[(\frac{n}{k} - I_i) I_i + \frac{1}{2} \sum_{j=0}^{k-1} I_{ij}(I_i - I_{ij}) ] 
	= \frac{n}{k}|\cI| - \frac{1}{2}\sum_{i=0}^{k-1}  (I^2_i + \sum_{j=0}^{k-1} I^2_{ij})$。
  从而式\eqref{eq:B_w}、\eqref{eq:A_w} 得证。
\end{proof}

从上面可以看到引理 \ref{lem:minus} 的证明并没有很多技巧性,主要涉及一些计数
的技巧。
当 $|\cI|$ 相比与 $n$ 很小时
我们有如下的引理, 该引理是 \citet{ye2020exact} 中 陈述 6 的拓展。 

\begin{lemma}
  \label{lem:enhanced_fb}
	对于 $t\in [\frac{1}{k}(b-a), 0], 0<\delta <1$
	和 $ |\cI| \le n/\log^{\delta} n$,我们有
\begin{equation} \label{eq:upmpt}
	\begin{aligned}
	& P_G(B_{\bar{\sigma}}-A_{\bar{\sigma}}\ge t |\cI| \log n)  \\
	\le & \exp\Big(|\cI|\log n
	\left(
    f_{\beta}(t) - \beta t -1	+ O((\log n)^{-\delta}) \right)\Big)
	\end{aligned}
	\end{equation}
	其中 $f_{\beta}(t) := \min_{s\geq 0} (g(s) - st) + \beta t \leq \tilde{g}(\beta) $。
  $P_G$中的下标$G$表示图$G$是由SSBM模型随机生成的。
  函数$g(\beta), \tilde{g}(\beta)$
  的定义分别参见
  \eqref{eq:g_beta_main_article}
  和
  \eqref{eq:g_tilde_beta_main_article} 式。
\end{lemma}

\begin{proof}
  由$f_{\beta}(t)$函数的定义,
  \eqref{eq:upmpt}式中
  $f_{\beta}(t) - \beta t$ 与$\beta$无关而只与
  $t$有关。
	由定义可知,
  $A_{\bar{\sigma}}\sim \Binom(N_{A_{\bar{\sigma}}},\frac{a\log n}{n})$
  且
	$B_{\bar{\sigma}}\sim\Binom(N_{B_{\bar{\sigma}}},\frac{b\log n}{n})$,
  并且二者彼此独立。
  由 (\ref{eq:N_w}-\ref{eq:A_w})可知,$N_{B_{\bar{\sigma}}} - N_{A_{\bar{\sigma}}}
  = N_{\bar{\sigma}}$。
  对于$s>0$,
  $B_{\bar{\sigma}}-A_{\bar{\sigma}}$ 的矩生成函数可以展开成
	\begin{align*}
	 \mathbb{E}[e^{s(B_{\bar{\sigma}}-A_{\bar{\sigma}})}] 
	& =\Big(1-\frac{b\log n}{n}+\frac{b\log n}{n} e^s \Big)^{N_{B_{\bar{\sigma}}}}
	\Big(1-\frac{a\log n}{n}+\frac{a\log n}{n} e^{-s} \Big)^{N_{A_{\bar{\sigma}}}}  \\
	& = \exp\left(\frac{\log n}{n}\left(N_{B_{\bar{\sigma}}}
  (be^s - b) + N_{A_{\bar{\sigma}}}(ae^{-s}-a) + N_{B_{\bar{\sigma}}}
  O \left(\frac{\log n}{n} \right) \right) \right)
	\end{align*}
	由\eqref{eq:B_w}
  和\eqref{eq:A_w}
  注意到$k$是个常数,
  当
  $|\cI| \le n/\log^{\delta}(n)$ 时
	我们能估计
  $N_{B_{\bar{\sigma}}}$和$N_{A_{\bar{\sigma}}}$
  的阶数。它们分别是
  $N_{A_{\bar{\sigma}}} = \frac{n}{k}|\cI| + O(|\cI|^2)$,
	 $N_{B_{\bar{\sigma}}} = \frac{n}{k}|\cI| + O(|\cI|^2)$。
	\begin{align*}
	\mathbb{E}[e^{s(B_{\bar{\sigma}}-A_{\bar{\sigma}})}]
   \le&
	\exp(\frac{\log n}{n}\Big((\frac{n}{k}|\cI| + O(|\cI|^2))(be^s - b) +(\frac{n}{k} |\cI| + O(|\cI|^2))(ae^{-s}-a) \\
  &+ \frac{n}{k}|\cI|O(\frac{\log n}{n}) \Big))\\
	=
  & \exp\Big(\frac{|\cI| \log n}{k}(a e^{-s}+b e^s-a-b +
	O(\log^{-\delta}(n))) \Big),
	\end{align*}
	最后一个等式来自于
  $ |\cI| \le n/\log^{\delta}(n)$这个假设。
	由切尔诺夫不等式(参见 \eqref{eq:chernoff_bound}),
  对于 $s>0$,我们有
	\begin{align*} 
	& P_G(B_{\bar{\sigma}}-A_{\bar{\sigma}}\ge t |\cI| \log n)\le
	\frac{\mathbb{E}[e^{s(B_{\bar{\sigma}}-A_{\bar{\sigma}})}]}{e^{st |\cI|  \log n}}  \\
	\le & \exp\Big(\frac{|\cI|\log n}{k} \big(a e^{-s}+b e^s -kst -a-b
	+ O(\log^{-\delta}(n)) \big)\Big)  .
	\end{align*}
	上式右端对$s\geq 0$取最小值即有
  \eqref{eq:upmpt}。  
	\end{proof}

  对应于定理 \ref{thm:phase_transition} 的三种情况,我们需要使用若干个非平凡的引理来辅助证明。

  \begin{lemma}\label{lem:sigmaX}
    令 $\gamma > b$, 且 $\bar{\sigma}$ 满足 $\Dist(\bar{\sigma}, X) \geq \frac{n}{\sqrt{\log n}}$
    和 $D(\bar{\sigma}, X)$
    (参见\eqref{eq:D_sigma_sigma_prime})
    两个条件。
    事件
    $P_{\sigma | G}(\sigma = \bar{\sigma} ) > \exp(-Cn) P_{\sigma | G}(\sigma = X)$
    发生的概率小于 $\exp(-\tau(\gamma,\beta) n \sqrt{\log  n} )$,
    其中 $C$ 是任意事先给定的常数而 
    $\tau(\gamma,\beta)$
   是某个依赖于$\gamma$ 和 $\beta$的取值为正值的函数。
  \end{lemma}
  \begin{proof}
    我们把 事件 $P_{\sigma | G}(\sigma = \bar{\sigma} ) > \exp(-Cn) P_{\sigma | G}(\sigma = X)$
    记为 $\widetilde{D}(\bar{\sigma}, C)$。
    由 \eqref{eq:Hgeneral} 式, $\widetilde{D}(\bar{\sigma}, C)$
    等价于:
  \begin{equation}\label{eq:BwA}
    (1 + \frac{\gamma \log n}{n})[B_{\bar{\sigma}} - A_{\bar{\sigma}}] >  \frac{\gamma \log n}{n} N_{\bar{\sigma}}  - \frac{C}{\beta} n
    \end{equation}  
    
    我们声称$\bar{\sigma}$  必须满足以下两个条件中的至少一个。
    \begin{enumerate}
      \item $\exists i\neq j$ s.t. $\frac{1}{k(k-1)}\frac{n}{\sqrt{\log n}} \leq I_{ij} \leq \frac{n}{k} - \frac{1}{k(k-1)}\frac{n}{\sqrt{\log n}}$
      \item $\exists i \neq j$ s.t. $I_{ij} > \frac{n}{k} - \frac{1}{k(k-1)}\frac{n}{\sqrt{\log n}}$ and $I_{ji} < \frac{1}{k(k-1)}\frac{n}{\sqrt{\log n}}$
    \end{enumerate}
    $I_{ij}$的定义参见引理
    \ref{lem:minus}。
    如果上面两个条件均不成立,则由条件1,对 $0 \leq i,j\leq k-1$,
    我们有
    $I_{ij} < \frac{1}{k(k-1)}\frac{n}{\sqrt{\log n}}$
    或者 $I_{ij} > \frac{n}{k} - \frac{1}{k(k-1)}\frac{n}{\sqrt{\log n}}$。
    
    因为 $\sum_{i,j} I_{ij} = |\cI| \geq \frac{n}{\sqrt{\log n}}$,
    存在 $i,j$ 使得
    $I_{ij} > \frac{n}{k} - \frac{1}{k(k-1)}\frac{n}{\sqrt{\log n}}$.
    对于这一对确定的 $i,j$,不妨先假设 $I_{ji} > \frac{n}{k} - \frac{1}{k(k-1)}\frac{n}{\sqrt{\log n}}$,
    下面我们将推出一对矛盾说明该假设不成立。
    取 $X'$ 为在$X$ 的基础上交换$w^i$ 和 $w^j$ 位置的向量,
    然后我们考虑
  \begin{align}
     \Dist(\bar{\sigma}, X') & - \Dist(\bar{\sigma}, X)= |\{ r \in [n]|X_r=w^i, \bar{\sigma}_r \neq w^j \}| \notag\\
    &+ |\{ r \in [n]|X_r=w^j, \bar{\sigma}_r \neq w^i \}| \notag\\
    &-|\{ r \in [n]|X_r=w^i, \bar{\sigma}_r \neq w^i \}| \notag\\
    &- |\{ r \in [n]|X_r=w^j, \bar{\sigma}_r \neq w^j \}| \notag\\
    & = \frac{n}{k} - I_{ij} +  \frac{n}{k} - I_{ji} - I_i - I_j\label{eq:distsmall} \\
    & < \frac{2}{k(k-1)}\frac{n}{\sqrt{\log n}} - I_i - I_j < 0\notag
    \end{align}
    这和 $\bar{\sigma}$ 离 $X$ 最近的条件矛盾。
    因此
    $I_{ji} > \frac{n}{k} - \frac{1}{k(k-1)}\frac{n}{\sqrt{\log n}}$
    不成立,由条件1
    则有  $I_{ji} < \frac{1}{k(k-1)}\frac{n}{\sqrt{\log n}}$。
    但这样一来 $(i, j)$ 二元组满足 条件 2,
    又与 $\bar{\sigma}$ 不满足如上两个条件的假设相矛盾。
    
    在条件1的假设下,我们能获得 $N_{A_{\bar{\sigma}}}$的一个下界。
    为方便估计该下界,我们首先改写\eqref{eq:A_w}式。
    对于 $i\neq j$ ,令 $I'_{ij} = I_{ij}$,
    而 $I'_{ii} = \frac{n}{k} - I_i$。
    则我们有:
    \begin{align*}
    N_{A_{\bar{\sigma}}} &=
    \frac{n}{k}|\cI| - \frac{1}{2}\sum_{i=0}^{k-1}
    (I^2_i + \sum_{j=0}^{k-1} I^2_{ij}) \\
    &=
    \frac{n}{k}|\cI| - \frac{1}{2}\sum_{i=0}^{k-1}
    ((\frac{n^2}{k^2} - 2\frac{n}{k} I'_{ii}) +
    \sum_{j=0}^{k-1} I'^2_{ij}) \\
    &=
    \frac{n}{k}|\cI| - \frac{1}{2}\sum_{i=0}^{k-1}
    ((\frac{n^2}{k^2} - 2\frac{n}{k} (\frac{n}{k} - I_i)))
    - \frac{1}{2}\sum_{i=0}^{k-1}\sum_{j=0}^{k-1} I'^2_{ij} \\
    &= \frac{n^2}{2k} - \frac{1}{2}
    \sum_{i=0}^{k-1} \sum_{j=0}^{k-1} I'^2_{ij}
   \end{align*}
    
    在求和式中 $\sum_{i=0}^{k-1} \sum_{j=0}^{k-1} I'^2_{ij}$
    的每一项 $\sum_{j=0}^{k-1} I'^2_{ij}$ 其满足 $\sum_{j=0}^{k-1}
    I'_{ij} = \frac{n}{k}$,当某个$I'_{ii}$取$\frac{n}{k}$
    而其余的项为零时$\sum_{j=0}^{k-1} I'^2_{ij}$
    最大。但由于条件1成立,有某一行不能取
    这种极端值,所以 $\sum_{i=0}^{k-1} \sum_{j=0}^{k-1} I'^2_{ij}$中
    共有 $(k-1)$行求和得$ (k-1)\frac{n^2}{k^2}$
    而其中一行则必须保留$I'_{ij}, I'_{ii}$ 这两个非零项。
    因此 $\sum_{i=0}^{k-1} \sum_{j=0}^{k-1} I'^2_{ij}
    \leq (k-1)\frac{n^2}{k^2} + (\frac{n}{k} - I_{ij})^2 + I^2_{ij}$。
    而这里的
    $I_{ij}$ 满足条件1。
    进一步的, $\sum_{i=0}^{k-1} \sum_{j=0}^{k-1} I'^2_{ij} \leq (k-1)\frac{n^2}{k^2} + (\frac{1}{k(k-1)}\frac{n}{\sqrt{\log n}})^2
    + (\frac{n}{k} - \frac{1}{k(k-1)}\frac{n}{\sqrt{\log n}})^2 = \frac{n^2}{k} - \frac{2n^2}{k^2 (k-1)\sqrt{\log n}}(1+o(1))$。
    从而我们得到
  \begin{equation}\label{eq:Asigma}
    A_{\bar{\sigma}} \geq \frac{n^2}{k^2 (k-1)\sqrt{\log n}}(1+o(1))
    \end{equation}
    
    
    在条件2的假设下, 我们能得到
    $N_{\bar{\sigma}}$的一个下界。
    因为
    $\Dist(\bar{\sigma}, X') - \Dist(\bar{\sigma}, X) \geq 0$,
     由 \eqref{eq:distsmall} 式 我们有
    $I_{ij} + I_{ji} + I_{i} + I_j \leq \frac{2n}{k} $。
    又因为 $I_i \geq I_{ij} > \frac{n}{k} - \frac{1}{k(k-1)}\frac{n}{\sqrt{\log n}}$,
    从而$I_j $ 满足
    $I_j \leq \frac{2}{k(k-1)}\frac{n}{\sqrt{\log n} }$。
    进一步有 $I'_j - I_j \geq I_{ij} - I_j \geq  \frac{n}{k} - \frac{3}{k(k-1)}\frac{n}{\sqrt{\log n} }$。
    由 \eqref{eq:N_w}
    式 $N_{\bar{\sigma}} \geq \frac{1}{2}(\frac{n}{k} - \frac{3}{k(k-1)}\frac{n}{\sqrt{\log n}})^2 = \frac{n^2}{2k^2}(1+o(1))$.
    
    下面我们用
     切尔诺夫不等式给出\eqref{eq:BwA}式描述的事件发生概率的上界。
     在分析中我们可以省略不等号左边的 $\frac{\gamma \log n}{n}$ 一项
     ,因为其远小于$1$。
    令 $Z \sim \textrm{Bernoulli}(\A), Z' \sim \textrm{Bernoulli}(\B)$,则:
    \begin{align*}
    &P_G(\widetilde{D}(\bar{\sigma}, C))
    \leq (\mathbb{E}[\exp(sZ')])^{N_{B_{\bar{\sigma}}}}
    (\mathbb{E}[\exp(-sZ)])^{N_{A_{\bar{\sigma}}}}\\
    &\cdot \exp(-s(\frac{\gamma \log n}{n} N_{\bar{\sigma}}  - \frac{C}{\beta}n)) \\
    & \leq \exp\Big(N_{B_{\bar{\sigma}}}
    \frac{b\log n}{n}(e^s -1) + N_{A_{\bar{\sigma}}}\frac{a\log n}{n} (e^{-s} - 1) \\
    &-s(\frac{\gamma \log n}{n} N_{\bar{\sigma}}  - \frac{C}{\beta}n)\Big) 
    \end{align*}
    
    使用 $N_{B_{\bar{\sigma}}} = N_{\bar{\sigma}} + N_{A_{\bar{\sigma}}}$
    我们可以进一步简化上式中的指数项为:
    $$
    \frac{\log n}{n} [N_{A_{\bar{\sigma}}}(b(e^s -1)+ a(e^{-s} - 1)) +
    N_{\bar{\sigma}} (b(e^s - 1)-\gamma s)]  + s \frac{C}{\beta}n
    $$
    
    下面我们考察 函数 $g_1(s) = b(e^s -1)+ a(e^{-s} - 1)$
    和 $g_2(s) = b(e^s - 1)-\gamma s$。
    两个函数 在 $s=0$ 处取值为零
    并且 $g_1'(s) = (be^s - ae^{-s}), g_2'(s) = be^s -\gamma$。
    因此, $g_1'(0) = b-a<0, g_2'(0) = b - \gamma < 0$
    我们可以选择 $s^*>0$,使得 $g_1(s^*) < 0,g_2(s^*) < 0$。
    为了补偿 $sCn/\beta$ 这一项的影响,我们只需要确保
    $\frac{\log n}{n} \min\{N_{A_{\bar{\sigma}}}, N_{\bar{\sigma}}\}$ 的阶次大于$n$。
    由于   $N_{A_{\bar{\sigma}}}\geq \frac{n^2}{k^2 (k-1)\sqrt{\log n}}(1+o(1))$ 或 $N_{\bar{\sigma}} \geq \frac{n^2}{2k^2}(1+o(1))$
    有一个成立,
    这一要求可以得到了满足。
  \end{proof}

\begin{lemma}\label{prop:small}
  若 $\gamma>b$, $\beta>\beta^\ast$,
  对于 $1\leq r \leq \frac{n}{\sqrt{\log n}}$
  和 $\forall \epsilon > 0$,
  存在集合 $\cG^{(r)}$ 使得:
\begin{equation}\label{eq:Gr}
  P_G(\cG^{(r)}_n) \ge 1 - n^{r(\tilde{g}(\beta)/2 + \epsilon)}
  \end{equation}
  并且对于任意 $G\in\cG^{(r)}_n$,有
\begin{equation}\label{eq:psigmaX}
  \frac{P_{\sigma|G}(\Dist(\sigma, X)=r | D(\sigma, X))}
  {P_{\sigma|G}(\sigma=X | D(\sigma, X))} <
  n^{r \tilde{g}(\beta) /2}
  \end{equation}
  
  对于 $r> \frac{n}{\sqrt{\log n}}$, 存在集合 $\cG^{(r)}$ 使得:
\begin{equation}\label{eq:Gr1}
  P(G\in\cG^{(r)}_n) \ge 1 - e^{-n}
  \end{equation}
  并且对于任意 $G\in\cG^{(r)}_n$,有
\begin{equation}\label{eq:psigmaX1}
  \frac{P_{\sigma|G}(\Dist(\sigma, X)=r | D(\sigma, X))}
  {P_{\sigma|G}(\sigma=X | D(\sigma, X))} <
  e^{-n}
  \end{equation}
\end{lemma}
\begin{proof}
  我们将讨论分为两种情况: $r\leq \frac{n}{\sqrt{\log n}}$
  和 $r > \frac{n}{\sqrt{\log n}}$。
  
  当 $r\leq \frac{n}{\sqrt{\log n}}$ 时,
  通过使用 关于 $\Dist$
  的三角不等式,
  我们可以证明 
  $\Dist(\sigma, X) = r$ 蕴含
  $D(\sigma, X)$。
   对于 $f \in S_k \backslash \{ \textrm{id} \}$,
   其中
   $\textrm{id}$ 是恒等映射,
   我们有:
  $$
  \frac{2n}{k} \leq \Dist(f(X), X) \leq \Dist(\sigma, f(X)) + \Dist(\sigma, X)
  $$
  
  因此,
  $\Dist(\sigma, f(X)) \geq \frac{2n}{k} - \frac{n}{\sqrt{\log n}} \geq \Dist(\sigma, X)$。
  从而我们得到
  \eqref{eq:psigmaX} 式 等价于:
  \begin{equation}\label{eq:psigmaX2}
  \frac{P_{\sigma|G}(\Dist(\sigma, X)=r)}
  {P_{\sigma|G}(\sigma=X)} <
  n^{r \tilde{g}(\beta) /2}
  \end{equation}
  
  上式左边可以写为:
  \begin{align*}
  \frac{P_{\sigma|G}(\Dist(\sigma, X)=r)}
  {P_{\sigma|G}(\sigma=X)}  &= \sum_{\Dist(\bar{\sigma}, X)=r}\exp(-\beta(H(\bar{\sigma})-H(X)))\\
  \textrm{ by \eqref{eq:Hgeneral} } &\leq \sum_{\Dist(\bar{\sigma}, X)=r}\exp(\beta_n(B_{\bar{\sigma}}-A_{\bar{\sigma}}))
  \end{align*}
  其中 
  \begin{equation}\label{eq:beta_n_def}
  \beta_n = \beta(1+\gamma\frac{\log n}{n})
  \end{equation}
  
  定义
  $\Xi_n(r): = \sum_{\Dist(\bar{\sigma}, X)=r}\exp(\beta_n(B_{\bar{\sigma}}-A_{\bar{\sigma}}))$,
  我们只需证明:
  \begin{equation}
  P_{G}(\Xi_n(r) \geq n^{r \tilde{g}(\beta) /2}) \leq  n^{r (\tilde{g}(\beta) /2 + \epsilon)}
  \end{equation}
  
  定义 事件 $\Lambda_n(G,r):=\{B_{\bar{\sigma}} -A_{\bar{\sigma}} < 0, \forall \bar{\sigma}\, s.t. \Dist(\bar{\sigma}, X)=r\}$,
  然后我们采用下面的方式进行处理
  \begin{align*}
  P_{G}(\Xi_n(r) \geq n^{r \tilde{g}(\beta) /2}) \leq
  P_G(\Lambda_n(G,r)^c)
  + P_G\left(
    \Xi_n(r) \geq n^{r \tilde{g}(\beta) /2} |\Lambda_n(G,r) 
    \right)
  \end{align*}
  
  对于第一项, 因为
  $|\{ \bar{\sigma} | \Dist(\bar{\sigma}, X) = r \}| \leq (k-1)^r n^r$,
  由 引理 \ref{lem:enhanced_fb} 可得
  $P_G(\Lambda_n(G,r)^c) \leq (k-1)^r n^{rg(\bar{\beta})} \leq n^{r (\tilde{g}(\beta) /2 + \epsilon/2)}$。

  对于第二项,我们使用马尔可夫不等式:
  \begin{align*}
  P_G(\Xi_n(r) \geq n^{r \tilde{g}(\beta) /2} |\Lambda_n(G,r) )
  \leq \mathbb{E}[\Xi_n(r)|\Lambda_n(G,r)]n^{-r \tilde{g}(\beta) /2} 
  \end{align*}
  
  而上式中的条件期望可以做如下估计:
  \begin{align*}
  &\mathbb{E}[\Xi_n(r)|\Lambda_n(G,r)]=
  \sum_{\Dist(\bar{\sigma}, X) = r}\sum_{tr\log n = -\infty }^{-1} 
   P_G(B_{\bar{\sigma}} -A_{\bar{\sigma}}=tr\log n)\exp(\beta_n tr \log n) \\
  & \leq (k-1)^r n^{r+r\beta_n(b-a)/k} +
  \sum_{\Dist(\bar{\sigma}, X) = r}\sum_{tr\log n = r\frac{b-a}{k}\log n }^{-1} 
   P_G(B_{\bar{\sigma}} -A_{\bar{\sigma}}=tr\log n)\exp(\beta_n tr \log n)
  \end{align*}
  $1+\beta_n(b-a)/k = f_{\beta_n}(\frac{b-a}{k}) < \tilde{g}(\beta_n)$,
  因此,
  $(k-1)^r n^{r+r\beta_n (b-a)/k}n^{-r \tilde{g}(\beta) /2} \leq$
   $n^{r (\tilde{g}(\beta) /2 + \epsilon/2)} $。
  使用 引理 \ref{lem:enhanced_fb}, 我们有:
  \begin{align*}
  P_G(B_{\bar{\sigma}} -A_{\bar{\sigma}}=tr\log n)\exp(\beta_n rt \log n) \leq 
  n^{r(f_{\beta_n}(t)-1 + O(\frac{1}{\sqrt{\log n}}))}
  \end{align*}
  
  因为 $\beta_n \to \beta$, $\forall \epsilon$,
  当 $n$ 充分大时
  我们有 $\tilde{g}(\beta_n) \leq \tilde{g}(\beta) + \epsilon /2$。
  因此,
  \begin{align*}
  &n^{-r \tilde{g}(\beta) /2}\sum_{\Dist(\bar{\sigma}, X) = r}\sum_{\substack{tr\log n = \\ r(b-a)/k\log n} }^{-1}
  P_G(B_{\bar{\sigma}} -A_{\bar{\sigma}}=t\log n)\exp(\beta_n rt \log n) \\
  & \leq  (k-1)^r r\frac{b-a}{k} ( \log n )
  n^{r(\tilde{g}(\beta_n) - \tilde{g}(\beta)/2)}\\
  & \leq  n^{r(\tilde{g}(\beta)/2 + \epsilon/2)} O(\log n) (k-1)^r
  \end{align*}
  
  结合上面的式子, 我们可以得到:
  \begin{align*}
  P_{G}(\Xi_n(r) \geq n^{r \tilde{g}(\beta) /2}) &\leq  n^{r(\tilde{g}(\beta)/2 + \epsilon/2)} O(\log n) (k-1)^r\\
  &\leq n^{r(\tilde{g}(\beta)/2 + \epsilon)}
  \end{align*}
  
  当 $r>\frac{n}{\sqrt{\log n}}$时,
  由引理 \ref{lem:sigmaX},
  我们选取充分大的常数  $C>1$
  使得 $k^n\exp(-Cn) < e^{-n}$:
  \begin{align*}
  &\frac{P_{\sigma|G}(\Dist(\sigma, X)=r | D(\sigma, X))}
  {P_{\sigma|G}(\sigma=X | D(\sigma, X))} = 
  \sum_{\substack{D(\bar{\sigma}, X) \\ 
  \Dist(\bar{\sigma}, X)=r}} \frac{P_{\sigma | G}(\sigma = \bar{\sigma}) }{P_{\sigma | X}(\sigma = X)} > \exp(-n)\\
  &\Rightarrow \exists \bar{\sigma},
  \frac{P_{\sigma | G}(\sigma = \bar{\sigma}) }{P_{\sigma | X}(\sigma = X)} > k^{-n}\exp(-n)
  \Rightarrow \exists \bar{\sigma},
  \frac{P_{\sigma | G}(\sigma = \bar{\sigma}) }{P_{\sigma | X}(\sigma = X)} > \exp(-Cn)\\
  \end{align*}
  最后一个事件发生的概率小于 $k^ne^{-\tau(\gamma, \beta) n\sqrt{\log n}}<e^{-n}$。
  因此,   \eqref{eq:psigmaX1} 式成立。
  \end{proof}

  如果 $\gamma > b$ 且 $\beta < \beta^*$,则我们有如下的引理:
 \begin{lemma}\label{lem:7}
	若 $\gamma > b$ 且 $\beta < \beta^*$,
  则存在集合 $\cG^{(1)}_n$ 使得
	$P_G(\cG^{(1)}_n) \geq 1-n^{g(\bar{\beta})}$
	并且:
 \begin{align}
	\mathbb{E}\left[
    \sum_{r=1}^n \exp(\beta_n (A_r^s - A_r^0)) | G \in \   cG^{(1)}_n\right] &= (1+o(1))n^{g(\beta_n)} \\
	\Var \left[
    \sum_{r=1}^n \exp(\beta_n (A_r^s - A_r^0)) | G \in \cG^{(1)}_n
    \right] &\leq n^{g(\beta_n)}
	\end{align}
\end{lemma}
引理 \ref{lem:7} 是 \citet{ye2020exact} 文中陈述 10 
的拓展,并且可以用几乎一样的技巧证明。
因此我们在这里省略引理\ref{lem:7}的证明。
% This line can be changed if more detailed
% proofs are better

\begin{lemma}\label{prop:large2}
	若 $\gamma > b$ 且 $\beta < \beta^*$,
 存在一个集合 $\cG'_n$ 使得:
\begin{equation}
	P_G(\cG'_n) \geq 1 - (1+o(1))\max\{n^{g(\bar{\beta})}, n^{- g(\beta_n) + \epsilon} \}
	\end{equation}
	并且对任意 $G \in \cG'_n$,
\begin{equation}\label{eq:diff1g}
	\frac{P_{\sigma|G}(\Dist(\sigma, X)=1 | D(\sigma, X))}
	{P_{\sigma|G}(\sigma=X | D(\sigma, X))} \geq (k-1+o(1))n^{g(\beta_n)}
	\end{equation}
  其中 $\beta_n$ 在\eqref{eq:beta_n_def}式 中定义。
\end{lemma}

\begin{proof}
根据\eqref{eq:energy_diff} 式,
\eqref{eq:diff1g}式的左边可以写成:
\begin{equation}\label{eq:knd}
	\frac{P_{\sigma|G}(\Dist(\sigma, X)=1)}
{P_{\sigma|G}(\sigma=X)}= (1+o(1))\sum_{s=1}^{k-1}\sum_{r=1}^n \exp(\beta_n (A_r^s - A_r^0))
\end{equation}

令 $\cG^{(1)}_n$ 定义成 引理\ref{lem:7} 所示的形式
并且 $\cG^{(2)}_n: = \{|\sum_{r=1}^n \exp(\beta_n (A_r^s - A_r^0)) -\linebreak (1+o(1))n^{g(\beta_n)}  | \leq n^{g(\beta_n) - \epsilon / 2} \}$.

使用切比雪夫不等式, 我们有:
\begin{equation*}
  P_G(G \not\in \cG^{(2)}_n \Big\vert  G \in \cG^{(1)}_n) \leq n^{- g(\beta_n) + \epsilon}
  \end{equation*}
  
  令 $\cG'_n = \cG^{(1)}_n \cap \cG^{(2)}_n$:
  \begin{align*}
  P_G(G \in \cG'_n) &= P_G(\cG^{(1)}_n) P_G(G \in \cG_n^{(2)} | G \in \cG_n^{(1)}) \\
  & \geq (1-n^{ - g(\beta_n) + \epsilon})(1-n^{g(\bar{\beta})}) \\
  &= 1-(1+o(1))\max\{n^{g(\bar{\beta})}, n^{- g(\beta_n) + \epsilon} \}
  \end{align*}
  则对任意的 $G\in\cG'_n$,
  \begin{equation*}
  \sum_{r=1}^n \exp(\beta_n (A_r^s - A_r^0)) = (1+o(1)) n^{g(\beta_n)}
  \end{equation*}
  
  因此,
  由\eqref{eq:knd} 式我们可得:
  \begin{equation*}
    \frac{P_{\sigma|G}(\Dist(\sigma, X)=1)}
  {P_{\sigma|G}(\sigma=X)} \geq (k-1+o(1)) n^{g(\beta_n)}
  \end{equation*}
\end{proof}

\newglossaryentry{not:all_one_vector_n}
{
  type=notation,
  name={\ensuremath{\mathbf{1}_n}},
  description={长度为 $n$的全1的向量}
}


令
\begin{equation}\label{eq:Big_Lambda}
  \Lambda := \{ \omega^j  \cdot \mathbf{1}_n | j=0, \dots,k-1\}   
\end{equation}

其中 \gls{not:all_one_vector_n} 是 \glsdesc{not:all_one_vector_n} %confirm.
则我们有如下的引理:
 \begin{lemma}\label{lem:small}
	假设 $\gamma < b $ 且 $\bar{\sigma}$
  满足 
  $\Dist(\bar{\sigma}, \mathbf{1}_n) \geq \frac{n}{\sqrt{\log  n}}$
	且 $D(\bar{\sigma}, \mathbf{1}_n)$.
	则事件
	$P_{\sigma | G}(\sigma = \bar{\sigma} ) > \exp(-Cn) P_{\sigma | G}(\sigma = \mathbf{1}_n)$
	发生的概率 小于 $\exp(-\tau(\gamma,\beta) n \sqrt{\log  n} )$,
  其中 $C$ 是任意事先给定的常数,
  $\tau(\gamma,\beta)$ 是某个依赖于$\gamma$和$\beta$的取值为正值的函数。
\end{lemma}
引理\ref{lem:small}和引理\ref{lem:sigmaX} 具有某种对称性,但其证明不同,
引理\ref{lem:small}的证明如下。
\begin{proof}
	令 $n_r = |\{\bar{\sigma}_i = w^r | i\in [n] \}|$。
  因为 $\arg\,\min_{\sigma'\in \Lambda} \Dist(\bar{\sigma}, \sigma') = \mathbf{1}_n$,
  我们有 $n_0 \geq n_r$ 对 $r=1, \dots, k-1$成立 
  。
	不失一般性 我们假设 \mbox{$n_0 \geq n_1 \dots \geq n_{k-1}$}。
	定义
  $N_{\bar{\sigma}} = \frac{1}{2}(n(n-1) - \sum_{r=0}^{k-1} n_r(n_r-1))
	=\frac{1}{2}(n^2 - \sum_{r=0}^{k-1} n_r^2)$。
	简记事件
  $P_{\sigma | G}(\sigma = \bar{\sigma} ) > \exp(-Cn) P_{\sigma | G}(\sigma = \mathbf{1}_n)$ 为
  $D'(\bar{\sigma}, C)$,
	该不等式等价于:
\begin{align}
	(1 + \frac{\gamma \log n}{n})
	\left( \sum_{\bar{\sigma}_i  \neq \bar{\sigma}_j, X_i = X_j} Z_{ij} +
	\sum_{\bar{\sigma}_i  \neq \bar{\sigma}_j, X_i \neq X_j} Z_{ij} \right)\notag\\
	\leq \frac{\gamma \log n}{n} N_{\bar{\sigma}} + \frac{C}{\beta} n\label{eq:small}
\end{align}
		
	首先,我们估计 $N_{\bar{\sigma}}$的阶数,
  显而易见 $N_{\bar{\sigma}} \leq \frac{1}{2} n^2$。
	其次,使用 \citet{chen2016information} 附录 A 中的结论,
  我们有:
\begin{equation}
	\sum_{r=0}^{k-1} n_r^2 \leq
	\begin{cases}
	n n_0 & n_0 \leq \frac{n}{2} \\
	n^2 - 2n_0(n-n_0) & n_0 > \frac{n}{2}
	\end{cases}
	\end{equation}
	
	由$\Dist(\bar{\sigma}, \mathbf{1}_n) \geq \frac{n}{\sqrt{\log n}}$
  的假设,
  我们有 $n_0 \leq n - \frac{n}{\sqrt{\log n}}$
	并且从 $n_0 \geq n_r$ 推出 $n_0 \geq \frac{n}{k}$。
	当 $n_0 > \frac{n}{2}$ 时,
	我们有
  $N_{\bar{\sigma}} \geq n_0 (n - n_0) \geq \frac{n^2}{\sqrt{\log n}}(1+o(1))$.
	当 $n_0 = n - \frac{n}{\sqrt{\log n}}$时,第二个不等式可达。
  当 $n_0 < \frac{n}{2}$ 时,
	$N_{\bar{\sigma}} \geq \frac{n^2 - nn_0}{2} \geq \frac{n^2}{4}$,
  并且当 $n_0 = \frac{n}{2}$时,
  第二个不等式成立。
	因此一般地我们有 $\frac{n^2}{\sqrt{\log n}}(1+o(1)) \leq N_{\bar{\sigma}} \leq \frac{n^2}{2}$。
	
	因为 $\frac{C}{\beta} n = o(\frac{\log n}{n} N_{\bar{\sigma}})$,
  我们可以将  \eqref{eq:small} 式写成:
\begin{equation}
	\left( \sum_{\substack{\bar{\sigma}_i  \neq \bar{\sigma}_j \\ X_i = X_j}} -Z_{ij}
	+ \sum_{\substack{\bar{\sigma}_i  \neq \bar{\sigma}_j \\ X_i \neq X_j}} -Z_{ij} \right)\geq -\gamma\frac{\log n N_{\bar{\sigma}}}{n}(1+o(1))
	\end{equation}
	
	令 $N_1 = \sum_{\bar{\sigma}_i  \neq \bar{\sigma}_j, X_i = X_j} 1$
	且 $N_2 = \sum_{\bar{\sigma}_i  \neq \bar{\sigma}_j, X_i \neq X_j} 1 = N_{\bar{\sigma}} - N_1$。
	
	应用 切尔诺夫不等式 我们有:
	\begin{align*}
	P_G(D'(\bar{\sigma}, C))&
	\leq (\mathbb{E}[\exp(-s Z )])^{N_1} (\mathbb{E}[\exp(-s Z' )])^{N_2} \cdot \exp(\gamma \frac{\log n N_{\bar{\sigma}} s}{n}(1+o(1))) \\
	&= \exp \Big( \frac{\log n}{n}(1+o(1))(e^{-s}-1)(aN_1 + bN_2) 
	+\gamma \frac{\log n N_{\bar{\sigma}} s}{n}(1+o(1))\Big)
	\end{align*}
	
	因为 $s > 0$ 且 $a>b$,
  我们进一步有:
	\begin{align*}
	P_G(D'(\bar{\sigma}, C))
	& \leq \exp( \frac{N_{\bar{\sigma}}\log n }{n}(b(e^{-s}-1)+ \gamma s + o(1))) 
	\end{align*}
	
	令 $h_b(x) = x - b -x\log \frac{x}{b}$,
  对于 $0<x<b$,该函数满足  $h_b(x) < 0$,
	取 $s=-\log\frac{\gamma}{b} > 0$,使用
	$N_{\bar{\sigma}} \geq \frac{n^2}{\sqrt{\log n}}$ 我们有:
	\begin{align*}
	P_G(D'(\bar{\sigma}, C))&\leq \exp( N_{\bar{\sigma}} \frac{\log n}{n} h_b(\gamma)(1+o(1))) \\
	& \leq \exp (h_b(\gamma) n \sqrt{\log n} (1+o(1)))
	\end{align*}
\end{proof}

\begin{proof}[定理 \ref{thm:phase_transition} 的证明]
	(1) 因为 $P_{\sigma}(\sigma \not \in S_k(X)) = \sum_{f\in S_k} P_{\sigma}(\sigma \neq f(X) | D(\sigma, f(X))) P_{\sigma}(D(\sigma, f(X)))$
  \footnote{$P_{\sigma}$中的下标 $\sigma$ 表示图$G$是随机的},
	我们只需证明 $P_{\sigma}(\sigma \neq X | D(\sigma, X)) \leq  n^{\tilde{g}(\beta)/2 + \epsilon}$。
	由 \mbox{引理 \ref{prop:small}}, 我们能 $\cG_n^{(r)}$ for $r=1,\dots, n$.
	令 $\cG'_n = \cap_{r=1}^n \cG_n^{(r)}$ 
  并且在 \eqref{eq:Gr} 式中选取 $\frac{\epsilon}{2}$;
  由 \eqref{eq:Gr} 和 \eqref{eq:Gr1} 式,我们有:
	\begin{align*}
	P_G(\cG'^c_n) &= P_G(\cup_{r=1}^n (\cG_n^{(r)})^c) \\
	&\leq \sum_{r=1}^{n/\sqrt{\log n } } n^{r(\tilde{g}(\beta)/2 + \epsilon/2)}  + n e^{-n} \\
	& \leq \frac{1}{2} n^{\tilde{g}(\beta)/2 + \epsilon}
	\end{align*}
	其中最后一个不等式估计用了等比数列的求和。

	另一方面, 对于每个 $G \in \cG'_n$,
  由 \eqref{eq:psigmaX} 和 \eqref{eq:psigmaX1} 式,
	我们有:
	\begin{align*}
	\frac{P_{\sigma | G}(\sigma \neq X | D(\sigma, X))}{1-P_{\sigma | G}(\sigma \neq X | D(\sigma, X))} &= \frac{P_{\sigma | G}(\sigma \neq X | D(\sigma, X))}{P_{\sigma|G}(\sigma=X | D(\sigma, X))} \\
	&< \sum_{r=1}^{n/\sqrt{\log n }}  n^{r\tilde{g}(\beta)/2} + n e^{-n}
	\end{align*}
	从中我们可以得到估计式 $P_{\sigma | G}(\sigma \neq X | D(\sigma, X))\leq \frac{1}{2}n^{\tilde{g}(\beta)/2 + \epsilon}$。
	最后我们有: 
	\begin{align*}
	P_{\sigma}(\sigma \neq X|D(\sigma, X)) &= \sum_{G\in \cG'_n} P_G(G)P_{\sigma |G}(\sigma \neq X | D(\sigma, X)) \\
	+ P_G(\cG'^c_n)
	& \leq n^{\tilde{g}(\beta)/2 + \epsilon}.
	\end{align*}
	
	(2) 当 $\beta < \beta^*$,
  由引理 \ref{prop:large2} 可知,对于任意的 $G \in \cG'_n$
	我们有
	$$
	\frac{1-P_{\sigma | G}(\sigma=X | D(\sigma, X))}{P_{\sigma | G}(\sigma=X | D(\sigma, X))}\geq (k-1+o(1))n^{g(\beta_n)}
	$$
	
	进而得到:
	$$
	P_{\sigma | G}(\sigma=X| D(\sigma, X)) \leq \left(\frac{1}{k-1}+o(1) \right) n^{-g(\beta_n)}
	$$
	
	因此:
	\begin{align*}
	&P_{\sigma}(\sigma=X| D(\sigma, X))  \leq  P(\cG'^c_n) \\
	&+ \sum_{G\in \cG'_n}P_G(G) P_{\sigma|G}(\sigma=X| D(\sigma, X)) \\
	& \leq \left(
    \frac{1}{k-1}+o(1) \right)
    n^{-g(\beta_n)} + (1+o(1)) \max\{n^{g(\bar{\beta})}, n^{- g(\beta_n) + \epsilon} \}\\
	& \leq (1+o(1)) \max\{n^{g(\bar{\beta})}, n^{-g(\beta) + \epsilon}  \}
	\end{align*}
	
	(3) 当 $\gamma < b$,
  对于任意的 $f\in S_k$,
  我们有
  $\Dist(f(X), \Lambda) = \frac{(k-1)n}{k} > \frac{n}{\sqrt{\log n}}$,
  其中 $\Lambda$ 在\eqref{eq:Big_Lambda}式中定义。
	因此,由引理 \ref{lem:small},
  我们可以找到一个图的集合 $\cG'_n$
  使得 $P_G(\cG'_n) \leq \exp(-nC)$
	并且对 $G\not \in \cG'_n$, $P_{\sigma |G}(\sigma = f(X)) \leq \exp(-Cn)$。
  因此
	\begin{align*}
	P_a(\hat{X}^*) \leq P_G(\cG'_n) + k! \exp(-Cn) = (1+k!)\exp(-Cn)
	\end{align*}
	因为 $C$
  可取任意正数,
	故$P_a(\hat{X}^*) \leq \exp(-Cn)$
  的结论成立。
\end{proof}
\section{定理\ref{thm:error_rate} 的证明}
在本节我们首先给出获得\eqref{eq:hatX_double_prime}
和\eqref{eq:PG_energy}式 的推导过程。

由能量函数的表达式\eqref{eq:energy},
我们有:
\begin{equation}
	H(\bar{\sigma}) := (\gamma \frac{\log n}{n} + 1)\sum_{\{i,j\}\not\in E(G)} \delta(\bar{\sigma}_i, \bar{\sigma}_j)
	- \sum_{1<i<j<n} \delta(\bar{\sigma}_i, \bar{\sigma}_j)
\end{equation}
注意到$\sum_{1<i<j<n} \delta(\bar{\sigma}_i, \bar{\sigma}_j)=\frac{k}{2}\frac{n}{k}(\frac{n}{k}-1)$
是一个与$\bar{sigma}$ 无关的常数。因此极小化
$H(\bar{\sigma}) $ 等价于极小化
\eqref{eq:hatX_double_prime}式。

另外,由 \eqref{eq:GmL} 式,
对数似然函数可写成
\begin{align*}
  \log P_G(Z=z|X=\sigma)
  &= \log p \sum_{\sigma_i = \sigma_j} z_{ij}
  +\log q \sum_{\sigma_i \neq \sigma_j}z_{ij}\\
  &+\log (1-p) \sum_{\sigma_i = \sigma_j}(1-z_{ij})
  +\log (1-q) \sum_{\sigma_i \neq \sigma_j}(1-z_{ij})\\
  &= \log p \sum_{\{i,j\}\in E(G)}
  \delta(\sigma_i, \sigma_j)
  +\log q \sum_{\{i,j\}\in E(G)}(1-\delta(\sigma_i, \sigma_j))\\
  &+\log (1-p) \sum_{\{i,j\}\not\in E(G)}
  \delta(\sigma_i, \sigma_j)
  +\log (1-q) \sum_{\{i,j\}\not\in E(G)}
  (1-\delta(\sigma_i, \sigma_j))\\
  &=\log \frac{a}{b} \sum_{\{i,j\}\in E(G)}
  \delta(\sigma_i, \sigma_j) + (\log(1-p)-\log(1-q))
  \sum_{\{i,j\}\not \in E(G)}
  \delta(\sigma_i, \sigma_j)+ C
\end{align*}
从而得到\eqref{eq:PG_energy}式。

定理 \ref{thm:error_rate}的证明依赖下述引理:
\begin{lemma}\label{lem:mZW}
	令 $m$ 为一正整数,
	$Z_1, \dots, Z_m$
  是独立同分布的 Bernoulli($\B$),
   $W_1, \dots, W_m$ 是 i.i.d Bernoulli($\A$) ,
  并且与 $Z_1, \dots, Z_m$独立,
	则我们有:
\begin{equation}
	P(\sum_{i=1}^m (Z_i  - W_i) \geq 0) \leq \exp(-\frac{m \log n}{n}(\sqrt{a} - \sqrt{b})^2)
	\end{equation}
\end{lemma}
引理 \ref{lem:mZW} 的证明与引理\ref{lem:enhanced_fb}
的证明类似,均为先计算
$\E[\exp(s \sum_{i=1}^m (Z_i - W_i))]
=\exp(m \log(1-\B + \B e^s) + m \log(1-\A + \A e^{-s}))$。这里
我们直接用 $\log(1+x)\leq x$的不等式进行放缩,
可以避免出现量阶的估计项。即
$\E[\exp(s \sum_{i=1}^m (Z_i - W_i))]
\leq \exp(m(1-\B + \B e^s - \A + \A e^{-s}))$。
再取$s$使得 $be^s=ae^{-s}$。
最后再应用切尔诺夫不等式引理\ref{lem:mZW}即可得证。

\begin{proof}[Proof of Theorem \ref{thm:error_rate}]

	令 $P_F^{(r)}$ 表示
  下述事件发生的概率:
  存在 $\sigma$ 同时满足
  $\Dist(\sigma, X) = r$ 且
  $H(\sigma) < H(X)$。
	
	由 \eqref{eq:energy_diff} 式,当
  $\sigma$ 与 $X$ 仅在一个坐标上不同时,
  由 引理 \ref{lem:mZW},
  $H(\sigma) < H(X)$ 发生的概率为
	$P_G(A_r^s - A_r^0 > 0) \leq n^{-\frac{(\sqrt{a}-\sqrt{b})^2}{k}}$。
  因此,$P_F^{(1)}  \leq (k-1)n^{g(\bar{\beta})}$。
	我们由 $H(\bar{\sigma}) < H(X)$ 可推出 $A_{\bar{\sigma}} < B_{\bar{\sigma}}$,进而由引理 \ref{lem:enhanced_fb}
  可得,
  对于 $ r \leq \frac{n}{\sqrt{\log n}}$
  我们有
  $P_F^{(r)} \leq (k-1)^r n^{rg(\bar{\beta})}$。
	对于
  $ r \geq \frac{n}{\sqrt{\log n}}$, 在引理
  \ref{lem:sigmaX} 中取 $C=0$,我们可得
  $\sum_{r\geq \frac{n}{\sqrt{\log n}}}P_F^{(r)} \leq e^{-n}$。
	$\sum_{r\leq \frac{n}{\sqrt{\log n}}}P_F^{(r)}$有一个多项式的量阶上界,
  而其他部分的上界是指数衰减的,因此
  关于$\hat{X}'$ 总的误差上界可以写成:
  \begin{align*}
	P_F = & \sum_{r=1}^n P_F^{(r)} \leq (1+o(1)) \sum_{r=1}^{\infty} (k-1)^r n^{rg(\bar{\beta})}\\
	\leq & (1+o(1))\frac{(k-1) n^{g(\bar{\beta})}}{1-(k-1) n^{g(\bar{\beta})}} = (k-1+o(1))n^{g(\bar{\beta})}
	\end{align*}   
	
当 $\sigma \in W^*$, 由\eqref{eq:number_of_sigma_r_equal_omega_i}式
 我们有 $I'_i = I_i$。
由 引理\ref{lem:minus}可得
$|B_{\bar{\sigma}}| = |A_{\bar{\sigma}}|$
且 $N_{\bar{\sigma}} = 0$。
由 \eqref{eq:Hgeneral}式,
$H(\bar{\sigma}) < H(X)$ 等价于
$B_{\bar{\sigma}} > A_{\bar{\sigma}}$。

Therefore, when $ \Dist(\bar{\sigma}, X) \geq \frac{n}{\sqrt{\log n} }$ and $D(\sigma, X)$,
from Equation \eqref{eq:Asigma}, we have\linebreak $A_{\bar{\sigma}} \geq \frac{n^2}{k^2(k-1)\sqrt{\log n} } (1+o(1))$.
We use Lemma \ref{lem:mZW} by letting $m=|A_{\bar{\sigma}}|$ when $m \geq \frac{n}{ \sqrt{\log n}}$; the~error term is bounded
by $\sum_{r\geq \frac{n}{ \sqrt{\log n}}} P_F^{(r)} \leq \sum_{r\geq \frac{n}{ \sqrt{\log n}}} (k-1)^r \exp(-\frac{(\sqrt{a} - \sqrt{b})^2}{k^2(k-1)} n \sqrt{\log n}))
\leq \exp(-n)$, which decreases exponentially fast.
For $m < \frac{n}{ \sqrt{\log n}}$, we can use Lemma \ref{lem:enhanced_fb} directly 
by considering $\sum_{r=2}^{\frac{n}{ \sqrt{\log n}}} P_F^{(r)}$. The~summation starts from $r=2$ since $\sigma \in W^*$.
Therefore,
\begin{align*}
P_F = & \sum_{r=2}^n P_F^{(r)} \leq (1+o(1)) \sum_{r=2}^{\infty} (k-1)^r n^{rg(\bar{\beta})}\\
\leq & ((k-1)^2+o(1))n^{2g(\bar{\beta})}
\end{align*}
\end{proof}