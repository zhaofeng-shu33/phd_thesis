\chapter{附录}
\section{数据集}


\subsection{PSP算法性能比较所用数据集}
Gaussian-blob 数据集与 Gaussian 数据集
的区别在于每类$n$个数据点,其中$n=[1,2,3,4,5]*100$。

Two-level graph 数据集中取参数
$z_{\mathrm{in}_1}=s^2-2,
z_{\mathrm{in}_2}=s-1,
z_{\mathrm{out}}=1$,其中$s=[3,4,5,6,7]$。
\subsection{异常值检测所用数据集}
GaussianBlob 数据集 共有 300 个点,
其中 有 15\% 的点采样自 $[-6,6]\times [-6,6]$
内的均匀分布,为异常值点。其余的点采样自标准差为0.5,
中心在原点的 二维高斯分布, 为正常值点。

Moon 数据集类似 GaussianBlob 数据集, 共有 345 个点,
其中 有45 个点采样自 $[-6,6]\times [-6,6]$,经过个别点的坐标修正
后视为异常值点,
其余的点用 scikit-learn 的 make\_moons 函数生成,并做适当的
伸缩变换后视为正常值点。

Lymphography 数据集共148个数据点,
分成4类,因为第1和第4类总共有6个数据点,被视为异常值点,
其余的点为正常值点。

Glass 数据集在数据聚类中用到过,这里因为其第6个类别的点数量最少,
视其为整体的异常值点,而其他数据点为正常值点。

Ionosphere 数据集经常预处理后共 351个数据点,特征的维度为32,
其中 126个为异常值点。

\section{算法超参数}

\subsection{社团的层次化发现算法}
针对社团的层次化发现场景,信息聚类和 GN 算法均是不含
超参的。
BHCD 算法的超参数较多,
我们取 $\delta=0.2, \lambda=0.85$,其他参数取自文章\citet{RN23}
中的默认值。
\subsection{异常值检测算法}
除信息检测算法外,我们使用 scikit-learn 算法包提供
的接口,除接口默认的参数外,各算法的超参数如表 \ref{tab:outlier_detection_alg_hyperparameter}
所示。信息检测算法根据式\ref{eq:outlier_detection_proposal}介绍的规则自动判断异常值点的数量,
其他算法则根据已知的异常值点的数量对异常值进行检测。

\begin{table}[!ht]
  \begin{adjustbox}{width=\columnwidth,center}
  \begin{tabular}{|l|c|c|c|c|c|c|}
  \hline
   异常值检测算法  &   参数名称 &
   GaussianBlob &  Moon &  Lymphography &   Glass &   Ionosphere \\
  \hline
   \multirow{2}{*}{信息聚类}  & affinity
   & rbf & rbf & laplacian
   & rbf & rbf \\
   \cline{2-7}
   & $\gamma$ & 0.5 & 0.8 & 0.1 & 0.05 & 0.29 \\
   \hline
   局部异常因子  &  
   邻居数量 & 20 &  10
   &  60
   & 2 &   9 \\
   \hline
   \multirow{2}{*}{一类支持向量机}  & 
   affinity & rbf &  rbf
   & rbf
   & rbf & rbf \\
   \cline{2-7}
   & $\gamma$ & 0.1 & 0.1 & 0.1 & 0.1 & 0.4\\
  \hline
  \end{tabular}
\end{adjustbox}
\caption{异常值检测算法超参数}\label{tab:outlier_detection_alg_hyperparameter}
\end{table}
