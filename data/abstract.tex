% !TeX root = ../thuthesis-example.tex

% 中英文摘要和关键字

\begin{abstract}
  随着大数据时代的来临,图结构的数据越来越多地应用在
  网络科学、信息科学和生命科学等多个领域。
  社群发现作为自动推断社群结构的一种无监督学习方法,
  其在刻画社群性质、洞见社群相互关系等方面发挥了重要的作用。
  然而,社群发现的算法设计和参数调节多以经验和启发的方式进行,
  在消耗大量计算资源的同时算法的性能无法得到保障。有鉴于此,
  本文基于信息论的度量提出了社群发现平均损失和精确恢复的性能指标,
  并基于指标优化
  的思路设计了新的社群发现算法,同时揭示了与已有算法的联系。

  具体而言,通过研究信息聚类这一的理论指标的性质、我们优化了已有的基于平均损失设计的社群发现算法,使其时间
  复杂度降低了一个数量级。该优化后的算法不仅可以解决社群的层次发现问题,通过我们的进一步探索,
  也可用于解决异常值检测等领域的问题。另一方面,我们针对具有多个社群结构的随机块模型
  研究了一类随机算法实现精确恢复的条件,得出了这类随机算法具有的相变性质。
  在极限情形下,我们挖掘了该条件与最大似然算法、最大模块度算法以及模拟退火算法
  的联系。从而在理论层面上印证了这些算法的有效性。
  

  此外,针对社群发现实际应用中节点存在额外信息的场景,我们建立了合理的
  概率统计模型对其开展深入研究。在理论层面上我们导出了该模型的误差衰减速率,
  而在算法设计层面上我们基于半正定规划设计了可以实现精确恢复的高效算法,
  并通过仿真实验验证了算法的有效性。

  总体而言,本文从理论层面上揭示了社群发现算法可达的最优恢复率,
  籍此指导算法设计和进一步优化算法性能,对在实际场景下用社群发现解决问题具有
  重要参考价值和借鉴意义。
  % 关键词用“英文逗号”分隔,输出时会自动处理为正确的分隔符
  \thusetup{
    keywords = {社群发现, 信息聚类, 随机块模型, 精确恢复, 额外信息},
  }
\end{abstract}

\begin{abstract*}
  An abstract of a dissertation is a summary and extraction of research work and contributions.
  Included in an abstract should be description of research topic and research objective, brief introduction to methodology and research process, and summary of conclusion and contributions of the research.
  An abstract should be characterized by independence and clarity and carry identical information with the dissertation.
  It should be such that the general idea and major contributions of the dissertation are conveyed without reading the dissertation.

  An abstract should be concise and to the point.
  It is a misunderstanding to make an abstract an outline of the dissertation and words “the first chapter”, “the second chapter” and the like should be avoided in the abstract.

  Keywords are terms used in a dissertation for indexing, reflecting core information of the dissertation.
  An abstract may contain a maximum of 5 keywords, with semi-colons used in between to separate one another.

  % Use comma as separator when inputting
  \thusetup{
    keywords* = {Community Detection, Info-Clustering, Stochastic Block Model, Exact Recovery, Side Information},
  }
\end{abstract*}
