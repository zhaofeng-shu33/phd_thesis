% !TeX root = ../thuthesis-example.tex

% 中英文摘要和关键字

\begin{abstract}
  随着大数据时代的来临,网络化的数据越来越多地应用在
  社会科学、信息科学和生命科学等多个领域。
  社团发现作为自动推断网络中社团结构的一种无监督学习方法,
  其在刻画社团性质、洞见社团相互关系等方面发挥了重要的作用。
  然而,社团发现的算法设计和参数调节多以经验和启发的方式进行,
  在消耗大量计算资源的同时,算法的性能无法得到保障。有鉴于此,
  本文从信息论度量的角度入手,对社团发现问题进行理论建模并对算法进行性能优化。
  我们在改进已有算法效率的同时,也获得了有理论误差收敛率保证的社团发现算法,为社团发现领域的研究开辟了新的思路。

  本文的工作贡献可总结为以下三个部分:

  首先,基于多变量互信息这一信息论度量,
  我们建立了对弱独立的随机变量进行聚类与计算图上的主划分序列问题的联系。
  进一步地,我们利用多变量互信息优化了计算图上的主划分序列的算法,
  其时间复杂度相比现有计算主划分序列的方法降低了一个数量级。
  该优化后的算法不仅可以解决社团的层次化发现问题,
  也可用于数据聚类、异常值检测等领域。
  
  另一方面,基于速率函数,我们针对具有多个社团结构的随机块模型
 研究了随机采样的玻茨模型算法实现精确恢复的条件及其误差率,得出了该算法的相变性质,
 为算法超参的选取提供了理论依据。
  在极限情形下,我们进一步挖掘了该算法与最大似然算法、最大模块度算法以及最小割问题
  的联系,从而在理论层面上建立了分析社团发现常用优化指标的统一框架。
  

  此外,基于雷尼散度,针对社团发现实际应用中节点存在辅助信息的场景,我们建立了合理的
  概率模型对其开展深入研究。在理论层面上,
  我们导出了达到精确恢复所需的样本数量和该模型的最优误差衰减速率;
  而在算法设计层面上,我们基于半正定规划设计了可以实现精确恢复的高效算法,
  并通过仿真实验验证了算法的有效性。

  总体而言,本文使用了多变量互信息、速率函数以及雷尼散度等信息论度量,
  揭示了
  极限情况下社团发现算法可达的最优恢复率,为算法的检测效果提供了理论保证。
  并籍此指导算法设计和进一步优化算法性能,节约计算资源,对于实际网络中的社团发现问题有
 重要参考价值和借鉴意义。

  % 关键词用“英文逗号”分隔,输出时会自动处理为正确的分隔符
  \thusetup{
    keywords = {社团发现, 随机块模型, 多变量互信息, 速率函数, 雷尼散度},
  }
\end{abstract}

\begin{abstract*}
  With the advent of the era of big data, graph-structured data is increasingly used in
  social science, information science, life science, and other fields.
  Community discovery is an unsupervised learning method for automatically inferring community structure
  in complex networks.
  It has played an important role in describing the community and gaining insight into the relationship between different communities.
  However, the algorithm design and parameter fine-tuning of community discovery are mostly carried out in an empirical way.
  While consuming a large number of computing resources,
  the performance of the algorithm cannot be guaranteed.
  In view of this,
  we use information theoretic metrics to tackle the theoretical modeling and performance optimization of community discovery algorithms.
  While improving the efficiency of existing algorithms, we obtain community discovery algorithms with theoretical error rate bounds,
  which opens up new ideas for research in the field of community discovery.

  The contribution of this thesis can be summarized in the following three parts:

  Firstly, based on the multivariate mutual information,
  we establish the connection between clustering weakly independent random variables and computing the principal sequence of partition
  of a graph.
  Further, we use multivariate mutual information to optimize the algorithm for calculating the principal sequence of partition of a graph.
  Compared with existing methods for calculating the principal sequence of partition,
  the time complexity is reduced by an order of magnitude.
  The optimized algorithm can not only solve the problem of hierarchical community discovery, but can also be used in data clustering, outlier detection and other fields.

  On the other hand, using the rate function,
  we study the exact recovery condition and the error rate of a random sampling algorithm based on Potts model
  for the stochastic block model with multiple community structures.
  We obtain the phase transition property of the algorithm and provide a theoretical basis
  for the choice of hyper-parameters.
  In the asymptotic case, we further explore the connection between our method and other algorithms, including the maximum likelihood algorithm, the maximum modularity algorithm and the minimum cut optimization.
  In the process we theoretically establish a unified framework for analyzing common object functions for community discovery.

  In addition, based on Rényi divergence,
  we establish a reasonable probabilistic model
  to conduct in-depth research on the scenario
  where the nodes of community have side information.
  From the perspective of theoretical study,
  for the stochastic block model with side information
  we derive the required sample size to achieve exact recovery and optimal error decay rate.
  On the algorithm design level,
  an efficient algorithm is proposed, which can achieve exact recovery based on positive semi-definite programming.
  The effectiveness of the algorithm is verified through numerical experiments.

  Generally speaking, this paper uses information theoretical metrics
  such as multivariate mutual information,
  rate function, and Rényi divergence to reveal the asymptotic
  optimal recovery rate achievable by the community discovery algorithms,
  and to provide theoretical guarantees for the detection accuracy.
  Further, our analysis can guide the algorithm design and optimization process,
  which helps to save the computing resources and provide salutary reference
  for community detection methods in practical scenarios.
  % Use comma as separator when inputting
  \thusetup{
    keywords* = {Community Detection, Info-Clustering, Stochastic Block Model, Exact Recovery, Side Information},
  }
\end{abstract*}
