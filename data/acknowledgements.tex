% !TeX root = ../thuthesis-example.tex

\begin{acknowledgements}
  首先,衷心感谢导师张林教授和指导老师黄绍伦教授对我的悉心指导,
  张老师以前瞻性的眼光为我的博士课题指明了方向,而黄老师在具体细节上给予了我耐心、
  细致的建议。他们在学术道路上勇于拼搏进取的精神、敏锐的洞察力和为人处事的高风亮节
  对我产生了深刻的影响,也将继续指引我接下来的人生道路。

  感谢李阳老师、叶旻老师在课题上给予我的指导与帮助:
  李老师 为我的论文研究工作提供了具有深刻启发性的建议,而叶老师则以其严谨的治学态度为我的研究课题提供了可行的思路。

 感谢TBSI的丁文伯老师、电子工程系的王钺老师、沈渊老师 、董宇涵老师为我的论文选题和学术报告提供的建议,另外还要感谢夏树涛、廖庆敏、张旭东等老师在专业课程课程方面给予我的引导,为我开展后续研究打下了坚实的基础。
 
 感谢与加州理工学院的司马晋同学的合作解决了我研究课题中的瓶颈问题,
 司马晋学长学术修养和人格魅力是我学习的榜样。
 
 感谢师兄徐祥祥给予我学术论文写作、研究思路等方面的指导,
 徐师兄焚膏继晷,兀兀穷年的治学精神令我十分敬仰。
 
感谢实验室同窗马飞、武智源在实验设计与实现方面给予我的支持,
以及童鑫熠、王伟达在理论推导和论文表达细节的改进建议。

感谢实验室的其他同学给予我的帮助,他们包括
李名扬、李国栋、徐健、刘宁、刘心宇、张跃、陈翔宇、
王彦龙、余新春、宋子午、李宛达、张安平、
彭天任、傅好好等。

感谢我的室友赵鹏阳以及其他朋友们:
雷晓宇、白杨、王瑜、赵原芳、孔德强、姜伟峰、张宇、
张晓冰、黄少平、牛行知、蔡建锋、秦荧瑢、周安、黄冬瑜、朱慧、
陈祥、赵雨欣等。

感谢实验室秘书李胜男老师、田冰子老师、电子系教务张静一老师、
已离职的齐韬老师和周放老师等对我的帮助。

感谢我的父母和其他亲属在我的博士工作期间给予我的关怀,正是他们的鼓励和支持使我免除了
后顾之忧,得以专心于博士研究工作。

最后,我还要感谢清华大学及其下属的电子工程系和深圳研究生院
为我提供的学习、科研环境。
\end{acknowledgements}
