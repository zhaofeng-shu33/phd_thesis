% !TeX root = ../thuthesis-example.tex

\begin{resolution}
    在网络科学中,社团发现是其研究的重点问题之一,在社会工程、生物化学和计算机科学等领域均有重要应用。论文作者从信息论度量的角度入手,对社团发现问题进行了深入研究,具有重要理论意义。
论文工作的主要创新点如下:
\begin{enumerate}
    \item 提出了一种基于多变量互信息的社团层次化发现算法,与现有的求解主划分序列算法相比,其时间复杂度具有较大程度降低;
    \item 针对随机块模型,研究了基于玻茨模型的算法精确恢复条件与误差衰减速率,为算法分析提供了统一的框架;
    \item 在有辅助信息的社团发现问题上,基于雷尼散度,给出了随机块模型实现精确恢复的充要条件,并导出了最优误差衰减率的理论极限。
\end{enumerate}

论文工作表明作者在信息与通信工程学科具有坚实宽广的基础理论和系统深入的专门知识,具有独立从事科学研究工作的能力。
论文写作条理清楚、逻辑严密、图表格式规范,答辩叙述清楚、回答问题正确。
经答辩委员会5人无记名投票表决,一致同意通过论文答辩,并建议授予赵丰工学博士学位。



\end{resolution}
