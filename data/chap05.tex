\chapter{有额外信息的随机块模型}
在本章中,我们首先引入我们研究的有额外信息的随机块模型
的定义。然后研究它的精确恢复的条件以及实现其精确恢复的半正定规划算法。

\section{精确恢复条件}
\newacronym{acr:sbmsi}{SBMSI}{Stochastic Block Model with side information}
带有额外信息的随机块模型(\gls{acr:sbmsi})
是 \ref{sec:sbm} 节介绍的
SSBM 模型的推广。这里我们在2个社群的随机块模型的
基础上引入额外信息。
除了图 $Z$ 和节点标签 $X$ 外,
第 $i$  个节点 ($1\leq i \leq n$) 还含有 $m=\lceil \gamma \log n \rceil $ 个相关的数据样本 $Y^{i}_{j}$, $j\in \{1,\ldots,m\}$。若 $X_i=1$
这些样本 i.i.d. 采样自分布 $P_0$,若  $X_i=-1$ 则采样自 $P_1$。
注意到 给定 标签 $X_i$,
对于 $i\in\{1,\ldots,n\}$,数据样本 $Y^{i}_{j}$, $j\in \{1,\ldots,m\}$ 与 $\{Z_{i,j}\}_{1\le i<j\le n}$ 独立。
 因此,在标签 $X$ 给定的情况下,
  $(\{Z_{i,j}\}_{1\le i<j\le n},\{Y^i_{j}\}_{1\le i\le n,1\le j\le m})$ 的联合分布为  
\begin{align}\label{eq:lh}
    &P(y=\{y^i_{j}\}_{1\le i\le n,1\le j\le m},z=\{z_{i,j}\}_{1\le i<j\le n}| (x_1,\ldots,x_n)) \nonumber\\
    &= \prod_{1\le i,j\le n}P(z_{i,j}|x_i,x_j)\prod_{i=1}^n \prod_{j=1}^m P(y^i_j|x_i), 
\end{align}
其中 
\begin{equation*}
    P  (z_{i,j}=1|x_i,x_j) = \begin{cases}
        p & \text{if } x_i = x_j \\
        q & \text{if } x_i\ne x_j
    \end{cases},
\end{equation*}
并且
\begin{equation*}
    P(y^i_j|x_i) = \begin{cases}
        P_0(y^i_j) & x_i = 1 \\
        P_1(y^i_j) & x_i = -1
    \end{cases}
\end{equation*}
 $P(\{x^i_{j}\}_{1\le i\le n,1\le j\le m},\{z_{i,j}\}_{1\le i<j\le n}| y_1,\ldots,y_n)$ 
 的条件分布 依赖于
 $n,m,p, q, P_0$ 和 $P_1$ 这些参数,因此我们把 SBMSI 写成 $\SBMSI(n,m,p,q,P_0,P_1)$ 的形式。
 给定由 $\SBMSI(n,m,p,q,P_0,P_1)$ 生成的图$Z$和数据$Y$, 我们的目标是恢复 出$X$。
 
\section{误差误差速率}
