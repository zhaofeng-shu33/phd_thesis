% !TeX root = ../thuthesis-example.tex

\chapter{信息聚类理论与算法}
数据聚类方法与社群发现方法有着天然的联系。

\section{社群发现场景下的多变量互信息理论}
在这一节中,我们首先把\ref{sec:local_geometry}节介绍
的弱独立的概念推广到多个随机变量弱独立,
并在多个随机变量弱独立的条件下,对
\ref{sec:info_clustering}节介绍的多变量互信息
进行化简。
\begin{definition}\label{def:general}
  称$Z_1, \dots, Z_n (n\geq 2)$
  是$\epsilon$-弱独立的,如果存在一个随机变量 $U$
  使得
  $(Z_1, \dots, Z_n)|U$的PMF 在 $(Z_1, \dots, Z_n)$的
  $\sqrt[n]{\epsilon}$邻域内并且
  $Z_1, \dots Z_n$关于
  $U$的条件分布是独立的。
  \end{definition}
\begin{example}\label{ex:xy_weak_ext}
    考虑例 \ref{ex:Pweak_1}给出的分布,给定$X,Y$
    的分布为$P(x,y)=\frac{1}{4}(1+\epsilon(-1)^{x+y})$。
    我们构造一分布$U$如表所示:
    \begin{table}
      \begin{tabular}{|c|c|c|c|c|}
        \hline
        $P(U=i|X=j, Y=k)$ & $j=0,k=0$ &
        $j=0,k=1$ & $j=1,k=0$  & $j=1,k=1$ \\
        \hline
        $i=0$ & $\frac{(1-\sqrt{\epsilon})^2}{2(1+\epsilon)}$
        & $\frac{1}{2}$ & $\frac{1}{2}$ 
        &  $\frac{(1+\sqrt{\epsilon})^2}{2(1+\epsilon)}$\\
        \hline
        $i=1$ & $\frac{(1+\sqrt{\epsilon})^2}{2(1+\epsilon)}$
        & $\frac{1}{2}$ & $\frac{1}{2}$
        & $\frac{(1-\sqrt{\epsilon})^2}{2(1+\epsilon)}$
        \\
        \hline
      \end{tabular}
    \end{table}

    不难验证$P(X=j, Y=k | U=i)=P(X=j | U=i)
      P(Y=k| U=i)$。此外, $(X, Y)|U=0$
      的PMF 可以写成向量的形式
      $\frac{1}{4}[(1-\sqrt{\epsilon})^2,
      1-\epsilon,
      1-\epsilon,
      (1+\sqrt{\epsilon})^2]$,它在
      $(X,Y)$的PMF 向量
      $\frac{1}{4}[1+\epsilon, 1-\epsilon, 1-\epsilon, 1+\epsilon]$
      的 $\sqrt{\epsilon}$-邻域内。因此
      $X,Y$是$\epsilon$-弱独立的。
\end{example}
例\ref{ex:xy_weak_ext}展示了定义\ref{def:general}和
定义\ref{def:weak_indepedent}在两变量情形的一个共用的例子。
事实上,我们可以证明定义\ref{def:general}是
定义\ref{def:weak_indepedent}的拓展。
\begin{theorem}\label{thm:weak_independence_equivalent}
  如果 对于任何 $x \in \mathcal{X}$,
$Y$关于$X$的条件 PMF 
$P_{Y|X}(\cdot |x)$在$Y$的$\epsilon$邻域内,那么存在
一个随机变量 $U$
  使得
  $(X, Y)|U$的PMF 在 $(X, Y)$的
  $\sqrt{\epsilon}$邻域内并且
  $X, \dots Y$关于
  $U$的条件分布是独立的。
\end{theorem}
\begin{proof}
  由$\epsilon$邻域的定义式\ref{def:eps_neighborhood},
  $P_{Y|X=x}(y) = P_Y(y) + \sqrt{P_Y(y)}\phi_{Y|X=x}(y)
  \epsilon$。定义$\phi_{XY}(x,y)=\sqrt{P_X(x)}\phi_{Y|X=x}(y)$
  则有
  $P_{XY}(x,y) = P_X(x)P_Y(y) + \sqrt{P_X(x)}\sqrt{P_Y(y)}\phi_{XY}(x,y)
  \epsilon$。
  假设$x\in \mathcal{X}$ 且 $y\in \mathcal{Y}$,
  $\mathcal{X}, \mathcal{Y}$ 均为有限的字母集。
  $\phi_{XY}$ 是 $|\mathcal{X}| \times |\mathcal{Y}|$
  的矩阵,其秩为 $r$。可以通过$SVG$分解为
  $\frac{1}{r}\sum_{i=1}^r \phi_i \psi^T_i$,其中$\phi, \psi$
  分别为长度为$|\mathcal{X}|, |\mathcal{Y}|$的
  列向量。构造 $U$ 是$\{1, 2, \dots, 2r\}$ 上面的均匀分布。
  $Z_1, Z_2$ 做如下构造:
  \begin{align*}
    P(Z_1=x|U=i) &= P_X(x) + (-1)^i\sqrt{P_X(x)}\phi_{\lceil i/2 \rceil}(x) \sqrt{\epsilon}, x \in \mathcal{X} \\
    P(Z_2=y|U=i) &= P_Y(y) + (-1)^i\sqrt{P_Y(y)}\psi_{\lceil i/2 \rceil}(y) \sqrt{\epsilon}, y \in \mathcal{Y}\\
  \end{align*}
  $Z_1 | U$ 与$Z_2 | U$ 独立,因此,
  $Z_1, Z_2$ 的联合分布为:
  \begin{align*}
  P(Z_1=x, Z_2=y)& =\sum_{i=1}^{2r}P(U=i)P(Z_1=x|U=i)P(Z_2=y|U=i)\\
  &=P_X(x)P_Y(y) + \sqrt{P_X(x)}\sqrt{P_Y(y)}\frac{\epsilon}{r}
  \sum_{i=1}^{r}\phi_i(x)
  \psi(y) =P(X=x,Y=y)
  \end{align*}
  因此,$Z_1, Z_2$与$X,Y$具有相同的分布。
  不难验证$(Z_1, Z_2)|U$在$(Z_1, Z_2)$的$\sqrt{\epsilon}$邻域内,
  故结论得证。
  \end{proof}
  定理\ref{thm:weak_independence_equivalent}的证明提供了一种生成
  满足弱独立条件的随机变量的方法。即给定均匀分布
  $U$ 在 $n! \times r$个点上取值,
  并且$Z_i|U$在分布$Z_i$的
  $\sqrt[n]{\epsilon}$ 邻域内:
  \begin{equation}
    P(Z_j=z|U=i) = P_{Z_j}(z) + 
    (-1)^{i \,\mathrm{mod}\, n!}\sqrt{P_{Z_j(z)}}
    \phi_{\lceil\, i/n!\, \rceil}(z) \sqrt[n]{\epsilon}, z \in \mathcal{Z_j}
  \end{equation}
  假设$Z_1|U, \dots, Z_n|U$ 独立,可得到
  $(Z_1, \dots, Z_n)|U$的分布。
  不然验证通过这种方法构造出来的分布$Z_1, \dots, Z_n$是弱独立的。

与\ref{sec:info_clustering}节介绍的 PIN 模型类似,
在多个随机变量弱独立的条件下,KL散度的计算可以
与图结构进行对应。即有如下定理:
\begin{theorem}\label{thm:DPX}
若 $Z_1, \dots, Z_n$ $\epsilon$-弱独立, 则有
\begin{equation}\label{eq:PXV}
D(P_{Z_V} || \prod_{C\in \P} P_{Z_{C}}) = {1 \over 2}
\sum_{\substack{(i,j) \not\in C\\ C\in \P}} \norm{B_{ij}}_F^2 + o(\epsilon^2)
\end{equation}
其中 $B_{ij}$ 是 随机变量  $Z_i$ 和 $Z_j$
之间的$B$ 矩阵(参见式\ref{eq:Ixy})而 $\P$是$V$的一个分割(
参见式\ref{eq:IPZV})。 
\end{theorem}
若$ n = 2$,定理\ref{thm:DPX} 即是用 $B$ 矩阵估计互信息,
与 式 \eqref{eq:Ixy} 相同。
因此,定理\ref{thm:DPX}可看成式 \eqref{eq:Ixy} 
的拓展。

定理 \ref{thm:DPX} 是关于弱独立的随机变量的。
现在我们把它拓展到针对数据样本。
给定一 $K$ 个聚类簇的数据集,一共有 $n$ 个样本。
每个 聚类簇被看成一个随机变量。
假设第 $i$ 个 聚类簇 $Z_i$ 字母集 为 $\abs{\mathcal{Z}_i}$,
全局约束是 $\sum_{i=1}^K \abs{\mathcal{Z}_i} = n$。
假设$Z_1, \dots, Z_K$弱独立,
由弱独立的定义式 \ref{def:general}, $Z_i$ 和 $Z_j$ $\epsilon$-弱独立 ($i\neq j$),
于是我们有
\begin{equation}\label{eq:phi_w}
P_{Z_i Z_j}(z_i, z_j) = P_{Z_i}(z_i)P_{Z_j}(z_j) + \epsilon \sqrt{P_{Z_i}(z_i)P_{Z_j}(z_j)} \phi_{Z_i Z_j}(z_i, z_j) + o(\epsilon)
\end{equation}
因此,由 \eqref{eq:Ixy}式,
 $\norm{B_{ij}}_F^2 = \epsilon^2 \sum_{z_r \in \mathcal{Z}_i, z_s \in \mathcal{Z}_j} \phi^2_{Z_i Z_j}(z_r, z_s)$ 
 并且 式 \eqref{eq:PXV} 可以展开成
\begin{equation}\label{eq:PXV_Data}
D(P_{Z_V} || \prod_{C\in \P} P_{Z_{C}}) =
{\epsilon^2\over 2}\sum_{\substack{(i,j) \not\in C\\ C\in \P}}
\sum_{z_r \in \mathcal{Z}_i, z_s \in \mathcal{Z}_j}  \phi^2_{Z_i Z_j}(z_r, z_s) + o(\epsilon^2)
\end{equation}
我们可以把
$\phi^2_{Z_i Z_j}(z_r, z_s)$ 这一项
当成一个有$n$个节点的图的
边的权值。
图的每一个节点对应一个数据点。
为简化符号, 令 $w_{rs} = \phi^2_{Z_{d(z_r)}Z_{d(z_s)}}(z_r, z_s)$
\footnote{当 $d(z_r) = d(z_s)$ 时,
我们仍可以形式化的定义 $w_{rs}$ 为远大于$\max\{\phi^2_{Z_i Z_j}(z_r, z_s), i\neq j\}$
的值。},
其中,$d(z_r)$ 把节点映射到它所属的随机变量的序号, 定义域 为 $1\leq r,s \leq \abs{V}$。
我们也可以展开 每个 $Z_i$ 到它的节点集 并且将分割 $\P$
看成是对节点集的分割。
假设我们考虑的图 $G(V, E)$ 是有向的
\footnote{有向图的假设可以减少计算量而不失一般性},
类似 式 \ref{eq:IP} 我们定义图的入割函数 (in-cut function) $f(C)$ for $C\subseteq V$ as $f(C) = \sum_{i\not\in C, j\in C, (i,j) \in E} w_{ij}$,
它是所有进入$C$的有向边的权值之和。
基于上面定义的符号,表示KL散度的式 \eqref{eq:PXV_Data} 可以写成聚类簇之间边的权值之和
的形式:
\begin{equation}\label{eq:PXV_Data_Simplified}
D(P_{Z_V} || \prod_{C\in \P} P_{Z_{C}}) = \epsilon^2 \sum_{C \in \P} f(C)+ o(\epsilon^2)
\end{equation}
从而我们获得了在多变量弱独立条件下数据聚类的表达式。
由于$\epsilon$是给定的无穷小量,我们更关心$\epsilon^2$
的系数,该系数即与式\eqref{eq:IP}中的$f[\P]$一样,
也在我们求解等价的优化问题式\eqref{eq:hlambda}中出现过。
\section{与数据聚类问题的联系}
\label{sec:data_clustering}
社群发现是输入一张图获得节点所属的类别,而
数据聚类的任务与社群发现相同,但其输入是一个数据矩阵,
其中行数是数据的个数而
每一行的向量代表该数据的特征。二者可以通过
输入数据的格式转换来实现算法互通。一般而言,
从数据矩阵到图的变换是可以通过选取一
相似度度量$d(\cdot,\cdot)$作用到数据上
获得图中边的权值,即$w_{ij}=d(x_i, x_j)$。
在这一节中,
我们研究的重点是如何用
\eqref{eq:hlambda} 实现数据聚类的任务。
在永野清仁的文章中\cite{mac},已经有用RBF核作为相似度度量的
尝试,但局限于该相似度度量在小规模真实世界数据集上表
现不佳,此外相似度度量本身具有一些超参数,也会影响聚类
效果。我们通过使用交叉验证、网格搜索,及对
不同相似度度量的枚举,试图寻找特定问题下具有
较好表现的相似度度量及其超参数。

我们使用5种数据集对基于图分割的数据聚类方法
进行测试,各数据集的基本情况如表\ref{tab:clustering_dataset}
所示。
\begin{table}[!ht]
  \centering
  \begin{tabular}{|c|c|c|c|}
    \toprule
    名称 & 样本数量 & 类别数量 & 特征维度 \\
    \midrule
    Gaussian & 100 & 4 & 2 \\
    Circle & 300 & 3 & 2 \\
    Iris & 150 & 3 & 4 \\
    Glass & 214 & 6 & 9 \\
    Libras & 360 & 15 & 90 \\
    \bottomrule
  \end{tabular}
  \caption{数据聚类测试数据}\label{tab:clustering_dataset}
\end{table}
其中,Gaussian 和 Circle 是人工生成的数据集。
Iris, Glass 和 Libras 是来自 UCI 机器学习标准数据集
\cite{Dua:2019}。

首先我们展示在两个人工生成的数据集上的聚类效果,
Gaussian 数据集的相似度度量取 式\ref{eq:rbf_kernel} 所示的 RBF 核,
$\gamma=0.6$。
\begin{equation}\label{eq:rbf_kernel}
  w_{ij}=\exp(-\gamma ||x_i-x_j||^2)
\end{equation}
Circle 数据集的相似度度量通过 k近邻 进行计算,即仅与自己距离最近的$k$
个数据点有边相连\footnote{所有边权值均为1},取$k=7$。
聚类效果如图 \ref{fig:artificial_dataset_effect} 所示。

\begin{figure}[!ht]
  \begin{subfigure}[b]{\linewidth}
  \includegraphics[width=\textwidth]{4part.pdf}
  \caption{有4个类别的 Gaussian 数据集}
  \label{fig:4p}
\end{subfigure}
\begin{subfigure}[b]{\linewidth}
  \includegraphics[width=\textwidth]{3circle.pdf}
  \caption{有3个类别的 Circle 数据集}
  \label{fig:3c}
\end{subfigure}
\caption{人工生成的数据集聚类效果}
\label{fig:artificial_dataset_effect}
\end{figure}

从图中可以看到,通过选取适当的阈值 $\lambda$,
可获得与数据的真实类别十分接近的聚类结果,对应于图
\ref{fig:4p} 中的 $\lambda=0.42$
和图\ref{fig:3c} 中的 $\lambda=0.0$。

\section{社群的层次发现算法}
\label{subsec:cd}
尽管在发现社群的单层结构方面
有很多算法取得了成功,自动发现
复杂网络中的社群层次结构仍是一个困难的问题。
在本节中,我们用实验来证明PSP算法
在恰当定义“边的权重”后,可以成功恢复两层图的层次结构。

对于一个无权图,如果我们简单地把
所有边的权值赋成1,
我们只能得到平凡的聚类树,也即聚类树
只有根节点和叶节点。这个论断可以从如下
定理中得到:
\begin{theorem}\label{thm:triangle}
  在图$G$中,对于没有边相连的两个节点 $w_{ij}=0$。
  并且对于任意三元组$i,j,k \in V$ 权值满足三角不等式 
  $w_{ij} + w_{jk} \geq w_{ki}$,
  则该图的聚类树是平凡的。
\end{theorem}
  
我们通过实验发现使用下述的方法对无权图进行重新
赋权可以达到较理想的效果:
\begin{equation}\label{eq:wij_scheme}
    w_{ij} = 1 + \abs{\{k | (i,k),(j,k) \in E \}} \textrm{ for } (i,j) \in E
\end{equation}
式 \ref{eq:wij_scheme} 对$w_{ij}$
的赋权方法是计算从$i$到$j$不超过两跳的路径数量。

下面我们在一个具有两层结构的图上验证我们的
赋权方法。该数据集在文献
\cite{RN22} 中用于研究动态网络的同步行为,
其结构如图 \ref{fig:c1} 所示。 

\begin{figure}
	\centering
	\begin{subfigure}{0.45\textwidth}
		\includegraphics[width=\textwidth]{two_level.pdf}
		\caption{拓扑结构,参数取值为$z_{\mathrm{in}_1} = 14,$ $z_{\mathrm{in}_2} = 3, z_{\mathrm{out}}=1$.}\label{fig:c1}
	\end{subfigure}
	\begin{subfigure}{0.45\textwidth}
		\includegraphics[width=\textwidth]{tree_info-clustering.pdf}
		\caption{聚类树结构,可用PSP算法完全恢复出来。每个叶节点仍包含16个子节点未画出。}
    \label{fig:c2}
	\end{subfigure}
	\caption{具有两层结构的图示意}
\end{figure}

该两层结构的图在宏观层面它包含有 4个 社群,
而每一个中等规模的社群在微观层面又包含4个小社群,
每个小社群有16个节点。
我们用 $z_{\mathrm{in}_1}$ 表示
在小社群内部每个节点连接的边的平均数量;
用 $z_{\mathrm{in}_2}$ 表示
每个大社群内部的中等规模社群之间相互连接的边的平均数量;
用 $z_{\mathrm{out}}$ 表示
不同大社群之间相互连接的边的平均数量。
通过变化参数设定 $\{z_{\mathrm{in}_1}, z_{\mathrm{in}_2}, z_{\mathrm{out}} \}$
我们能得到不同疏密程度的图。

To compare the deduced hierarchical tree structure with the ground truth, we use the normalized Robinson-Foulds distance as the metric. It describes the distance between two trees with different topology and falls within $[0,1]$. We compare the RF distance of graph-based info-clustering with that of GN (Girvan-Newman algorithm) and BHCD (a Bayesian Hierarchical method in \cite{RN23}). The result is shown in Fig. \ref{fig:cdr}. As can be seen from Fig.\ref{fig:cdr}, in either case, graph-based info-clustering can produce more similar tree structures than that of the ground truth.

\section{基于图分割的社群发现算法及其改进}
在本节中,我们首先介绍我们对PSP算法的改进,并在一定
的假设下证明其理论复杂度,最后比较求解图分割的三种不同实现下
的算法效率。
\subsection{HPSP算法}
我们改进的PSP算法仍将使用
算法 \ref{alg:psp} 中提到的
\texttt{DT} 子函数。
在算法\ref{alg:psp} 中,我们观察到
每次调用 \texttt{DT} 是相互独立的,
并没有利用到聚类树的 固有性质。
如果聚类树的嵌套性质可以被考虑进来的话, 
我们可以在后面的计算中
仅在子图上进行 \texttt{DT} 的调用,
从而可以实现计算效率比PSP算法提升一个数量级。


具体来说,
假设我们第一次调用DT后得到了
$P_i = \{C_1, \dots, C_t\}$。
然后我们 分别对每个
$C_i(i=1,\dots, t)$
计算 PSP分割。
并从子图的计算中构造 $P_j(j>i)$。
对于 $P_j(j<i)$,
我们可以把 图 $G$
缩成 图 $G^t$,其中 每个$C_i$集合中的点缩成一个单独的节点。
因此图 $G^t$共有 $t$ 个节点。
通过在$G^t$ 上调用 DT 函数我们可以得到  $P_j(j<i)$。
我们把我们的改进方法称为 HPSP,其中 H 表示 “层次的”
英文首字母\footnote{hierarchical},
其描述在算法\ref{alg:psp_i_simplified}中给出。

\begin{algorithm}[!ht]
	\caption{改进的求解主分割序列的算法}\label{alg:psp_i_simplified}
	\begin{algorithmic}[1]
		\REQUIRE 有向图 $G(V, E)$; 边的权值函数 $w(e)$,其中 $e\in E$
		\ENSURE 聚类树 $\mathcal{T}(K, E)$ 其中 $K \subseteq 2^{V}$ 表示节点集
    而 $E$ 表示边的集合。
		\STATE 初始化 树 $\mathcal{T}$:
     $V$ 是根节点,
     $\{j\}(j \in V)$ 是诸叶节点,
     无其他节点。
		\STATE \texttt{Split}($G, V$)
		\FUNCTION{\texttt{Split}($\widetilde{G}, \widetilde{V}$)}
		\STATE $w$ 是 $\widetilde{G}$ 所有边的权值之和。
		\STATE $\gamma' = \frac{w}{\abs{V(\widetilde{G})}-1}$
    其中 $V(\widetilde{G})$ 是 图$\widetilde{G}$
    的节点集。
    \label{alg:gamma_apostrophe}
		\STATE $(\tilde{h}, P') = \texttt{DT}(\widetilde{G}, \gamma')$ 其中
    $\P'$ 是在 式 \eqref{eq:hlambda} 中达到
    最小值 $h(\gamma')$的分割
    并且 $\tilde{h}$ 是对应的最小值。 \label{line:DT}
		\IF{$\tilde{h} = - \gamma'$}
		\STATE 在聚类树
    $\mathcal{T}$ 中把权值 $\gamma'$ 标记在从 $\widetilde{V}$ 出发
    到其所有子节点的边上。
		\ELSE
		\FOR{$S$ in $P'$ and $\abs{S}>1$}
		\STATE 在聚类树
    $\mathcal{T}$ 中构造新的节点$S$,并修改 $\widetilde{V}$的子节点的父亲节点 为$S$,
    而$S$的父节点为$\widetilde{V}$。
		\STATE \texttt{Split}($\widetilde{G}[S], S$)
    其中 $\widetilde{G}[S]$ 是 $\widetilde{G}$ 限制在 $S$
    上的子图。\label{line:SplitDown}
		\STATE 在 $\widetilde{G}$ 中将 $S$ 缩成一个节点。 % graph \widetilde{G} is modified
		\ENDFOR 
		\STATE \texttt{Split}($\widetilde{G}, \widetilde{V}$)		\label{line:SplitUp}
		\ENDIF
		\ENDFUNCTION
	\end{algorithmic}
\end{algorithm}

\begin{figure}[!ht]
	\centering
	\includegraphics[width=10cm]{alg_illustration.pdf}
	\caption{HPSP 求解图(a)的主分割序列,聚类树经(b), (c) 演化到 (d) }\label{fig:alg_eg}
\end{figure}

\begin{example}
	我们用一个简单的例子来说明
  算法 \ref{alg:psp_i_simplified}。
  考虑图 $G(V, E)$,有 $V=\{1,2,3,4\}, E=\{(1,2),(1,3),(2,3),(1,4),(2,4)\}$。
  边的权值为 $w_{13}=2, w_{12}=5, w_{23}=2, w_{14}=1, w_{24}=1$。
	图 \ref{fig:alg_eg} (a) 中画出了 $G$
  的结构。
  最初的聚类树结构如图 \ref{fig:alg_eg} (b)
  所示。
  根据 
  算法 \ref{alg:psp_i_simplified} 的
  第\ref {alg:gamma_apostrophe}行,
  我们计算 $\gamma' = \frac{11}{4-1}, \tilde{h} = -\frac{16}{3} < -\gamma' $
  且 $\P' = \{\{1,2,3\},\{4\}\}$,
  于是我们得到如图\ref{fig:alg_eg} (c) 所示
  的聚类树结构。
	
	然后我们在 子图 $G[\{1,2,3\}]$ 上运行 PSP 算法,
  $\gamma' = \frac{9}{2}, \tilde{h} = -5 < -\gamma'$
  并且 $\P= \{\{1,2\},\{3\}\}$。 
  于是我们得到了最终的聚类树结构。
  其余的计算给出了聚类树剩余的边的权值,如
  图 \ref{fig:alg_eg} (d) 所示。
\end{example}	
\subsection{不同PSP算法的效率比较}
在本节中,我们将比较 HPSP 算法和PSP算法
(算法\ref{alg:psp})
以及基于参数最大流改进方法\cite{kolmogorov}
的运行效率。我们使用两种不同的数据集,图的边具有不同的
稠密程度。

第一个数据集叫做 Gaussian-blobs,
类似\ref{sec:data_clustering}
节的 Gaussian 数据集,它也是由
4个相同大小的高斯块组成,但我们可以改变
每个块的样本数量。由于我们使用式\ref{eq:rbf_kernel}
计算图中边的权值,边的数量满足 $|E|=\Theta(|V|^2)$。
The second dataset is a two-level graph with $s^3$ nodes. The dataset with $s=4$ is the same with that used in Section \ref{subsec:cd}.

We can construct a graph from the first dataset and the edge weight is computed with the rbf kernel. The characteristics of these two datasets are summarized in
Table \ref{tab:alg_compare}.
\begin{table}[!ht]
\centering
\begin{tabular}{ccc}
\hline
name & weight type & density \\
\hline
Gaussian-blobs & int & $|E|=O(|V|^2)$\\
Two-level graph & float & $|E|=O(|V|^{3/2})$\\
\hline
\end{tabular}
\caption{Dataset properties used in algorithm speed comparison} \label{tab:alg_compare}
\end{table}
We use CPU times to measure the time complexity of the algorithms. By varying the number of nodes, we can get the experiment results, shown in Figure \ref{fig:esc}.
As can be seen, our implementation is much faster than previous ones.

\begin{figure}
	\centering
	\begin{subfigure}{0.45\textwidth}
		\includegraphics[width=\textwidth]{2019-08-26-gaussian.pdf}
	\end{subfigure}
	\begin{subfigure}{0.45\textwidth}
		\includegraphics[width=\textwidth]{2019-09-19-two_level.pdf}
	\end{subfigure}
	\caption{Empirical speed comparison of different PSP algorithms}\label{fig:esc}
\end{figure}

We notice that parametric flow based PSP \citep{kolmogorov}  does not perform well in practice. This may be related with the extra cost to construct intermediate graph structures and maintain concurrent threads, whose time overhead can not be neglectable in implementation.

\section{基于图分割的社群发现算法在异常值检测领域的应用}

\section{代码实现}
尽管最大流算法包比较多,但我们尚未发现
求解层次分割(式\eqref{eq:PSP_structure})的开源实现。
因此我们使用 C++ 实现了跨平台的PSP算法\ref{alg:psp}
(内嵌迪尔沃思截断算法\ref{alg:dt}),并且
我们提供了Python编程语言的接口函数,并在 \url{pypi.org}
平台上以 pspartition
的名字发布,以供后面的研究者使用。
我们的实现使用 CMake 编译,依赖于第三方库 LEMON \cite{dezsHo2011lemon}。 
LEMON 库在我们的算法包中用于构建所需的图数据结构和最大流算法。
这里顺便提及的一点是 LEMON 库里使用的
最大流算法是基于最大标签选择策略的前置推送-标签重贴算法。
根据文献\citet{ahuja1997computational}的比较,该算法
的效率比起其他解最大流问题的算法有明显的优势。

在Python的接口函数方面,我们采用了CPython
跨语言编程的方式,将由 NetworkX \cite{SciPyProceedings_11} 构建的图
转换成用边表示的图传递到C++中的类构造函数中,
计算完成后再把C++中的集合转换成Python中的列表
进行返回。用于求解例 \ref{ex:psp}
的示例 Python 代码如下。
\begin{lstlisting}[language=Python]
  from pspartition import PsPartition
  a = [[0,1,1], [0,2,5], [1,2,1]] # a graph
  p = PsPartition(3, a) # 3 nodes
  p.run()  
  cv = p.get_critical_values()
  pl = p.get_partitions()
  print(cv)
  print(pl)
  \end{lstlisting}
\section{实验结果}


中文论文的数学符号默认遵循 GB/T 3102.11—1993《物理科学和技术中使用的数学符号》
\footnote{原 GB 3102.11—1993,自 2017 年 3 月 23 日起,该标准转为推荐性标准。}。
该标准参照采纳 ISO 31-11:1992 \footnote{目前已更新为 ISO 80000-2:2019。},
但是与 \TeX{} 默认的美国数学学会(AMS)的符号习惯有所区别。
具体地来说主要有以下差异:
\begin{enumerate}
  \item 大写希腊字母默认为斜体,如
    \begin{equation*}
      \Gamma \Delta \Theta \Lambda \Xi \Pi \Sigma \Upsilon \Phi \Psi \Omega.
    \end{equation*}
    注意有限增量符号 $\increment$ 固定使用正体,模板提供了 \cs{increment} 命令。
  \item 小于等于号和大于等于号使用倾斜的字形 $\le$、$\ge$。
  \item 积分号使用正体,比如 $\int$、$\oint$。
  \item 行间公式积分号的上下限位于积分号的上下两端,比如
    \begin{equation*}
      \int_a^b f(x) \dif x.
    \end{equation*}
    行内公式为了版面的美观,统一居右侧,如 $\int_a^b f(x) \dif x$ 。
  \item
    偏微分符号 $\partial$ 使用正体。
  \item
    省略号 \cs{dots} 按照中文的习惯固定居中,比如
    \begin{equation*}
      1, 2, \dots, n \quad 1 + 2 + \dots + n.
    \end{equation*}
  \item
    实部 $\Re$ 和虚部 $\Im$ 的字体使用罗马体。
\end{enumerate}

以上数学符号样式的差异可以在模板中统一设置。
另外国标还有一些与 AMS 不同的符号使用习惯,需要用户在写作时进行处理:
\begin{enumerate}
  \item 数学常数和特殊函数名用正体,如
    \begin{equation*}
      \uppi = 3.14\dots; \quad
      \symup{i}^2 = -1; \quad
      \symup{e} = \lim_{n \to \infty} \left( 1 + \frac{1}{n} \right)^n.
    \end{equation*}
  \item 微分号使用正体,比如 $\dif y / \dif x$。
  \item 向量、矩阵和张量用粗斜体(\cs{symbf}),如 $\symbf{x}$、$\symbf{\Sigma}$、$\symbfsf{T}$。
  \item 自然对数用 $\ln x$ 不用 $\log x$。
\end{enumerate}


英文论文的数学符号使用 \TeX{} 默认的样式。
如果有必要,也可以通过设置 \verb|math-style| 选择数学符号样式。

关于量和单位推荐使用
\href{http://mirrors.ctan.org/macros/latex/contrib/siunitx/siunitx.pdf}{\pkg{siunitx}}
宏包,
可以方便地处理希腊字母以及数字与单位之间的空白,
比如:
\SI{6.4e6}{m},
\SI{9}{\micro\meter},
\si{kg.m.s^{-1}},
\SIrange{10}{20}{\degreeCelsius}。



\section{数学公式}

数学公式可以使用 \env{equation} 和 \env{equation*} 环境。
注意数学公式的引用应前后带括号,建议使用 \cs{eqref} 命令,比如式\eqref{eq:example}。
\begin{equation}
  \frac{1}{2 \uppi \symup{i}} \int_\gamma f = \sum_{k=1}^m n(\gamma; a_k) \mathscr{R}(f; a_k)
  \label{eq:example}
\end{equation}
注意公式编号的引用应含有圆括号,可以使用 \cs{eqref} 命令。

多行公式尽可能在“=”处对齐,推荐使用 \env{align} 环境。
\begin{align}
  a & = b + c + d + e \\
    & = f + g
\end{align}



\section{数学定理}

定理环境的格式可以使用 \pkg{amsthm} 或者 \pkg{ntheorem} 宏包配置。
用户在导言区载入这两者之一后,模板会自动配置 \env{thoerem}、\env{proof} 等环境。

\begin{theorem}[Lindeberg--Lévy 中心极限定理]
  设随机变量 $X_1, X_2, \dots, X_n$ 独立同分布, 且具有期望 $\mu$ 和有限的方差 $\sigma^2 \ne 0$,
  记 $\bar{X}_n = \frac{1}{n} \sum_{i+1}^n X_i$,则
  \begin{equation}
    \lim_{n \to \infty} P \left(\frac{\sqrt{n} \left( \bar{X}_n - \mu \right)}{\sigma} \le z \right) = \Phi(z),
  \end{equation}
  其中 $\Phi(z)$ 是标准正态分布的分布函数。
\end{theorem}
\begin{proof}
  Trivial.
\end{proof}

同时模板还提供了 \env{assumption}、\env{definition}、\env{proposition}、
\env{lemma}、\env{theorem}、\env{axiom}、\env{corollary}、\env{exercise}、
\env{example}、\env{remar}、\env{problem}、\env{conjecture} 这些相关的环境。
