% !TeX root = ../thuthesis-example.tex

\chapter{信息聚类理论与算法}
数据聚类方法与社群发现方法有着天然的联系。

\section{社群发现场景下的多变量互信息理论}
在这一节中,我们首先把\ref{sec:local_geometry}节介绍
的弱独立的概念推广到多个随机变量弱独立,
并在多个随机变量弱独立的条件下,对
\ref{sec:info_clustering}节介绍的多变量互信息
进行化简。
\begin{definition}\label{def:general}
  称$Z_1, \dots, Z_n (n\geq 2)$
  是$\epsilon$-弱独立的,如果存在一个随机变量 $U$
  使得
  $(Z_1, \dots, Z_n)|U$的PMF 在 $(Z_1, \dots, Z_n)$的
  $\sqrt[n]{\epsilon}$邻域内并且
  $Z_1, \dots Z_n$关于
  $U$的条件分布是独立的。
  \end{definition}
\begin{example}\label{ex:xy_weak_ext}
    考虑例 \ref{ex:Pweak_1}给出的分布,给定$X,Y$
    的分布为$P(x,y)=\frac{1}{4}(1+\epsilon(-1)^{x+y})$。
    我们构造一分布$U$如表所示:
    \begin{table}
      \begin{tabular}{|c|c|c|c|c|}
        \hline
        $P(U=i|X=j, Y=k)$ & $j=0,k=0$ &
        $j=0,k=1$ & $j=1,k=0$  & $j=1,k=1$ \\
        \hline
        $i=0$ & $\frac{(1-\sqrt{\epsilon})^2}{2(1+\epsilon)}$
        & $\frac{1}{2}$ & $\frac{1}{2}$ 
        &  $\frac{(1+\sqrt{\epsilon})^2}{2(1+\epsilon)}$\\
        \hline
        $i=1$ & $\frac{(1+\sqrt{\epsilon})^2}{2(1+\epsilon)}$
        & $\frac{1}{2}$ & $\frac{1}{2}$
        & $\frac{(1-\sqrt{\epsilon})^2}{2(1+\epsilon)}$
        \\
        \hline
      \end{tabular}
    \end{table}

    不难验证$P(X=j, Y=k | U=i)=P(X=j | U=i)
      P(Y=k| U=i)$。此外, $(X, Y)|U=0$
      的PMF 可以写成向量的形式
      $\frac{1}{4}[(1-\sqrt{\epsilon})^2,
      1-\epsilon,
      1-\epsilon,
      (1+\sqrt{\epsilon})^2]$,它在
      $(X,Y)$的PMF 向量
      $\frac{1}{4}[1+\epsilon, 1-\epsilon, 1-\epsilon, 1+\epsilon]$
      的 $\sqrt{\epsilon}$-邻域内。因此
      $X,Y$是$\epsilon$-弱独立的。
\end{example}
例\ref{ex:xy_weak_ext}展示了定义\ref{def:general}和
定义\ref{def:weak_indepedent}在两变量情形的一个共用的例子。
事实上,我们可以证明定义\ref{def:general}是
定义\ref{def:weak_indepedent}的拓展。
\begin{theorem}\label{thm:weak_independence_equivalent}
  如果 对于任何 $x \in \mathcal{X}$,
$Y$关于$X$的条件 PMF 
$P_{Y|X}(\cdot |x)$在$Y$的$\epsilon$邻域内,那么存在
一个随机变量 $U$
  使得
  $(X, Y)|U$的PMF 在 $(X, Y)$的
  $\sqrt{\epsilon}$邻域内并且
  $X, \dots Y$关于
  $U$的条件分布是独立的。
\end{theorem}
\begin{proof}
  由$\epsilon$邻域的定义式\ref{def:eps_neighborhood},
  $P_{Y|X=x}(y) = P_Y(y) + \sqrt{P_Y(y)}\phi_{Y|X=x}(y)
  \epsilon$。定义$\phi_{XY}(x,y)=\sqrt{P_X(x)}\phi_{Y|X=x}(y)$
  则有
  $P_{XY}(x,y) = P_X(x)P_Y(y) + \sqrt{P_X(x)}\sqrt{P_Y(y)}\phi_{XY}(x,y)
  \epsilon$。
  假设$x\in \mathcal{X}$ 且 $y\in \mathcal{Y}$,
  $\mathcal{X}, \mathcal{Y}$ 均为有限的字母集。
  $\phi_{XY}$ 是 $|\mathcal{X}| \times |\mathcal{Y}|$
  的矩阵,其秩为 $r$。可以通过$SVG$分解为
  $\frac{1}{r}\sum_{i=1}^r \phi_i \psi^T_i$,其中$\phi, \psi$
  分别为长度为$|\mathcal{X}|, |\mathcal{Y}|$的
  列向量。构造 $U$ 是$\{1, 2, \dots, 2r\}$ 上面的均匀分布。
  $Z_1, Z_2$ 做如下构造:
  \begin{align*}
    P(Z_1=x|U=i) &= P_X(x) + (-1)^i\sqrt{P_X(x)}\phi_{\lceil i/2 \rceil}(x) \sqrt{\epsilon}, x \in \mathcal{X} \\
    P(Z_2=y|U=i) &= P_Y(y) + (-1)^i\sqrt{P_Y(y)}\psi_{\lceil i/2 \rceil}(y) \sqrt{\epsilon}, y \in \mathcal{Y}\\
  \end{align*}
  $Z_1 | U$ 与$Z_2 | U$ 独立,因此,
  $Z_1, Z_2$ 的联合分布为:
  \begin{align*}
  P(Z_1=x, Z_2=y)& =\sum_{i=1}^{2r}P(U=i)P(Z_1=x|U=i)P(Z_2=y|U=i)\\
  &=P_X(x)P_Y(y) + \sqrt{P_X(x)}\sqrt{P_Y(y)}\frac{\epsilon}{r}
  \sum_{i=1}^{r}\phi_i(x)
  \psi(y) =P(X=x,Y=y)
  \end{align*}
  因此,$Z_1, Z_2$与$X,Y$具有相同的分布。
  不难验证$(Z_1, Z_2)|U$在$(Z_1, Z_2)$的$\sqrt{\epsilon}$邻域内,
  故结论得证。
  \end{proof}
  定理\ref{thm:weak_independence_equivalent}的证明提供了一种生成
  满足弱独立条件的随机变量的方法。即给定均匀分布
  $U$ 在 $n! \times r$个点上取值,
  并且$Z_i|U$在分布$Z_i$的
  $\sqrt[n]{\epsilon}$ 邻域内:
  \begin{equation}
    P(Z_j=z|U=i) = P_{Z_j}(z) + 
    (-1)^{i \,\mathrm{mod}\, n!}\sqrt{P_{Z_j(z)}}
    \phi_{\lceil\, i/n!\, \rceil}(z) \sqrt[n]{\epsilon}, z \in \mathcal{Z_j}
  \end{equation}
  假设$Z_1|U, \dots, Z_n|U$ 独立,可得到
  $(Z_1, \dots, Z_n)|U$的分布。
  不然验证通过这种方法构造出来的分布$Z_1, \dots, Z_n$是弱独立的。

与\ref{sec:info_clustering}节介绍的 PIN 模型类似,
在多个随机变量弱独立的条件下,KL散度的计算可以
与图结构进行对应。即有如下定理:
\begin{theorem}\label{thm:DPX}
若 $Z_1, \dots, Z_n$ $\epsilon$-弱独立, 则有
\begin{equation}\label{eq:PXV}
D(P_{Z_V} || \prod_{C\in \P} P_{Z_{C}}) = {1 \over 2}
\sum_{\substack{(i,j) \not\in C\\ C\in \P}} \norm{B_{ij}}_F^2 + o(\epsilon^2)
\end{equation}
其中 $B_{ij}$ 是 随机变量  $Z_i$ 和 $Z_j$
之间的$B$ 矩阵(参见式\ref{eq:Ixy})而 $\P$是$V$的一个分割(
参见式\ref{eq:IPZV})。 
\end{theorem}
若$ n = 2$,定理\ref{thm:DPX} 即是用 $B$ 矩阵估计互信息,
与 式 \eqref{eq:Ixy} 相同。
因此,定理\ref{thm:DPX}可看成式 \eqref{eq:Ixy} 
的拓展。

定理 \ref{thm:DPX} 是关于弱独立的随机变量的。
现在我们把它拓展到针对数据样本。
给定一 $K$ 个聚类簇的数据集,一共有 $n$ 个样本。
每个 聚类簇被看成一个随机变量。
假设第 $i$ 个 聚类簇 $Z_i$ 字母集 为 $\abs{\mathcal{Z}_i}$,
全局约束是 $\sum_{i=1}^K \abs{\mathcal{Z}_i} = n$。
假设$Z_1, \dots, Z_K$弱独立,
由弱独立的定义式 \ref{def:general}, $Z_i$ 和 $Z_j$ $\epsilon$-弱独立 ($i\neq j$),
于是我们有
\begin{equation}\label{eq:phi_w}
P_{Z_i Z_j}(z_i, z_j) = P_{Z_i}(z_i)P_{Z_j}(z_j) + \epsilon \sqrt{P_{Z_i}(z_i)P_{Z_j}(z_j)} \phi_{Z_i Z_j}(z_i, z_j) + o(\epsilon)
\end{equation}
因此,由 \eqref{eq:Ixy}式,
 $\norm{B_{ij}}_F^2 = \epsilon^2 \sum_{z_r \in \mathcal{Z}_i, z_s \in \mathcal{Z}_j} \phi^2_{Z_i Z_j}(z_r, z_s)$ 
 并且 式 \eqref{eq:PXV} 可以展开成
\begin{equation}\label{eq:PXV_Data}
D(P_{Z_V} || \prod_{C\in \P} P_{Z_{C}}) =
{\epsilon^2\over 2}\sum_{\substack{(i,j) \not\in C\\ C\in \P}}
\sum_{z_r \in \mathcal{Z}_i, z_s \in \mathcal{Z}_j}  \phi^2_{Z_i Z_j}(z_r, z_s) + o(\epsilon^2)
\end{equation}
We can treat the term
$\phi^2_{Z_i Z_j}(z_r, z_s)$ as the weight of an edge for a graph with $n$ nodes.
Each vertex of the graph is corresponding to a data sample.
To simplify the notation, let $w_{rs} = \phi^2_{Z_{d(z_r)}Z_{d(z_s)}}(z_r, z_s)$ \footnote{When $d(z_r) = d(z_s)$, we can still formally define $w_{rs}$ to be some value larger than $\phi^2_{Z_i Z_j}(z_r, z_s)$ for $i\neq j$.} where $d(z_r)$ maps the vertex to its underlining random variable index, $1\leq r,s \leq \abs{V}$.
We can also expand each $Z_i$ to its alphabet and treat $\P$ as the partition of graph vertices.
Suppose the graph $G(V, E)$ we considered is directed
\footnote{Without loss of generality we only consider directly graph in this article to reduce computational cost in algorithm part},
we can define the graph in-cut function $f(C)$ for $C\subseteq V$ as $f(C) = \sum_{i\not\in C, j\in C, (i,j) \in E} w_{ij}$, which is the total weight of edges coming
into $C$.
From Equation \eqref{eq:PXV_Data}, we can characterize the KL divergence by the total inter-cluster edge weight as
\begin{equation}\label{eq:PXV_Data_Simplified}
D(P_{Z_V} || \prod_{C\in \P} P_{Z_{C}}) = \epsilon^2 \sum_{C \in \P} f(C)+ o(\epsilon^2)
\end{equation}
Notice that for directed graph, each edge weight is counted only once in the summation. Therefore there is no need to divide 2 in
Equation \eqref{eq:PXV_Data_Simplified}.
Combining Equation \eqref{eq:IZV} and \eqref{eq:PXV_Data_Simplified} we get the expression for the second order coefficient of multivariate mutual information (MMI)

\section{基于图分割的社群发现算法及其改进}

\section{基于图分割的社群发现算法在异常值检测领域的应用}

\section{代码实现}
尽管最大流算法包比较多,但我们尚未发现
求解层次分割(式\eqref{eq:PSP_structure})的开源实现。
因此我们使用 C++ 实现了跨平台的PSP算法\ref{alg:psp}
(内嵌迪尔沃思截断算法\ref{alg:dt}),并且
我们提供了Python编程语言的接口函数,并在 \url{pypi.org}
平台上以 pspartition
的名字发布,以供后面的研究者使用。
我们的实现使用 CMake 编译,依赖于第三方库 LEMON \cite{dezsHo2011lemon}。 
LEMON 库在我们的算法包中用于构建所需的图数据结构和最大流算法。
这里顺便提及的一点是 LEMON 库里使用的
最大流算法是基于最大标签选择策略的前置推送-标签重贴算法。
根据文献\citet{ahuja1997computational}的比较,该算法
的效率比起其他解最大流问题的算法有明显的优势。

在Python的接口函数方面,我们采用了CPython
跨语言编程的方式,将由 NetworkX \cite{SciPyProceedings_11} 构建的图
转换成用边表示的图传递到C++中的类构造函数中,
计算完成后再把C++中的集合转换成Python中的列表
进行返回。用于求解例 \ref{ex:psp}
的示例 Python 代码如下。
\begin{lstlisting}[language=Python]
  from pspartition import PsPartition
  a = [[0,1,1], [0,2,5], [1,2,1]] # a graph
  p = PsPartition(3, a) # 3 nodes
  p.run()  
  cv = p.get_critical_values()
  pl = p.get_partitions()
  print(cv)
  print(pl)
  \end{lstlisting}
\section{实验结果}


中文论文的数学符号默认遵循 GB/T 3102.11—1993《物理科学和技术中使用的数学符号》
\footnote{原 GB 3102.11—1993,自 2017 年 3 月 23 日起,该标准转为推荐性标准。}。
该标准参照采纳 ISO 31-11:1992 \footnote{目前已更新为 ISO 80000-2:2019。},
但是与 \TeX{} 默认的美国数学学会(AMS)的符号习惯有所区别。
具体地来说主要有以下差异:
\begin{enumerate}
  \item 大写希腊字母默认为斜体,如
    \begin{equation*}
      \Gamma \Delta \Theta \Lambda \Xi \Pi \Sigma \Upsilon \Phi \Psi \Omega.
    \end{equation*}
    注意有限增量符号 $\increment$ 固定使用正体,模板提供了 \cs{increment} 命令。
  \item 小于等于号和大于等于号使用倾斜的字形 $\le$、$\ge$。
  \item 积分号使用正体,比如 $\int$、$\oint$。
  \item 行间公式积分号的上下限位于积分号的上下两端,比如
    \begin{equation*}
      \int_a^b f(x) \dif x.
    \end{equation*}
    行内公式为了版面的美观,统一居右侧,如 $\int_a^b f(x) \dif x$ 。
  \item
    偏微分符号 $\partial$ 使用正体。
  \item
    省略号 \cs{dots} 按照中文的习惯固定居中,比如
    \begin{equation*}
      1, 2, \dots, n \quad 1 + 2 + \dots + n.
    \end{equation*}
  \item
    实部 $\Re$ 和虚部 $\Im$ 的字体使用罗马体。
\end{enumerate}

以上数学符号样式的差异可以在模板中统一设置。
另外国标还有一些与 AMS 不同的符号使用习惯,需要用户在写作时进行处理:
\begin{enumerate}
  \item 数学常数和特殊函数名用正体,如
    \begin{equation*}
      \uppi = 3.14\dots; \quad
      \symup{i}^2 = -1; \quad
      \symup{e} = \lim_{n \to \infty} \left( 1 + \frac{1}{n} \right)^n.
    \end{equation*}
  \item 微分号使用正体,比如 $\dif y / \dif x$。
  \item 向量、矩阵和张量用粗斜体(\cs{symbf}),如 $\symbf{x}$、$\symbf{\Sigma}$、$\symbfsf{T}$。
  \item 自然对数用 $\ln x$ 不用 $\log x$。
\end{enumerate}


英文论文的数学符号使用 \TeX{} 默认的样式。
如果有必要,也可以通过设置 \verb|math-style| 选择数学符号样式。

关于量和单位推荐使用
\href{http://mirrors.ctan.org/macros/latex/contrib/siunitx/siunitx.pdf}{\pkg{siunitx}}
宏包,
可以方便地处理希腊字母以及数字与单位之间的空白,
比如:
\SI{6.4e6}{m},
\SI{9}{\micro\meter},
\si{kg.m.s^{-1}},
\SIrange{10}{20}{\degreeCelsius}。



\section{数学公式}

数学公式可以使用 \env{equation} 和 \env{equation*} 环境。
注意数学公式的引用应前后带括号,建议使用 \cs{eqref} 命令,比如式\eqref{eq:example}。
\begin{equation}
  \frac{1}{2 \uppi \symup{i}} \int_\gamma f = \sum_{k=1}^m n(\gamma; a_k) \mathscr{R}(f; a_k)
  \label{eq:example}
\end{equation}
注意公式编号的引用应含有圆括号,可以使用 \cs{eqref} 命令。

多行公式尽可能在“=”处对齐,推荐使用 \env{align} 环境。
\begin{align}
  a & = b + c + d + e \\
    & = f + g
\end{align}



\section{数学定理}

定理环境的格式可以使用 \pkg{amsthm} 或者 \pkg{ntheorem} 宏包配置。
用户在导言区载入这两者之一后,模板会自动配置 \env{thoerem}、\env{proof} 等环境。

\begin{theorem}[Lindeberg--Lévy 中心极限定理]
  设随机变量 $X_1, X_2, \dots, X_n$ 独立同分布, 且具有期望 $\mu$ 和有限的方差 $\sigma^2 \ne 0$,
  记 $\bar{X}_n = \frac{1}{n} \sum_{i+1}^n X_i$,则
  \begin{equation}
    \lim_{n \to \infty} P \left(\frac{\sqrt{n} \left( \bar{X}_n - \mu \right)}{\sigma} \le z \right) = \Phi(z),
  \end{equation}
  其中 $\Phi(z)$ 是标准正态分布的分布函数。
\end{theorem}
\begin{proof}
  Trivial.
\end{proof}

同时模板还提供了 \env{assumption}、\env{definition}、\env{proposition}、
\env{lemma}、\env{theorem}、\env{axiom}、\env{corollary}、\env{exercise}、
\env{example}、\env{remar}、\env{problem}、\env{conjecture} 这些相关的环境。
