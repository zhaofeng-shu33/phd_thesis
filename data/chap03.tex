% !TeX root = ../thuthesis-example.tex

\chapter{信息聚类理论与算法}
数据聚类方法与社群发现方法有着天然的联系。

\section{社群发现场景下的多变量互信息理论}
在这一节中,我们首先把\ref{sec:local_geometry}节介绍
的弱独立的概念推广到多个随机变量弱独立,
并在多个随机变量弱独立的条件下,对
\ref{sec:info_clustering}节介绍的多变量互信息
进行化简。
\begin{definition}\label{def:general}
  称$Z_1, \dots, Z_n (n\geq 2)$
  是$\epsilon$-弱独立的,如果存在一个随机变量 $U$
  使得
  $(Z_1, \dots, Z_n)|U$的PMF 在 $(Z_1, \dots, Z_n)$的
  $\sqrt[n]{\epsilon}$邻域内并且
  $Z_1, \dots Z_n$关于
  $U$的条件分布是独立的。
  \end{definition}
\begin{example}\label{ex:xy_weak_ext}
    考虑例 \ref{ex:Pweak_1}给出的分布,给定$X,Y$
    的分布为$P(x,y)=\frac{1}{4}(1+\epsilon(-1)^{x+y})$。
    我们构造一分布$U$如表所示:
    \begin{table}
      \begin{tabular}{|c|c|c|c|c|}
        \hline
        $P(U=i|X=j, Y=k)$ & $j=0,k=0$ &
        $j=0,k=1$ & $j=1,k=0$  & $j=1,k=1$ \\
        \hline
        $i=0$ & $\frac{(1-\sqrt{\epsilon})^2}{2(1+\epsilon)}$
        & $\frac{1}{2}$ & $\frac{1}{2}$ 
        &  $\frac{(1+\sqrt{\epsilon})^2}{2(1+\epsilon)}$\\
        \hline
        $i=1$ & $\frac{(1+\sqrt{\epsilon})^2}{2(1+\epsilon)}$
        & $\frac{1}{2}$ & $\frac{1}{2}$
        & $\frac{(1-\sqrt{\epsilon})^2}{2(1+\epsilon)}$
        \\
        \hline
      \end{tabular}
    \end{table}

    不难验证$P(X=j, Y=k | U=i)=P(X=j | U=i)
      P(Y=k| U=i)$。此外, $(X, Y)|U=0$
      的PMF 可以写成向量的形式
      $\frac{1}{4}[(1-\sqrt{\epsilon})^2,
      1-\epsilon,
      1-\epsilon,
      (1+\sqrt{\epsilon})^2]$,它在
      $(X,Y)$的PMF 向量
      $\frac{1}{4}[1+\epsilon, 1-\epsilon, 1-\epsilon, 1+\epsilon]$
      的 $\sqrt{\epsilon}$-邻域内。因此
      $X,Y$是$\epsilon$-弱独立的。
\end{example}
例\ref{ex:xy_weak_ext}展示了定义\ref{def:general}和
定义\ref{def:weak_indepedent}在两变量情形的一个共用的例子。
事实上,我们可以证明定义\ref{def:general}是
定义\ref{def:weak_indepedent}的拓展。
\begin{theorem}\label{thm:weak_independence_equivalent}
  如果 对于任何 $x \in \mathcal{X}$,
$Y$关于$X$的条件 PMF 
$P_{Y|X}(\cdot |x)$在$Y$的$\epsilon$邻域内,那么存在
一个随机变量 $U$
  使得
  $(X, Y)|U$的PMF 在 $(X, Y)$的
  $\sqrt{\epsilon}$邻域内并且
  $X, \dots Y$关于
  $U$的条件分布是独立的。
\end{theorem}
\begin{proof}
  由$\epsilon$邻域的定义式\ref{def:eps_neighborhood},
  $P_{Y|X=x}(y) = P_Y(y) + \sqrt{P_Y(y)}\phi_{Y|X=x}(y)
  \epsilon$。定义$\phi_{XY}(x,y)=\sqrt{P_X(x)}\phi_{Y|X=x}(y)$
  则有
  $P_{XY}(x,y) = P_X(x)P_Y(y) + \sqrt{P_X(x)}\sqrt{P_Y(y)}\phi_{XY}(x,y)
  \epsilon$。
  假设$x\in \mathcal{X}$ 且 $y\in \mathcal{Y}$,
  $\mathcal{X}, \mathcal{Y}$ 均为有限的字母集。
  $\phi_{XY}$ 是 $|\mathcal{X}| \times |\mathcal{Y}|$
  的矩阵,其秩为 $r$。可以通过$SVG$分解为
  $\frac{1}{r}\sum_{i=1}^r \phi_i \psi^T_i$,其中$\phi, \psi$
  分别为长度为$|\mathcal{X}|, |\mathcal{Y}|$的
  列向量。构造 $U$ 是$\{1, 2, \dots, 2r\}$ 上面的均匀分布。
  $Z_1, Z_2$ 做如下构造:
  \begin{align*}
    P(Z_1=x|U=i) &= P_X(x) + (-1)^i\sqrt{P_X(x)}\phi_{\lceil i/2 \rceil}(x) \sqrt{\epsilon}, x \in \mathcal{X} \\
    P(Z_2=y|U=i) &= P_Y(y) + (-1)^i\sqrt{P_Y(y)}\psi_{\lceil i/2 \rceil}(y) \sqrt{\epsilon}, y \in \mathcal{Y}\\
  \end{align*}
  $Z_1 | U$ 与$Z_2 | U$ 独立,因此,
  $Z_1, Z_2$ 的联合分布为:
  \begin{align*}
  P(Z_1=x, Z_2=y)& =\sum_{i=1}^{2r}P(U=i)P(Z_1=x|U=i)P(Z_2=y|U=i)\\
  &=P_X(x)P_Y(y) + \sqrt{P_X(x)}\sqrt{P_Y(y)}\frac{\epsilon}{r}
  \sum_{i=1}^{r}\phi_i(x)
  \psi(y) =P(X=x,Y=y)
  \end{align*}
  因此,$Z_1, Z_2$与$X,Y$具有相同的分布。
  不难验证$(Z_1, Z_2)|U$在$(Z_1, Z_2)$的$\sqrt{\epsilon}$邻域内,
  故结论得证。
  \end{proof}
  定理\ref{thm:weak_independence_equivalent}的证明提供了一种生成
  满足弱独立条件的随机变量的方法。即给定均匀分布
  $U$ 在 $n! \times r$个点上取值,
  并且$Z_i|U$在分布$Z_i$的
  $\sqrt[n]{\epsilon}$ 邻域内:
  \begin{equation}
    P(Z_j=z|U=i) = P_{Z_j}(z) + 
    (-1)^{i \,\mathrm{mod}\, n!}\sqrt{P_{Z_j(z)}}
    \phi_{\lceil\, i/n!\, \rceil}(z) \sqrt[n]{\epsilon}, z \in \mathcal{Z_j}
  \end{equation}
  假设$Z_1|U, \dots, Z_n|U$ 独立,可得到
  $(Z_1, \dots, Z_n)|U$的分布。
  不然验证通过这种方法构造出来的分布$Z_1, \dots, Z_n$是弱独立的。

与\ref{sec:info_clustering}节介绍的 PIN 模型类似,
在多个随机变量弱独立的条件下,KL散度的计算可以
与图结构进行对应。即有如下定理:
\begin{theorem}\label{thm:DPX}
若 $Z_1, \dots, Z_n$ $\epsilon$-弱独立, 则有
\begin{equation}\label{eq:PXV}
D(P_{Z_V} || \prod_{C\in \P} P_{Z_{C}}) = {1 \over 2}
\sum_{\substack{(i,j) \not\in C\\ C\in \P}} \norm{B_{ij}}_F^2 + o(\epsilon^2)
\end{equation}
其中 $B_{ij}$ 是 随机变量  $Z_i$ 和 $Z_j$
之间的$B$ 矩阵(参见式\ref{eq:Ixy})而 $\P$是$V$的一个分割(
参见式\ref{eq:IPZV})。 
\end{theorem}
若$ n = 2$,定理\ref{thm:DPX} 即是用 $B$ 矩阵估计互信息,
与 式 \eqref{eq:Ixy} 相同。
因此,定理\ref{thm:DPX}可看成式 \eqref{eq:Ixy} 
的拓展。

定理 \ref{thm:DPX} 是关于弱独立的随机变量的。
现在我们把它拓展到针对数据样本。
给定一 $K$ 个聚类簇的数据集,一共有 $n$ 个样本。
每个 聚类簇被看成一个随机变量。
假设第 $i$ 个 聚类簇 $Z_i$ 字母集 为 $\abs{\mathcal{Z}_i}$,
全局约束是 $\sum_{i=1}^K \abs{\mathcal{Z}_i} = n$。
假设$Z_1, \dots, Z_K$弱独立,
由弱独立的定义式 \ref{def:general}, $Z_i$ 和 $Z_j$ $\epsilon$-弱独立 ($i\neq j$),
于是我们有
\begin{equation}\label{eq:phi_w}
P_{Z_i Z_j}(z_i, z_j) = P_{Z_i}(z_i)P_{Z_j}(z_j) + \epsilon \sqrt{P_{Z_i}(z_i)P_{Z_j}(z_j)} \phi_{Z_i Z_j}(z_i, z_j) + o(\epsilon)
\end{equation}
因此,由 \eqref{eq:Ixy}式,
 $\norm{B_{ij}}_F^2 = \epsilon^2 \sum_{z_r \in \mathcal{Z}_i, z_s \in \mathcal{Z}_j} \phi^2_{Z_i Z_j}(z_r, z_s)$ 
 并且 式 \eqref{eq:PXV} 可以展开成
\begin{equation}\label{eq:PXV_Data}
D(P_{Z_V} || \prod_{C\in \P} P_{Z_{C}}) =
{\epsilon^2\over 2}\sum_{\substack{(i,j) \not\in C\\ C\in \P}}
\sum_{z_r \in \mathcal{Z}_i, z_s \in \mathcal{Z}_j}  \phi^2_{Z_i Z_j}(z_r, z_s) + o(\epsilon^2)
\end{equation}
我们可以把
$\phi^2_{Z_i Z_j}(z_r, z_s)$ 这一项
当成一个有$n$个节点的图的
边的权值。
图的每一个节点对应一个数据点。
为简化符号, 令 $w_{rs} = \phi^2_{Z_{d(z_r)}Z_{d(z_s)}}(z_r, z_s)$
\footnote{当 $d(z_r) = d(z_s)$ 时,
我们仍可以形式化的定义 $w_{rs}$ 为远大于$\max\{\phi^2_{Z_i Z_j}(z_r, z_s), i\neq j\}$
的值。},
其中,$d(z_r)$ 把节点映射到它所属的随机变量的序号, 定义域 为 $1\leq r,s \leq \abs{V}$。
我们也可以展开 每个 $Z_i$ 到它的节点集 并且将分割 $\P$
看成是对节点集的分割。
假设我们考虑的图 $G(V, E)$ 是有向的
\footnote{有向图的假设可以减少计算量而不失一般性},
类似 式 \ref{eq:IP} 我们定义图的入割函数 (in-cut function) $f(C)$ for $C\subseteq V$ as $f(C) = \sum_{i\not\in C, j\in C, (i,j) \in E} w_{ij}$,
它是所有进入$C$的有向边的权值之和。
基于上面定义的符号,表示KL散度的式 \eqref{eq:PXV_Data} 可以写成聚类簇之间边的权值之和
的形式:
\begin{equation}\label{eq:PXV_Data_Simplified}
D(P_{Z_V} || \prod_{C\in \P} P_{Z_{C}}) = \epsilon^2 \sum_{C \in \P} f(C)+ o(\epsilon^2)
\end{equation}
从而我们获得了在多变量弱独立条件下数据聚类的表达式。
由于$\epsilon$是给定的无穷小量,我们更关心$\epsilon^2$
的系数,该系数即与式\eqref{eq:IP}中的$f[\P]$一样,
也在我们求解等价的优化问题式\eqref{eq:hlambda}中出现过。
\section{与数据聚类问题的联系}
\label{sec:data_clustering}
社群发现是输入一张图获得节点所属的类别,而
数据聚类的任务与社群发现相同,但其输入是一个数据矩阵,
其中行数是数据的个数而
每一行的向量代表该数据的特征。二者可以通过
输入数据的格式转换来实现算法互通。一般而言,
从数据矩阵到图的变换是可以通过选取一
相似度度量$d(\cdot,\cdot)$作用到数据上
获得图中边的权值,即$w_{ij}=d(x_i, x_j)$。
在这一节中,
我们研究的重点是如何用
\eqref{eq:hlambda} 实现数据聚类的任务。
在永野清仁的文章中\cite{mac},已经有用RBF核作为相似度度量的
尝试,但局限于该相似度度量在小规模真实世界数据集上表
现不佳,此外相似度度量本身具有一些超参数,也会影响聚类
效果。我们通过使用交叉验证、网格搜索,及对
不同相似度度量的枚举,试图寻找特定问题下具有
较好表现的相似度度量及其超参数。

我们使用5种数据集对基于图分割的数据聚类方法
进行测试,各数据集的基本情况如表\ref{tab:clustering_dataset}
所示。
\begin{table}[!ht]
  \centering
  \begin{tabular}{|c|c|c|c|}
    \toprule
    名称 & 样本数量 & 类别数量 & 特征维度 \\
    \midrule
    Gaussian & 100 & 4 & 2 \\
    Circle & 300 & 3 & 2 \\
    Iris & 150 & 3 & 4 \\
    Glass & 214 & 6 & 9 \\
    Libras & 360 & 15 & 90 \\
    \bottomrule
  \end{tabular}
  \caption{数据聚类测试数据}\label{tab:clustering_dataset}
\end{table}
其中,Gaussian 和 Circle 是人工生成的数据集。
Iris, Glass 和 Libras 是来自 UCI 机器学习标准数据集
\cite{Dua:2019}。

首先我们展示在两个人工生成的数据集上的聚类效果,
Gaussian 数据集的相似度度量取 式\ref{eq:rbf_kernel} 所示的 RBF 核,
$\gamma=0.6$。
\begin{equation}\label{eq:rbf_kernel}
  w_{ij}=\exp(-\gamma ||x_i-x_j||^2)
\end{equation}
Circle 数据集的相似度度量通过 k近邻 进行计算,即仅与自己距离最近的$k$
个数据点有边相连\footnote{所有边权值均为1},取$k=7$。
聚类效果如图 \ref{fig:artificial_dataset_effect} 所示。

\begin{figure}[!ht]
  \begin{subfigure}[b]{\linewidth}
  \includegraphics[width=\textwidth]{4part.pdf}
  \caption{有4个类别的 Gaussian 数据集}
  \label{fig:4p}
\end{subfigure}
\begin{subfigure}[b]{\linewidth}
  \includegraphics[width=\textwidth]{3circle.pdf}
  \caption{有3个类别的 Circle 数据集}
  \label{fig:3c}
\end{subfigure}
\caption{人工生成的数据集聚类效果}
\label{fig:artificial_dataset_effect}
\end{figure}

从图中可以看到,通过选取适当的阈值 $\lambda$,
可获得与数据的真实类别十分接近的聚类结果,对应于图
\ref{fig:4p} 中的 $\lambda=0.42$
和图\ref{fig:3c} 中的 $\lambda=0.0$。

下面我们在表\ref{tab:clustering_dataset}
的五个数据集上测试比较不同的聚类算法的性能。
我们采用ARI\footnote{adjusted rand index,译为调整兰德系数}
作为衡量指标。该指标介于0到1之间,
其将聚类算法预测的各数据的类别与真实的类别
进行对比,数值越大说明聚类效果越好,等于1说明
两个类别完全一样。我们对比的算法有系统聚类法\cite{slink}、
近邻传播\cite{frey2007clustering}、k-平均算法
\cite{lloyd1982least}
和谱聚类 \cite{shi2000normalized}。
我们通过5轮交叉验证的方式选取各算法的超参数,
表\ref{tab:clustering_dataset}给出了在一定优化空间内
各算法最优的表现,
其对应的超参数详见附表\ref{tab:clustering_alg_hyperparameter}。

从表\ref{tab:clustering_dataset}
可以看出,
信息聚类的方法在 Gaussian, Cirlce, Glass 上表现最好,
在Iris 和 Libras 数据集上的表现不尽人意。
这与文献\citep{mac} 中信息聚类在各数据集上均表现最好的实验结果有出入,
我们由此推测信息聚类可能不适用于真实数据的聚类。

\begin{table}[!ht]
  \centering
\begin{tabular}{lrrrrr}
  \hline
   ARI  &   Gaussian &   Circle &   Iris &   Glass &   Libras \\
  \hline
   系统聚类法         &       1.00 &     1.00 &   0.76 &    0.21 &     0.29 \\
   近邻传播  &       1.00 &     0.14 &   0.19 &    0.19 &     0.23 \\
   信息聚类                  &       1.00 &     0.80 &   0.57 &    0.32 &     0.13 \\
   k-平均算法              &       1.00 &     0.01 &   0.73 &    0.17 &     0.33 \\
   谱聚类   &       1.00 &     0.60 &   0.76 &    0.14 &     0.37 \\
  \hline
\end{tabular}
\caption{聚类算法效果比较}\label{tab:clustering_dataset}
\end{table}

\section{社群的层次发现算法}
\label{subsec:cd}
尽管在发现社群的单层结构方面
有很多算法取得了成功,自动发现
复杂网络中的社群层次结构仍是一个困难的问题。
在本节中,我们用实验来证明PSP算法
在恰当定义“边的权重”后,可以成功恢复两层图的层次结构。

对于一个无权图,如果我们简单地把
所有边的权值赋成1,
我们只能得到平凡的聚类树,也即聚类树
只有根节点和叶节点。这个论断可以从如下
定理中得到:
\begin{theorem}\label{thm:triangle}
  在图$G$中,对于没有边相连的两个节点 $w_{ij}=0$。
  并且对于任意三元组$i,j,k \in V$ 权值满足三角不等式 
  $w_{ij} + w_{jk} \geq w_{ki}$,
  则该图的聚类树是平凡的。
\end{theorem}
  
我们通过实验发现使用下述的方法对无权图进行重新
赋权可以达到较理想的效果:
\begin{equation}\label{eq:wij_scheme}
    w_{ij} = 1 + \abs{\{k | (i,k),(j,k) \in E \}} \textrm{ for } (i,j) \in E
\end{equation}
式 \ref{eq:wij_scheme} 对$w_{ij}$
的赋权方法是计算从$i$到$j$不超过两跳的路径数量。

下面我们在一个具有两层结构的图上验证我们的
赋权方法。该数据集在文献
\cite{RN22} 中用于研究动态网络的同步行为,
其结构如图 \ref{fig:c1} 所示。 

\begin{figure}
	\centering
	\begin{subfigure}{0.45\textwidth}
		\includegraphics[width=\textwidth]{two_level.pdf}
		\caption{拓扑结构,参数取值为$z_{\mathrm{in}_1} = 14,$ $z_{\mathrm{in}_2} = 3, z_{\mathrm{out}}=1$.}\label{fig:c1}
	\end{subfigure}
	\begin{subfigure}{0.45\textwidth}
		\includegraphics[width=\textwidth]{tree_info-clustering.pdf}
		\caption{聚类树结构,可用PSP算法完全恢复出来。每个叶节点仍包含16个子节点未画出。}
    \label{fig:c2}
	\end{subfigure}
	\caption{具有两层结构的图示意}
\end{figure}

该两层结构的图在宏观层面它包含有 4个 社群,
而每一个中等规模的社群在微观层面又包含4个小社群,
每个小社群有16个节点。
我们用 $z_{\mathrm{in}_1}$ 表示
在小社群内部每个节点连接的边的平均数量;
用 $z_{\mathrm{in}_2}$ 表示
每个大社群内部的中等规模社群之间相互连接的边的平均数量;
用 $z_{\mathrm{out}}$ 表示
不同大社群之间相互连接的边的平均数量。
通过变化参数设定 $\{z_{\mathrm{in}_1}, z_{\mathrm{in}_2}, z_{\mathrm{out}} \}$
我们能得到不同疏密程度的图。

为比较通过算法推断出的聚类树的结构和真实聚类树的区别,
我们使用 归一化的 Robinson-Foulds 距离度量。
它描述了两个具有不同拓扑结构的树之间的距离,
其取值在$[0,1]$ 之间。
我们将基于图的信息聚类算法与
GN (Girvan-Newman 算法) 和 BHCD (一种贝叶斯层次聚类算法\cite{RN23})。
比较的结果如图 \ref{fig:cdr} 所示。
从图 \ref{fig:cdr}中可以看到,在三种不同的参数配置下,
通过基于图的信息聚类算法相比其他方法 可以获得
与真实聚类树更相似的聚类树结构。

\begin{figure}
	\centering
	\begin{subfigure}{0.33\textwidth}
		\includegraphics[width=\textwidth]{z_in_1.pdf}
		\caption{}
	\end{subfigure}~
	\begin{subfigure}{0.33\textwidth}
		\includegraphics[width=\textwidth]{z_in_2.pdf}
		\caption{}
	\end{subfigure}~
	\begin{subfigure}{0.33\textwidth}
		\includegraphics[width=\textwidth]{z_o.pdf}
		\caption{}
	\end{subfigure}
	\caption{不同层次社群发现算法的性能比较。
  更小的距离意味着与真实的聚类树更相似的结构。
  }\label{fig:cdr}	
\end{figure}

\section{基于图分割的社群发现算法及其改进}
在本节中,我们首先介绍我们对PSP算法的改进,并在一定
的假设下证明其理论复杂度,最后比较求解图分割的三种不同实现下
的算法效率。
\subsection{HPSP算法}
我们改进的PSP算法仍将使用
算法 \ref{alg:psp} 中提到的
\texttt{DT} 子函数。
在算法\ref{alg:psp} 中,我们观察到
每次调用 \texttt{DT} 是相互独立的,
并没有利用到聚类树的 固有性质。
如果聚类树的嵌套性质可以被考虑进来的话, 
我们可以在后面的计算中
仅在子图上进行 \texttt{DT} 的调用,
从而可以实现计算效率比PSP算法提升一个数量级。


具体来说,
假设我们第一次调用DT后得到了
$P_i = \{C_1, \dots, C_t\}$。
然后我们 分别对每个
$C_i(i=1,\dots, t)$
计算 PSP分割。
并从子图的计算中构造 $P_j(j>i)$。
对于 $P_j(j<i)$,
我们可以把 图 $G$
缩成 图 $G^t$,其中 每个$C_i$集合中的点缩成一个单独的节点。
因此图 $G^t$共有 $t$ 个节点。
通过在$G^t$ 上调用 DT 函数我们可以得到  $P_j(j<i)$。
我们把我们的改进方法称为 HPSP,其中 H 表示 “层次的”
英文首字母\footnote{hierarchical},
其描述在算法\ref{alg:psp_i_simplified}中给出。

\begin{algorithm}[!ht]
	\caption{改进的求解主分割序列的算法}\label{alg:psp_i_simplified}
	\begin{algorithmic}[1]
		\REQUIRE 有向图 $G(V, E)$; 边的权值函数 $w(e)$,其中 $e\in E$
		\ENSURE 聚类树 $\mathcal{T}(K, E)$ 其中 $K \subseteq 2^{V}$ 表示节点集
    而 $E$ 表示边的集合。
		\STATE 初始化 树 $\mathcal{T}$:
     $V$ 是根节点,
     $\{j\}(j \in V)$ 是诸叶节点,
     无其他节点。
		\STATE \texttt{Split}($G, V$)
		\FUNCTION{\texttt{Split}($\widetilde{G}, \widetilde{V}$)}
		\STATE $w$ 是 $\widetilde{G}$ 所有边的权值之和。
		\STATE $\gamma' = \frac{w}{\abs{V(\widetilde{G})}-1}$
    其中 $V(\widetilde{G})$ 是 图$\widetilde{G}$
    的节点集。
    \label{alg:gamma_apostrophe}
		\STATE $(\tilde{h}, P') = \texttt{DT}(\widetilde{G}, \gamma')$ 其中
    $\P'$ 是在 式 \eqref{eq:hlambda} 中达到
    最小值 $h(\gamma')$的分割
    并且 $\tilde{h}$ 是对应的最小值。 \label{line:DT}
		\IF{$\tilde{h} = - \gamma'$}
		\STATE 在聚类树
    $\mathcal{T}$ 中把权值 $\gamma'$ 标记在从 $\widetilde{V}$ 出发
    到其所有子节点的边上。
		\ELSE
		\FOR{$S$ in $P'$ and $\abs{S}>1$}
		\STATE 在聚类树
    $\mathcal{T}$ 中构造新的节点$S$,并修改 $\widetilde{V}$的子节点的父亲节点 为$S$,
    而$S$的父节点为$\widetilde{V}$。
		\STATE \texttt{Split}($\widetilde{G}[S], S$)
    其中 $\widetilde{G}[S]$ 是 $\widetilde{G}$ 限制在 $S$
    上的子图。\label{line:SplitDown}
		\STATE 在 $\widetilde{G}$ 中将 $S$ 缩成一个节点。 % graph \widetilde{G} is modified
		\ENDFOR 
		\STATE \texttt{Split}($\widetilde{G}, \widetilde{V}$)		\label{line:SplitUp}
		\ENDIF
		\ENDFUNCTION
	\end{algorithmic}
\end{algorithm}

\begin{figure}[!ht]
	\centering
	\includegraphics[width=10cm]{alg_illustration.pdf}
	\caption{HPSP 求解图(a)的主分割序列,聚类树经(b), (c) 演化到 (d) }\label{fig:alg_eg}
\end{figure}

\begin{example}
	我们用一个简单的例子来说明
  算法 \ref{alg:psp_i_simplified}。
  考虑图 $G(V, E)$,有 $V=\{1,2,3,4\}, E=\{(1,2),(1,3),(2,3),(1,4),(2,4)\}$。
  边的权值为 $w_{13}=2, w_{12}=5, w_{23}=2, w_{14}=1, w_{24}=1$。
	图 \ref{fig:alg_eg} (a) 中画出了 $G$
  的结构。
  最初的聚类树结构如图 \ref{fig:alg_eg} (b)
  所示。
  根据 
  算法 \ref{alg:psp_i_simplified} 的
  第\ref {alg:gamma_apostrophe}行,
  我们计算 $\gamma' = \frac{11}{4-1}, \tilde{h} = -\frac{16}{3} < -\gamma' $
  且 $\P' = \{\{1,2,3\},\{4\}\}$,
  于是我们得到如图\ref{fig:alg_eg} (c) 所示
  的聚类树结构。
	
	然后我们在 子图 $G[\{1,2,3\}]$ 上运行 PSP 算法,
  $\gamma' = \frac{9}{2}, \tilde{h} = -5 < -\gamma'$
  并且 $\P= \{\{1,2\},\{3\}\}$。 
  于是我们得到了最终的聚类树结构。
  其余的计算给出了聚类树剩余的边的权值,如
  图 \ref{fig:alg_eg} (d) 所示。
\end{example}	
\subsection{时间复杂度分析}

本小节我们分析上小节提出的 HPSP算法 (算法 \ref{alg:psp_i_simplified})的时间复杂度。
首次说明一些惯用语,我们称图是稠密,如果它的边的数量是$n^2$的量级。
也即 $\abs{V} = n, \abs{E} = O(n^2)$。对于稠密的图,由于最大流算法中需要多次遍历
边,会增大时间开销。一般PSP算法复杂度分析即针对最坏的情形,即稠密的图进行。
我们已知的结论有 \texttt{DT} 算法(算法\ref{alg:dt})
的时间复杂度是 $O(n^4)$, 而 PSP算法(\ref{alg:psp})的 
时间复杂度是 $O(n^5)$ \citep{pin}。

我们使用 $T(n)$ 来代表 
\texttt{Split} 函数调用的
时间复杂度。
从算法 \ref{alg:psp_i_simplified} 中可知,
$T(n)$ 的递推关系式为
\begin{equation}\label{eq:Tn}
T(n) = \max \{ C n^4 + \sum_{i=1}^k T(n_i) +
T(k)\delta_{k<n} |
\sum_{i=1}^k n_i = n, n_i \in \mathbb{Z}_{+} \}
\end{equation}
在 式\eqref{eq:Tn} 中,
$Cn^4$ 代表 \texttt{DT} 的时间复杂度。
$\delta_{k<n} = 1$ 是 指示函数,当满足条件 $k<n$时取值为1,
否则 $\delta_{k<n}=0$。
针对 $T(n)$,我们有如下定理:
\begin{theorem}\label{thm:alg_complexity}
	 若 $n_i \leq \frac{n}{2}, \textrm{ for } i=1,\dots,k$ 并且
   $ k \leq \frac{n}{2}$  在式 \eqref{eq:Tn} 中成立, 
   则 $T(n) = O(n^4)$.
\end{theorem}

定理 \ref{thm:alg_complexity} 所要求的条件
限制了聚类树的结构。
它指出,若对于每个非叶节点,其后代数都不超过 $\frac{n}{2}$,则HPSP
算法的时间复杂度是  $O(n^4)$。
该条件适用于很多聚类树的情形,即使有个别节点不满足其假设,$T(n) = O(n^4)$也成立。
我们在保证有相同计算结果的前提下,
提出的HPSP 的时间复杂度的结果要比
PSP算法 \ref{alg:psp} 的$O(n^5)$ 的结果快一个数量级。

即使是最坏的情况,也即,聚类树的深度达到了 $O(n)$ 的数量级。
则\texttt{DT} 不可避免地要调 用 $O(n)$ 次。
在这种情形下,
$T(n)$ 的复杂度的上界是 $C(3^4+4^4 + \dots + n^4) \sim \frac{1}{5}Cn^5$,
其比 算法 \ref{alg:psp} 的结果
快 $5$ 倍。

\subsection{不同PSP算法的效率比较}
在本节中,我们将比较 HPSP 算法和PSP算法
(算法\ref{alg:psp})
以及基于参数最大流改进方法\cite{kolmogorov}
的运行效率。我们使用两种不同的数据集,图的边具有不同的
稠密程度。

第一个数据集叫做 Gaussian-blobs,
类似\ref{sec:data_clustering}
节的 Gaussian 数据集,它也是由
4个相同大小的高斯块组成,但我们可以改变
每个块的样本数量。由于我们使用式\ref{eq:rbf_kernel}
计算图中边的权值,边的数量满足 $|E|=\Theta(|V|^2)$。
第二个数据集,two-level graph,类似 \ref{subsec:cd}节
的具有两层社群结构的图,一共有 $s^3$ 节点。
\ref{subsec:cd}节 相当于 $s=4$ 的情形。

我们使用CPU时间来衡量算法的时间复杂度。
通过改变节点的数量,
我们可以得到如图\ref{fig:esc}所示的实验结果,
可以看出,我们的实现比以前的实现要快得多。
\begin{figure}
	\centering
	\begin{subfigure}{0.45\textwidth}
		\includegraphics[width=\textwidth]{2019-08-26-gaussian.pdf}
	\end{subfigure}
	\begin{subfigure}{0.45\textwidth}
		\includegraphics[width=\textwidth]{2019-09-19-two_level.pdf}
	\end{subfigure}
	\caption{
  不同PSP实现的速度比较}\label{fig:esc}
\end{figure}

另外我们注意到,
基于参数化最大流的
算法\citep{kolmogorov}在实践中表现不佳。
这可能与计算过程中构造中间图结构
和维护并发线程的额外成本有关。在实际实现中,
并发线程的创建和销毁的时间开销是不可忽略的,而这一部分
在复杂度理论分析中考虑。

\section{基于图分割的社群发现算法在异常值检测领域的应用}
异常值检测是对异常数据(称为异常值点)的识别,
这些异常数据不同于其他多数数据(称为正常值点)
\citep{grubbs1969procedures}。
在本节中,我们考虑将信息聚类算法用到异常值检测领域。

首先我们在理论层面探讨异常值检测背景下信息聚类的应用。
假设一个简单的情形,所有正常值点属于同一类别,
假设集合$B$达到式\eqref{eq:largest_threshold}中的
最大值,
我们认为$B$中的元素是正常值点,而$V\backslash B$ 是异常值点。

我们这里探讨的问题是, 假设有一个新的点$i'$加入到
现有的图 $G$ 中,如何判断它是正常值点还是异常值点。 
事实上,对于加入$i'$的图来说,
我们可以用PSP算法计算一个新的最大临界值
$\gamma'_N$,
并把它与 $\gamma_N$作比较。 由式\eqref{eq:largest_threshold}易得
$\gamma'_N \geq \gamma_N$。如果不等号严格成立,我们看到$i'$的作用是使得
多变量互信息的最大值增大了,说明$i'$与图$G$结合得更紧密,是正常值点,否则为异常值点。
实际上,我们不需要重新计算$\gamma'_N$就可以做出相同的判断,这个理论保证由下面的
定理给出
\begin{theorem}\label{thm:main_N}
  假设 $B$ 关于图$G$ 达到式\eqref{eq:largest_threshold}中的
  最大值,$\gamma_N, \gamma'_N$ 分别是图$G$加入新的点$i'$前、后的最大临界值,则有
\begin{equation}
\gamma'_N > \gamma_N \iff  \sum_{i \in B} w_{ii'} > \gamma_N 
\end{equation}
\end{theorem}
定理 \ref{thm:main_N} 给出了一种划分正常值点边界的方法,
假设图$G$的第$i$个点
对应一个高维空间的向量$x_i$,新加入的点对应的向量是$x$,
边的权值由rbf核给出,则边界曲线的方程为:
\begin{equation}\label{eq:sum_exp_gamma_N}
  \sum_{j \in B} \exp(-\gamma \norm{x - x_j}^2)= \gamma_N
\end{equation}


然后我们在算法层面进行考虑,假设在一个多分类问题中存在若干异常值点,
我们的目标是获得数据的一个分割,其形式为
$\{C_1, \dots, C_K, \{x_1\}, \dots, \{x_M\}\}$。
这里 $K$ 表示类别数 而 $M$ 表示异常值点的个数。
这两个参数在实际问题中通常是未知的,这里我们也做此假定,
我们使用的方法是基于 PSP 算法来获得该形式与$K, M$的值。

如果 $K$ 和 $M$ 较小, 使用式 \eqref{eq:PSP_structure}
的记法, 处理这种情况的一种经验性但合理的方法是找到一种划分。
$\P_{k-r}$ from  $\{\P_k, \P_{k-1}, \dots, \P_0\}$ 使得$K,M$都比较小。
对于此多目标优化问题,我们需要一个折中方案,即通过枚举法$r=1,2,\dots$ 求解下面的优化问题
\begin{align}\label{eq:outlier_detection_proposal}
& \min_{r\in Z^+} r \\
s.t.\, & f(r):=(r+1)\abs{\P_{k-r}} < n
\end{align}
在式\eqref{eq:outlier_detection_proposal}中,我们注意到函数
$f(0)=|\P_k|=n$ 和 $f(k)=k+1<n$ ,可知优化问题有解。

然后我们用 $\P_{k-r}$ 中元素个数大于1的集合来标记各类别中的正常值点,
元素个数小于1的集合为异常值点的集合。

如果需要预测新的观测值是否是异常值点, 
我们使用类似于式\eqref{eq:sum_exp_gamma_N}给出的边界方程判决规则。
假设 $\lambda_{k-r}$ 是$\P_{k-r}, \P_{k-r+1}$ 两个分割对应的
分界值并且图$G$边的权值是使用 rbf 核 来构建的,
那么边界曲线方程可以写成下面的形式:
\begin{equation}
\sum_{j \in C_i} \exp(-\lambda \norm{x - x_j}^2)= \lambda_{k-r}, \textrm{ 其中 } \, C_i \in \P_{k-r} \textrm{且}\,  \abs{C_i} > 1
\end{equation}
注意到 $C_i$ 是其中一个类别,
并且 一共有 $K$ 个 边界曲线。
如果新的观测值落在所有这些边界曲线的外部,
那么我们将其归类为异常值点。

\section{代码实现}
尽管最大流算法包比较多,但我们尚未发现
求解层次分割(式\eqref{eq:PSP_structure})的开源实现。
因此我们使用 C++ 实现了跨平台的PSP算法\ref{alg:psp}
(内嵌迪尔沃思截断算法\ref{alg:dt}),并且
我们提供了Python编程语言的接口函数,并在 \url{pypi.org}
平台上以 pspartition
的名字发布,以供后面的研究者使用。
我们的实现使用 CMake 编译,依赖于第三方库 LEMON \cite{dezsHo2011lemon}。 
LEMON 库在我们的算法包中用于构建所需的图数据结构和最大流算法。
这里顺便提及的一点是 LEMON 库里使用的
最大流算法是基于最大标签选择策略的前置推送-标签重贴算法。
根据文献\citet{ahuja1997computational}的比较,该算法
的效率比起其他解最大流问题的算法有明显的优势。

在Python的接口函数方面,我们采用了CPython
跨语言编程的方式,将由 NetworkX \cite{SciPyProceedings_11} 构建的图
转换成用边表示的图传递到C++中的类构造函数中,
计算完成后再把C++中的集合转换成Python中的列表
进行返回。用于求解例 \ref{ex:psp}
的示例 Python 代码如下。
\begin{lstlisting}[language=Python]
  from pspartition import PsPartition
  a = [[0,1,1], [0,2,5], [1,2,1]] # a graph
  p = PsPartition(3, a) # 3 nodes
  p.run()  
  cv = p.get_critical_values()
  pl = p.get_partitions()
  print(cv)
  print(pl)
  \end{lstlisting}
\section{实验结果}


