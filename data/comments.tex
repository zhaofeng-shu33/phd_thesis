% !TeX root = ../thuthesis-example.tex

\begin{comments}
% \begin{comments}[name = {指导小组学术评语}]
% \begin{comments}[name = {Comments from Thesis Supervisor}]
% \begin{comments}[name = {Comments from Thesis Supervision Committee}]
    本文使用了信息论度量的分析方法揭示了极限情况下社团发现算法可达的最优恢复率,为算法的检测效果提供了理论保证。对于指导社团发现算法设计和优化算法性能具有重要参考价值和借鉴意义。

    本文也存在一些不足之处:本文对所研究的社团发现算法如何应用到实际的问题中讨论较少,对于使用基于信息论的方法比非信息论方法的具体优势的论证稍显不足。在这两方面可以进一步加强。
    
    论文书写符合规范,结构合理,图表展示条理清晰,语言流畅,内容翔实,实验设计合理,数据资料充分,对相关文献的引述完整、正确。
    
    论文工作表明作者对社团发现和信息论工具有坚实的掌握,具有思辨能力和独立从事科学研究及撰写论文的能力。
\end{comments}
