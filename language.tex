\documentclass{ctexart}
\begin{document}
我们把HPSP算法应用到自然语言的层次聚类问题。
一方面我们通过这个小例子展示HPSP在社群发现实际数据上的效果,
另一方面语言的分类对印欧语言研究较多,已有完善的标准。 
而对其他具体某一地区方言和少数民族语言研究较少,使用
算法自动聚类对于辅助语言学家在后者发力也有裨益 \cite{nasution2019visualizing}。

我们使用的数据是词汇相似性数据,该度量是一个百分比,表示两种语言中有相同词意和书写方式的词汇比例
\cite{bella2021database}。
我们从Bella提供的数据中选取了67种常用语言,       


\cite{al2017characterization} 用了普通的聚合聚类的方式对10种常用语言进行聚类。


\bibliographystyle{plain}
\bibliography{ref/refs.bib}

\end{document}
