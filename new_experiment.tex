\documentclass{ctexart}
\usepackage{amsmath}
\usepackage{amssymb}
\usepackage{adjustbox}
%\xeCJKsetup{CJKecglue={}}
\begin{document}
我们把本节提出的基于玻茨模型的算法应用于实际的网络中。
为了和经典的最大模块度等方法进行公平比较,
我们取超参$\gamma=\gamma_{\mathrm{MQ}}$(参见式\eqref{eq:Qtransform}),
使得$\gamma>b$的条件满足,并且能量函数的形式固定下来。
由式\eqref{eq:e_1},$\gamma = \frac{2|E|}{n \log n}$,
故$\gamma$可由输入网络计算。
通过算法 \ref{alg:m} 近似求解能量函数最大值,与模拟退火的优化策略不同,
我们的算法使用的逆温度参数不随
迭代次数而改变。
我们考虑 Karate\cite{zachary1977information},
American Football \cite{girvan2002community},
Dolphin \cite{lusseau2003emergent}
和 Polbooks \cite{newman2006modularity}
四个常用于社团发现算法测试的数据集。
这四个数据集均为有真实标签的无向图。
并使用归一化的互信息\cite{Danon_2005}作为算法评价指标。
我们对比的算法有
Greedy GN \cite{clauset2004finding},
Fluid \cite{pares2018fluid},
Louvain \cite{blondel2008fast},
标签传播\cite{cordasco2010community}。
结果如表 \ref{tab:flatten_result} 所示。
从该表可以看出,我们的算法可以精确恢复 Karate 数据的社团结构,
在其他数据集上也取得不逊于其他算法的结果。

\begin{table}
    \begin{adjustbox}{width=\columnwidth,center}
    \begin{tabular}{cccccc}
    \hline
    NMI               & Greedy GN & Fluid & Louvain  & label propagation & potts\\
    \hline
    Karate            & 0.692     & 0.837 & 0.587   & 0.445         & 1.000     \\
    American Football & 0.698     & 0.851 & 0.857    & 0.870         & 0.870    \\
    Dolphin           & 0.573     & 0.655 & 0.516    & 0.527        & 0.701     \\
    Polbooks          & 0.531     & 0.411 & 0.493  & 0.534          & 0.580     \\
    \hline
    \end{tabular}
\end{adjustbox}
    \caption{}\label{tab:flatten_result}
\end{table}

\bibliographystyle{plain}
\bibliography{ref/refs.bib}

\end{document}
