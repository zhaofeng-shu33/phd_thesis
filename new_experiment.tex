\documentclass{ctexart}
\usepackage{amsmath}
\usepackage{amssymb}
%\xeCJKsetup{CJKecglue={}}
\begin{document}
\begin{table}
    我们把本节提出的基于玻茨模型的算法应用于实际的网络中。
    为了和经典的最大模块度等方法进行公平比较,
    我们取超参$\gamma=\gamma_{\mathrm{MQ}}$(参见式\eqref{eq:Qtransform}),
    使得$\gamma>b$的条件满足,并且能量函数的形式固定下来。
    通过算法 \ref{alg:m} 近似求解能量函数最大值,与模拟退火的优化策略不同,
    我们的算法使用的逆温度参数不随
    迭代次数而改变。
    我们考虑 四个数据集。这四个数据集均为有真实标签的无向图。
    并使用归一化的互信息\cite{Danon_2005}作为算法评价指标。
    我们对比的算法有。结果如表 \ref{tab:flatten_result} 所示。
    从该表可以看出,我们的算法可以精确恢复 Karate 数据的社团结构,
    在其他
    \begin{tabular}{cccccc}
    NMI               & Greedy GN & Fluid & Louvain  & label propagation & potts\\
    Karate            & 0.692     & 0.837 & 0.587   & 0.445         & 1.000     \\
    American Football & 0.698     & 0.851 & 0.857    & 0.870         & 0.870    \\
    Dolphin           & 0.573     & 0.655 & 0.516    & 0.527        & 0.701     \\
    Polbooks          & 0.531     & 0.411 & 0.493  & 0.534          & 0.580     
    \end{tabular}
    \caption{}\label{tab:flatten_result}
\end{table}

\bibliographystyle{plain}
\bibliography{ref/refs.bib}

\end{document}
